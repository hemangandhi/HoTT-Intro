\documentclass{article}
\usepackage[utf8]{inputenc}
\usepackage{amsmath, amssymb}
\usepackage{tikz}
\usepackage{wrapfig}
\usepackage[margin=15mm]{geometry}

\newcommand{\U}{\mathcal{U}}
\renewcommand{\l}{\lambda}
\newcommand{\rec}{\text{rec}}

\begin{document}
\begin{center}
\subsection*{Notes on W-types}
\end{center}

Heman Gandhi
\hfill
April 2020\\

\subsubsection*{Lists}

From 5.1, we says lists are:
\begin{itemize}
\item $nil: List(A)$
\item $cons: A \to List(A) \to List(A)$
\end{itemize}

With W-types, this is simplified to $List(A) :\equiv W_{x: 1 + A}\ \rec_{1 + A}(\U, 0, \l a. 1, x)$.
This means that lists are either a null-ary empty thing or something that takes one (1)
argument. This seems at odds with ``cons'' above, but is reconciled by ``sup'':
$cons(a, l) = sup(inr(a), \l \star. l)$.

If we have a list $<a, b, c>: List(A)$, we can write:
\[
sup(inr(a), \l \star. sup(inr(b), \l \star. sup(inr(c), \l \star. sup(inl(\star), \l x. empty))))
\]
which is suitably ugly and where $empty$ is the 0-induction to get us the nil at the end of the list.

\subsubsection*{Binary Trees}

With bullet-points, we'd probably write:
\begin{itemize}
\item $nil: BinTree(A)$
\item $node: A \to BinTree(A) \to BinTree(A) \to BinTree(A)$
\end{itemize}

With W-types, we'd get something very similar to lists:
\[
BinTree(A) :\equiv W_{a: 1 + A}\ \rec_{1 + A}(\U, 0, 2, a)
\]

Where the sort of ``cons''-ing operator would combine two trees under a root with:
\begin{equation}
\begin{split}
cons&'(a, l, r) = sup(inr(a), C)\\
&where\ C\ 0_2 = l;\ C\ 1_2 = r
\end{split}
\end{equation}

(Writing out an actual tree is hecking ugly.)

\subsubsection*{Proofs over W-types}

(I don't think this bit actually matters -- will just grab from the book.)

\subsubsection*{All Proofs of the Same Thing Over a W-type are the Same}

``Proofs of the same thing'' means that $g, h: \prod_{w: W_{x: A} B(X)} E(w)$
and where $t = g, h$ have $G_t: \prod_{a, f} t(sup(a, f)) = e(a, f, \l b. t(f(b)))$.

In order to prove this, we consider our own dependent type over our W-type. We put
$D: \prod_{w: W_{x: A} B(x)} g(w) = h(w)$ which would prove $g = h$ by function extensionality.
In order to prove this, by induction on W-types, we need only inhabit
\[
d: \prod_{a: A} \prod_{f: B(a) \to W_{x: A} B(x)} \prod_{g': \prod_{b: B(a)} D(f(b))} D(sup(a, f))
\]
To do so, consider $e$ above. $g'$ gives us that $g(f(b)) = h(f(b))$ and then function
extensionality makes $p: \l b. g(f(b)) = \l b. h(f(b))$. $ap_{e(a, f)}(p)$ gives us that
$e(a, f, \l b. g(f(b))) = e(a, f, \l b. h(f(b)))$ and by the given $G_t$, we get
$g(sup(a, f)) = h(sup(a, f))$, which gives us $d$ as desired.

\end{document}
