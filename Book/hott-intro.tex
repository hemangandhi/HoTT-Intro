% arara: makechapters: {items: [syllabus, dtt, pi, inductive, identity, equivalences, contractible, fundamental, hierarchy, funext, pullback, univalence]}

\documentclass[10pt]{memoir} %[ebook,10pt,oneside]
  
\usepackage{hott}

\title{Introduction to homotopy type theory}
\author{Egbert Rijke}
\date{2019}%\\Version: \today}
%\address{Carnegie Mellon University}
%\email{erijke@andrew.cmu.edu}

\pretitle{\begin{center}\textsc\bgroup\LARGE}
\posttitle{\egroup\end{center}\vspace{2cm}}
\preauthor{\begin{center}\textsc\bgroup\Large}\postauthor{\egroup\end{center}\vfill}
\predate{\begin{center}\textsc\bgroup}{\postdate{\egroup\end{center}}

% The following is to avoid overfull hboxes in the table of contents.
% https://tex.stackexchange.com/questions/49887/overfull-hbox-warning-for-toc-entries-when-using-memoir-documentclass
%\renewcommand*{\cftdotsep}{1}
\setpnumwidth{2em}
\setrmarg{3em}
\setlength{\cftchapternumwidth}{2em}
\setlength{\cftsectionindent}{2em}
\setlength{\cftsectionnumwidth}{2em}
\setlength{\cftsubsectionindent}{4em}
\setlength{\cftsubsectionnumwidth}{3em}

% We number sections independently of chapters, and subsections will be
% numbered too.

\counterwithout{section}{chapter}
\settocdepth{subsection}

% We set up the exercise environment, which produces list environment in a new
% unnumbered subsection that also gets mentioned in the table of contents.

\newtotcounter{exercisecounter}
\newcommand{\exercise}{\item\stepcounter{exercisecounter}}

\newlist{exenum}{enumerate}{1}
\setlist[exenum]{noitemsep,label=\thesection.\arabic*}
  %,ref=\thechapter.\arabic*}

\crefname{exenumi}{Exercise}{Exercises}
  
\newlist{subexenum}{enumerate}{1}
\setlist[subexenum]{noitemsep,label=(\alph*),ref=\theexenumi.\alph*}
\crefname{subexenumi}{Exercise}{Exercises}
  
\newenvironment{exercises}
{%
\subsection*{Exercises}%
\addcontentsline{toc}{subsection}{Exercises}%
\sectionmark{Exercises}%
\begin{exenum}}
{%
%\addtocounter{exercisecounter}{\theexenumi}%
\end{exenum}}


%\makeatletter
%\AtEndDocument{\write\@auxout{\protect\total@exercises{\arabic{exercisecounter}}}}
%\def\total@exercises#1{\global\def\totalexercises{#1}}
%\total@exercises{0}
%\makeatother
      
\addbibresource{bibliography.bib}

\makeindex

\begin{document}

\begin{titlingpage}
  \maketitle 
\end{titlingpage}

\mbox{}
\vfill
Total number of exercises: \total{exercisecounter}

\bigskip
The author gratefully acknowledges the support of the Air Force Office of Scientific Research through MURI grant FA9550-15-1-0053.

\bigskip
\doclicenseThis\thispagestyle{empty}

\cleardoublepage

\frontmatter

\tableofcontents

%\include{intro}

\begin{comment}
\chapter{Introduction}

To include introduction:
\begin{enumerate}
\item What are types in mathematics. Dependent types and dependent functions are everywhere in mathematics.
\item Why univalent foundations. Why should homotopy be in the foundation of mathematics
\item Constructive nature of homotopy type theory. Discuss differences with set theory.
\item What this course is about
\item How to use this book
\item Formal type theory versus informal type theory
\item Mention the formalization
\end{enumerate}

\begin{rmk}
  One difference between set theory and type theory is that every well-formed term is specified along with its type and with its context. One way of looking at this is that there are three sorts in type theory: contexts, types, and terms. On the other hand, there is only one sort in set theory: sets. Sets are governed by the elementhood relation: the formula $x\in y$ is a well-formed formula of set theory for any two sets $x$ and $y$. In particular, for a given set $x$, the formula $x\in y$ can be true for many sets $y$, which is very different to the situation in type theory, where every term is assigned a unique type.
  
  Another important difference between set theory and type theory is that set theory is formulated in the language of first order logic, whereas type theory is its own deductive system, not making use of any ambient logic. We will see in the present chapter and in the next few chapters what this deductive system looks like.
\end{rmk}
\end{comment}

\mainmatter

\renewcommand{\thechapter}{\Roman{chapter}}
\setsecnumdepth{subsection}

\chapter{Martin-L\"of's dependent type theory}

In this first chapter we explain what dependent type theory is. We begin with the structural rules of dependent type theory. The important concepts here are contexts, types in context, and terms in context, and the concept of judgmental equality. The structural rules contain rules for substitution and weakening, but they do not yet contain rules for forming new types. The informed reader may recognize that the rules we present are those of Voevodsky's \emph{B-systems}, which are equivalent to Cartmell's \emph{contextual categories}.
\section{Dependent type theory}
\label{ch:dtt}

\index{dependent type theory|(}
Dependent type theory is a system of inference rules that can be combined to make \emph{derivations}. In these derivations, the goal is often to construct a term of a certain type. Such a term can be a function if the type of the constructed term is a function type; a proof of a property if the type of the constructed term is a proposition; an identification if the type of the constructed term is an identity type, and so on. In some respect, a type is just a collection of mathematical objects and constructing terms of a type is the everyday mathematical task or challenge. The system of inference rules that we call type theory offers a principled way of engaging in mathematical activity.

\subsection{Judgments and contexts in type theory}\label{sec:judgments}
\index{judgment|(}
\index{context|(}
An \define{inference rule}\index{inference rule|see {rule}} is an expression of the form
\begin{prooftree}
  \AxiomC{$\mathcal{H}_1$\quad $\mathcal{H}_2$ \quad \dots \quad $\mathcal{H}_n$}
  \UnaryInfC{$\mathcal{C}$}
\end{prooftree}
containing above the horizontal line\index{horizontal line|see {inference rule}} a finite list $\mathcal{H}_1$, $\mathcal{H}_2$, \dots, $\mathcal{H}_n$ of \emph{judgments} for the hypotheses\index{inference rule!hypotheses}, and below the horizontal line a single judgment $\mathcal{C}$ for the conclusion\index{inference rule!conclusion}. A very simple example that we will encounter in \cref{ch:pi} when we introduce function types\index{function type}, is the inference rule
\begin{prooftree}
  \AxiomC{$\Gamma\vdash a:A$}
  \AxiomC{$\Gamma\vdash f:A\to B$}
  \BinaryInfC{$\Gamma\vdash f(a):B$}
\end{prooftree}
This rule asserts that in any context $\Gamma$ we may use a term $a:A$ and a function $f:A\to B$ to obtain a term $f(a):B$. Each of the expressions
\begin{align*}
  \Gamma & \vdash a :A \\
  \Gamma & \vdash f : A \to B \\
  \Gamma & \vdash f(a):B
\end{align*}
are examples of judgments. There are four kinds of judgments in type theory:
\begin{enumerate}
\item \emph{$A$ is a (well-formed) \define{type} in context $\Gamma$.}
  \index{well-formed type}\index{type}
  The symbolic expression for this judgment is\index{Gamma turnstile A type@{$\Gamma\vdash A~\type$}}\index{judgment!Gamma turnstile A type@{$\Gamma\vdash A~\type$}}
  \begin{equation*}
    \Gamma\vdash A~\type
  \end{equation*}
\item \emph{$A$ and $B$ are \define{judgmentally equal types} in context $\Gamma$.}
  \index{judgmental equality!of types} The symbolic expression for this judgment is\index{Gamma turnstile A is B type@{$\Gamma\vdash A\jdeq B~\type$}}\index{judgment!Gamma turnstile A is B type@{$\Gamma\vdash A\jdeq B~\type$}}
  \begin{equation*}
    \Gamma\vdash A \jdeq B~\type
  \end{equation*}
\item \emph{$a$ is a (well-formed) \define{term}\index{well-formed term}\index{term} of type $A$ in context $\Gamma$.} The symbolic expression for this judgment is\index{Gamma turnstile a in A@{$\Gamma\vdash a:A$}}\index{judgment!Gamma turnstile a in A@{$\Gamma\vdash a:A$}}
  \begin{equation*}
    \Gamma \vdash a:A
  \end{equation*}
\item \emph{$a$ and $b$ are \define{judgmentally equal terms} of type $A$ in context $\Gamma$.}\index{judgmental equality!of terms} The symbolic expression for this judgment is\index{Gamma turnstile a is b in A@{$\Gamma\vdash a\jdeq b:A$}}\index{judgment!Gamma turnstile a is b in A@{$\Gamma\vdash a\jdeq b:A$}}
  \begin{equation*}
    \Gamma\vdash a\jdeq b:A
  \end{equation*}
\end{enumerate}
Thus we see that any judgment is of the form $\Gamma\vdash\mathcal{J}$, consisting of a context $\Gamma$ and an expression $\mathcal{J}$ asserting that $A$ is a type, that $A$ and $B$ are equal types, that $a$ is a term of type $A$, or that $a$ and $b$ are equal terms of type $A$. The role of a context is to declare what hypothetical terms\index{hypothetical term} are assumed, along with their types. More formally, a \define{context} is an expression of the form
\begin{equation}\label{eq:context}
x_1:A_1,~x_2:A_2(x_1),~\ldots,~x_n:A_n(x_1,\ldots,x_{n-1})
\end{equation}
satisfying the condition that for each $1\leq k\leq n$ we can derive, using the inference rules of type theory, that
\begin{equation}\label{eq:context-condition}
  x_1:A_1,~x_2:A_2(x_1),~\ldots,~x_{k-1}:A_{k-1}(x_1,\ldots,x_{k-2})\vdash A_k(x_1,\ldots,x_{k-1})~\type.
\end{equation}
In other words, to check that an expression of the form \cref{eq:context} is a context, one starts on the left and works their way to the right verifying that each hypothetical term $x_k$ is assigned a well-formed type. Hypothetical terms are commonly called \define{variables}\index{variable}, and we say that a context as in \cref{eq:context} \define{declares the variables}\index{variable declaration} $x_1,\ldots,x_n$. We may use variable names other than $x_1,\ldots,x_n$, as long as no variable is declared more than once.

The condition in \cref{eq:context-condition} that each of the hypothetical terms is assigned a well-formed type, is checked recursively. Note that the context of length $0$ satisfies the requirement in \cref{eq:context-condition} vacuously. This context is called the \define{empty context}\index{context!empty context}\index{empty context}. An expression of the form $x_1:A_1$ is a context if and only if $A_1$ is a well-formed type in the empty context. Such types are called \define{closed types}\index{closed type}\index{type!closed type}. We will soon encounter the type $\N$ of natural numbers\index{natural numbers}, which is an example of a closed type\index{natural numbers!is a closed type}. There is also the notion of \define{closed term}\index{closed term}\index{term!closed term}, which is simply a term in the empty context. The next case is that an expression of the form $x_1:A_1,~x_2:A_2(x_1)$ is a context if and only if $A_1$ is a well-formed type in the empty context, and $A_2(x_1)$ is a well-formed type, given a hypothetical term $x_1:A_1$. This process repeats itself for longer contexts.

It is a feature of \emph{dependent} type theory that all judgments are context-dependent, and indeed that even the types of the variables may depend on any previously declared variables. For example, when we introduce the \emph{identity type}\index{identity type} in \cref{chap:identity}, we make full use of the machinery of type dependency, as is clear from how they are introduced:
\begin{prooftree}
  \AxiomC{$\Gamma\vdash A~\type$}
  \UnaryInfC{$\Gamma,x:A,y:A\vdash x=y~\type$}
\end{prooftree}
This rule asserts that given a type $A$ in context $\Gamma$, we may form a type $x=y$ in context $\Gamma,x:A,y:A$. Note that in order to know that the expression $\Gamma,x:A,y:A$ is indeed a well-formed context, we need to know that $A$ is a well-formed type in context $\Gamma,x:A$. This is an instance of \emph{weakening}\index{weakening}, which we will describe shortly.

In the situation where we have
\begin{equation*}
  \Gamma,x:A\vdash B(x)~\type,
\end{equation*}
we say that $B$ is a \define{family}\index{family}\index{type family} of types over $A$ in context $\Gamma$. Alternatively, we say that $B(x)$ is a type \define{indexed}\index{indexed type}\index{type!indexed} by $x:A$, in context $\Gamma$. Similarly, in the situation where we have
\begin{equation*}
  \Gamma,x:A\vdash b(x):B(x),
\end{equation*}
we say that $b$ is a \define{section}\index{section of a family} of the family $B$ over $A$ in context $\Gamma$. Alternatively, we say that $b(x)$ is a term of type $B(x)$, \define{indexed}\index{indexed term}\index{term!indexed} by $x:A$ in context $\Gamma$. Note that in the above situations $A$, $B$, and $b$ also depend on the variables declared in the context $\Gamma$, even though we have not explicitly mentioned them. It is common practice to not mention every variable in the context $\Gamma$ in such situations.

\index{judgment|)}
\index{context|)}

\subsection{Inference rules}\label{sec:rules}

In this section we present the basic inference rules of dependent type theory. Those rules are valid to be used in any type theoretic derivation. There are only four sets of inference rules:
\begin{enumerate}
\item Rules for judgmental equality 
\item Rules for substitution
\item Rules for weakening
\item The ``variable rule''
\end{enumerate}

\subsubsection*{Judgmental equality}

\index{rules!for type dependency!rules for judgmental equality|(}
In this set of inference rules we ensure that judgmental equality (both on types and on terms) are equivalence relations, and we make sure that in any context $\Gamma$, we can change the type of any variable to a judgmentally equal type.

\begin{samepage}
The rules postulating that judgmental equality on types and on terms is an equivalence relation are as follows\index{judgmental equality!is an equivalence relation}:
\begin{center}
%\begin{small}
\begin{minipage}{.2\textwidth}
\begin{prooftree}
\AxiomC{$\Gamma\vdash A~\textrm{type}$}
\UnaryInfC{$\Gamma\vdash A\jdeq A~\textrm{type}$}
\end{prooftree}
\end{minipage}
\begin{minipage}{.25\textwidth}
\begin{prooftree}
\AxiomC{$\Gamma\vdash A\jdeq A'~\textrm{type}$}
\UnaryInfC{$\Gamma\vdash A'\jdeq A~\textrm{type}$}
\end{prooftree}
\end{minipage}
\begin{minipage}{.5\textwidth}
\begin{prooftree}
\AxiomC{$\Gamma\vdash A\jdeq A'~\textrm{type}$}
\AxiomC{$\Gamma\vdash A'\jdeq A''~\textrm{type}$}
\BinaryInfC{$\Gamma\vdash A\jdeq A''~\textrm{type}$}
\end{prooftree}
\end{minipage}
\\*
\bigskip
\begin{minipage}{.2\textwidth}
\begin{prooftree}
\AxiomC{$\Gamma\vdash a:A$}
\UnaryInfC{$\Gamma\vdash a\jdeq a : A$}
\end{prooftree}
\end{minipage}
\begin{minipage}{.25\textwidth}
\begin{prooftree}
\AxiomC{$\Gamma\vdash a\jdeq a':A$}
\UnaryInfC{$\Gamma\vdash a'\jdeq a: A$}
\end{prooftree}
\end{minipage}
\begin{minipage}{.5\textwidth}
\begin{prooftree}
\AxiomC{$\Gamma\vdash a\jdeq a' : A$}
\AxiomC{$\Gamma\vdash a'\jdeq a'': A$}
\BinaryInfC{$\Gamma\vdash a\jdeq a'': A$}
\end{prooftree}
\end{minipage}
%\end{small}
\end{center}
\end{samepage}

\bigskip
Apart from the rules postulating that judgmental equality is an equivalence relation, there are also \define{variable conversion rules}\index{judgmental equality!conversion rules}\index{variable conversion rules}\index{conversion rule!variable}\index{rules!for type dependency!variable conversion}.
Informally, these are rules stating that if $A$ and $A'$ are judgmentally equal types in context $\Gamma$, then any valid judgment in context $\Gamma,x:A$ is also a valid judgment in context $\Gamma,x:A'$. In other words: we can convert the type of a variable to a judgmentally equal type.

The first variable conversion rule states that
\begin{prooftree}
\AxiomC{$\Gamma\vdash A\jdeq A'~\textrm{type}$}
\AxiomC{$\Gamma,x:A,\Delta\vdash B(x)~\type$}
\BinaryInfC{$\Gamma,x:A',\Delta\vdash B(x)~\type$}
\end{prooftree}
In this conversion rule, the context of the form $\Gamma,x:A,\Delta$ is just any extension of the context $\Gamma,x:A$, i.e., a context of the form
\begin{equation*}
  x_1:A_1,\ldots,x_{n-1}:A_{n-1},x:A,x_{n+1}:A_{n+1},\ldots,x_{n+m}:A_{n+m}.
\end{equation*}

Similarly, there are variable conversion rules for judgmental equality of types, for terms, and for judgmental equality of terms. To avoid having to state essentially the same rule four times, we state all four variable conversion rules at once using a \emph{generic judgment} $\mathcal{J}$, which can be any of the four kinds of judgments.
\begin{prooftree}
\AxiomC{$\Gamma\vdash A\jdeq A'~\textrm{type}$}
\AxiomC{$\Gamma,x:A,\Delta\vdash \mathcal{J}$}
\BinaryInfC{$\Gamma,x:A',\Delta\vdash \mathcal{J}$}
\end{prooftree}
An analogous \emph{term conversion rule}, stated in \cref{ex:term_conversion}, converting the type of a term to a judgmentally equal type, is derivable using the rules for substitution and weakening, and the variable rule.
\index{rules!for type dependency!rules for judgmental equality|)}

\subsubsection*{Substitution}
\index{substitution|(}\index{rules!for type dependency!rules for substitution|(}
If we are given a term $a:A$ in context $\Gamma$, then for any type $B$ in context $\Gamma,x:A,\Delta$ we can simultaneously substitute $a$ for all occurences of the variable $x$ in $\Delta$ and in $B$, to obtain a type $B[a/x]$ in context $\Gamma,\Delta[a/x]$. You are already familiar with simultaneous substitution, e.g., substituting $0$ for $x$ in the polynomial
\begin{equation*}
  1+x+x^2+x^3
\end{equation*}
resuts in the number $1+0+0^2+0^3$, which can be computed to the value $1$. 

Type theoretic substitution is similar. In a bit more detail, suppose we have well-formed type
\begin{equation*}
  x_1:A_1,\ldots,x_{n-1}:A_{n-1},x_n:A_n,x_{n+1}:A_{n+1},\ldots,x_{n+m}:A_{n+m}\vdash B~\textrm{type}
\end{equation*}
and a term $a:A_n$ in context $x_1:A_1,\ldots,x_{n-1}:A_{n-1}$. Then we can form the type
\begin{equation*}
  x_1:A_1,\ldots,x_{n-1}:A_{n-1},x_{n+1}:A_{n+1}[a/x_n],\ldots,x_{n+m}:A_{n+m}[a/x_n]\vdash B[a/x_n]~\textrm{type}
\end{equation*}
by substituting $a$ for all occurences of $x_n$. Note that the variables $x_{n+1},\ldots,x_{n+m}$ are assigned new types after performing the substitution of $a$ for $x_n$.

This operation of substituting $a$ for $x$ is understood to be defined recursively over the length of $\Delta$. When $B$ is a family of types over $A$ and $a:A$, we also say that $B[a/x]$ is the \define{fiber}\index{family!fiber of}\index{fiber!of a family} of $B$ at $a$. We will usually write $B(a)$ for $B[a/x]$. Similarly we obtain for any term $b:B$ in context $\Gamma,x:A,\Delta$ a term $b[a/x]:B[a/x]$. The term $b[a/x]$ is called the \define{value} of $b$ at $a$. When we substitute in a judgmental equality, either of types or terms, we simply subtitute on both sides of the equation.

We can now postulate the \define{substitution rule} as follows:
\begin{prooftree}
\AxiomC{$\Gamma\vdash a:A$}
\AxiomC{$\Gamma,x:A,\Delta\vdash \mathcal{J}$}
\RightLabel{$S$}
\BinaryInfC{$\Gamma,\Delta[a/x]\vdash \mathcal{J}[a/x]$}
\end{prooftree}
In other words, the substitution rule asserts that substitution preserves well-formedness and judgmental equality of types and terms. Furthermore, we postulate that substitution by judgmentally equal terms results in judgmentally equal types
\begin{prooftree}
\AxiomC{$\Gamma\vdash a\jdeq a':A$}
\AxiomC{$\Gamma,x:A,\Delta\vdash B~\type$}
\BinaryInfC{$\Gamma,\Delta[a/x]\vdash B[a/x]\jdeq B[a'/x]~\type$}
\end{prooftree}
and it also results in judgmentally equal terms
\begin{prooftree}
\AxiomC{$\Gamma\vdash a\jdeq a':A$}
\AxiomC{$\Gamma,x:A,\Delta\vdash b:B$}
\BinaryInfC{$\Gamma,\Delta[a/x]\vdash b[a/x]\jdeq b[a'/x]:B[a/x]$}
\end{prooftree}
To see that these rules make sense, we observe that both $B[a/x]$ and $B[a'/x]$ are types in context $\Delta[a/x]$, provided that $a\jdeq a'$. This is immediate by recursion on the length of $\Delta$.
\index{substitution|)}\index{rules!for type dependency!rules for substitution|)}

\subsubsection*{Weakening}
\index{weakening|(}\index{rules!for type dependency!rules for weakening|(}
If we are given a type $A$ in context $\Gamma$, then any judgment made in a longer context $\Gamma,\Delta$ can also be made in the context $\Gamma,x:A,\Delta$, for a fresh variable $x$. The \define{weakening rule}\index{weakening} asserts that weakening by a type $A$ in context preserves well-formedness and judgmental equality of types and terms.
\begin{prooftree}
\AxiomC{$\Gamma\vdash A~\textrm{type}$}
\AxiomC{$\Gamma,\Delta\vdash \mathcal{J}$}
\RightLabel{$W$}
\BinaryInfC{$\Gamma,x:A,\Delta \vdash \mathcal{J}$}
\end{prooftree}
This process of expanding the context by a fresh variable of type $A$ is called \define{weakening} (by $A$).

In the simplest situation where weakening applies, we have two types $A$ and $B$ in context $\Gamma$. Then we can weaken $B$ by $A$ as follows
\begin{prooftree}
  \AxiomC{$\Gamma\vdash A~\textrm{type}$}
  \AxiomC{$\Gamma\vdash B~\textrm{type}$}
  \RightLabel{$W$}
  \BinaryInfC{$\Gamma,x:A\vdash B~\type$}
\end{prooftree}
in order to form the type $B$ in context $\Gamma,x:A$. The type $B$ in context $\Gamma,x:A$ is called the \define{constant family}\index{family!constant family}\index{constant family} $B$, or the \define{trivial family}\index{family!trivial family}\index{trivial family} $B$.
\index{weakening|)}\index{rules!for type dependency!rules for weakening|)}

\subsubsection*{The variable rule}
If we are given a type $A$ in context $\Gamma$, then we can weaken $A$ by itself to obtain that $A$ is a type in context $\Gamma,x:A$. The \define{variable rule}\index{variable rule}\index{rules!for type dependency!variable rule} now asserts that any hypothetical term $x:A$ in context $\Gamma$ is a well-formed term of type $A$ in context $\Gamma,x:A$.
\begin{prooftree}
\AxiomC{$\Gamma\vdash A~\textrm{type}$}
\RightLabel{$\delta$}
\UnaryInfC{$\Gamma,x:A\vdash x:A$}
\end{prooftree}
One of the reasons for including the variable rule is that it provides an \emph{identity function}\index{identity function} on the type $A$ in context $\Gamma$.

\subsection{Derivations}\label{sec:derivations}

\index{derivation|(}
A derivation in type theory is a finite tree in which each node is a valid rule of inference. At the root of the tree we find the conclusion, and in the leaves of the tree we find the hypotheses. We give two examples of derivations: a derivation showing that any variable can be changed to a fresh one, and a derivation showing that any two variables that do not mutually depend on one another can be swapped in order.

Given a derivation with hypotheses $\mathcal{H}_1,\ldots,\mathcal{H}_n$ and conclusion $\mathcal{C}$, we can form a new inference rule
\begin{prooftree}
  \AxiomC{$\mathcal{H}_1$}
  \AxiomC{$\cdots$}
  \AxiomC{$\mathcal{H}_n$}
  \TrinaryInfC{$\mathcal{C}$}
\end{prooftree}
Such a rule is called \define{derivable}, because we have a derivation for it. In order to keep proof trees reasonably short and manageable, we use the convention that any derived rules can be used in future derivations.

\subsubsection*{Changing variables}

\index{change of variables}
Variables can always be changed to fresh variables. We show that this is the case by showing that the inference rule\index{rules!for type dependency!change of variables}
\begin{prooftree}
  \AxiomC{$\Gamma,x:A,\Delta\vdash \mathcal{J}$}
  \RightLabel{$x'/x$}
  \UnaryInfC{$\Gamma,x':A,\Delta[x'/x]\vdash \mathcal{J}[x'/x]$}
\end{prooftree}
is derivable, where $x'$ is a variable that does not occur in the context $\Gamma,x:A,\Delta$. 

Indeed, we have the following derivation using substitution, weakening, and the variable rule:
\begin{prooftree}
  \AxiomC{$\Gamma\vdash A~\type$}
  \RightLabel{$\delta$}
  \UnaryInfC{$\Gamma,x':A\vdash x':A$}
  \AxiomC{$\Gamma\vdash A~\type$}
  \AxiomC{$\Gamma,x:A,\Delta\vdash \mathcal{J}$}
  \RightLabel{$W$}
  \BinaryInfC{$\Gamma,x':A,x:A,\Delta\vdash \mathcal{J}$}
  \RightLabel{$S$}
  \BinaryInfC{$\Gamma,x':A,\Delta[x'/x]\vdash \mathcal{J}[x'/x]$}
\end{prooftree}
In this derivation it is the application of the weakening rule where we have to check that $x'$ does not occur in the context $\Gamma,x:A,\Delta$.

\subsubsection*{Interchanging variables}

The \define{interchange rule}\index{rules!for type dependency!interchange}\index{interchange rule} states that if we have two types $A$ and $B$ in context $\Gamma$, and we make a judgment in context $\Gamma,x:A,y:B,\Delta$, then we can make that same judgment in context $\Gamma,y:B,x:A,\Delta$ where the order of $x:A$ and $y:B$ is swapped. More formally, the interchange rule is the following inference rule
\begin{prooftree}
\AxiomC{$\Gamma\vdash B~\textrm{type}$}
\AxiomC{$\Gamma,x:A,y:B,\Delta\vdash \mathcal{J}$}
\BinaryInfC{$\Gamma,y:B,x:A,\Delta\vdash \mathcal{J}$}
\end{prooftree}
Just as the rule for changing variables, we claim that the interchange rule is a derivable rule.

The idea of the derivation for the interchange rule is as follows: If we have a judgment
\begin{equation*}
  \Gamma,x:A,y:B,\Delta\vdash\mathcal{J},
\end{equation*}
then we can change the variable $y$ to a fresh variable $y'$ and weaken the judgment to obtain the judgment
\begin{equation*}
  \Gamma,y:B,x:A,y':B,\Delta[y'/y]\vdash\mathcal{J}[y'/y].
\end{equation*}
Now we can substitute $y$ for $y'$ to obtain the desired judgment $\Gamma,y:B,x:A,\Delta\vdash\mathcal{J}$. The formal derivation is as follows:
%\begin{small}
\begin{prooftree}
\AxiomC{$\Gamma\vdash B~\textrm{type}$}
\RightLabel{$\delta$}
\UnaryInfC{$\Gamma,y:B\vdash y:B$}
\RightLabel{$W$} 
\UnaryInfC{$\Gamma,y:B,x:A\vdash y:B$}
\AxiomC{$\Gamma,x:A,y:B,\Delta\vdash \mathcal{J}$}
\RightLabel{$y'/y$}
\UnaryInfC{$\Gamma,x:A,y':B,\Delta[y'/y]\vdash \mathcal{J}[y'/y]$}
\RightLabel{$W$}
\UnaryInfC{$\Gamma,y:B,x:A,y':B,\Delta[y'/y]\vdash \mathcal{J}[y'/y]$}
\RightLabel{$S$}
\BinaryInfC{$\Gamma,y:B,x:A,\Delta\vdash \mathcal{J}$}
\end{prooftree}
%\end{small}
\index{derivation|)}

\begin{exercises}
\exercise \label{ex:term_conversion}Give a derivation for the following \define{term conversion rule}\index{term conversion rule}\index{rules!for type dependency!term conversion}\index{term conversion rule}\index{conversion rule!term}:
\begin{prooftree}
\AxiomC{$\Gamma\vdash A\jdeq A'~\textrm{type}$}
\AxiomC{$\Gamma\vdash a:A$}
\BinaryInfC{$\Gamma\vdash a:A'$}
\end{prooftree}
\exercise Consider a type $A$ in context $\Gamma$. In this exercise we establish a correspondence between types in context $\Gamma,x:A$, and uniform choices of types $B_a$, where $a$ ranges over terms of $A$ in a uniform way. A similar connection is made for terms.
  \begin{subexenum}
  \item We define a \define{uniform family}\index{uniform family} over $A$ to consist of a type
    \begin{equation*}
      \Delta,\Gamma\vdash B_a~\type
    \end{equation*}
    for every context $\Delta$, and every term $\Delta,\Gamma\vdash a:A$, subject to the condition that one can derive
    \begin{prooftree}
      \AxiomC{$\Delta\vdash d:D$}
      \AxiomC{$\Delta,y:D,\Gamma\vdash a:A$}
      \BinaryInfC{$\Delta,\Gamma\vdash B_a[d/y]\jdeq B_{a[d/y]}~\type$}
    \end{prooftree}
    Define a bijection between the set of types in context $\Gamma,x:A$ modulo judgmental equality, and the set of uniform families over $A$ modulo judgmental equality. 
  \item Consider a type $\Gamma,x:A\vdash B$. We define a \define{uniform term}\index{uniform term} of $B$ over $A$ to consist of a type
    \begin{equation*}
      \Delta,\Gamma\vdash b_a:B[a/x]~\type
    \end{equation*}
    for every context $\Delta$, and every term $\Delta,\Gamma\vdash a:A$, subject to the condition that one can derive
    \begin{prooftree}
      \AxiomC{$\Delta\vdash d:D$}
      \AxiomC{$\Delta,y:D,\Gamma\vdash a:A$}
      \BinaryInfC{$\Delta,\Gamma\vdash b_a[d/y]\jdeq b_{a[d/y]}:B[a/x][d/y]$}
    \end{prooftree}
    Define a bijection between the set of terms of $B$ in context $\Gamma,x:A$ modulo judgmental equality, and the set of uniform terms of $B$ over $A$ modulo judgmental equality. 
  \end{subexenum}
\end{exercises}
\index{dependent type theory|)}

\section{Dependent function types}
\label{ch:pi}

\index{Pi-type@{$\Pi$-type}|see {dependent function type}}
\index{dependent function type|(}
A fundamental concept in dependent type theory is that of a dependent function. A dependent function is a function of which the type of the output may depend on the input. They are a generalization of ordinary functions, because an ordinary function $f:A\to B$ is a function of which the output $f(x)$ has type $B$ regardless of the value of $x$.

\subsection{Dependent function types}
Consider a section $b$ of a family $B$ over $A$ in context $\Gamma$, i.e.,
\begin{equation*}
  \Gamma,x:A\vdash b(x):B(x).
\end{equation*}
From one point of view, such a section $b$ is an operation, or a program\index{program}, that takes as input $x:A$ and produces a term $b(x):B(x)$. From a more mathematical point of view we see $b$ as a choice of an element of each $B(x)$. In other words, we may see $b$ as a function that takes $x:A$ to $b(x):B(x)$. Note that the type $B(x)$ of the output is dependent on $x:A$. In this section we postulate rules for the \emph{type} of all such dependent functions: whenever $B$ is a family over $A$ in context $\Gamma$, there is a type
\begin{equation*}
  \prd{x:A}B(x)
\end{equation*}
in context $\Gamma$, consisting of all the dependent functions of which the output at $x:A$ has type $B(x)$. There are four principal rules for $\Pi$-types:
\begin{enumerate}
\item The formation rule, which tells us how we may form dependent function types.
\item The introduction rule, which tells us how to introduce new terms of dependent function types.
\item The elimination rule, which tells us how to use arbitrary terms of dependent function types.
\item The computation rules, which tell us how the introduction and elimination rules interact. These computation rules guarantee that every term of a dependent function type behaves as expected: as a dependent function.
\end{enumerate}
In the cases of the formation rule, the introduction rule, and the elimination rule, we also need rules that assert that all the constructions respect judgmental equality. Those rules are called \define{congruence rules}.

\subsubsection{The $\Pi$-formation rule}
\index{dependent function type!formation rule}
\define{Dependent function types} are formed by the following \define{$\Pi$-formation rule}\index{rules!for dependent function types!formation}:
\begin{prooftree}
\AxiomC{$\Gamma,x:A\vdash B(x)~\textrm{type}$}
\RightLabel{$\Pi$.}
\UnaryInfC{$\Gamma\vdash \prd{x:A}B(x)~\type$}
\end{prooftree}

The congruence rule for $\Pi$-formation asserts that formation of dependent function types respects judgmental equality of types:
\index{rules!for dependent function types!congruence}
\index{dependent function type!congruence rule}
\begin{prooftree}
\AxiomC{$\Gamma\vdash A\jdeq A'~\type$}
\AxiomC{$\Gamma,x:A\vdash B(x)\jdeq B'(x)~\textrm{type}$}
\RightLabel{$\Pi$-eq.}
\BinaryInfC{$\Gamma\vdash \prd{x:A}B(x)\jdeq\prd{x:A'}B'(x)~\type$}
\end{prooftree}

There is one last rule that we need about the formation of $\Pi$-types, asserting that it does not matter what name we use for the variable $x$ that appears in the expression
\begin{equation*}
  \prd{x:A}B(x).
\end{equation*}
More precisely, when $x'$ is a fresh variable, i.e., which does not occur in the context $\Gamma,x:A$, we postulate that
\index{rules!for dependent function types!change of bound variable}
\index{dependent function type!change of bound variable}
\begin{prooftree}
\AxiomC{$\Gamma,x:A\vdash B(x)~\textrm{type}$}
\RightLabel{$\Pi$-$x'/x$.}
\UnaryInfC{$\Gamma\vdash \prd{x:A}B(x)\jdeq \prd{x':A}B(x')~\type$}
\end{prooftree}
This rule is also known as \define{$\alpha$-conversion} for $\Pi$-types.

\subsubsection{The $\Pi$-introduction rule}
The introduction rule%
\index{dependent function type!introduction rule|see {$\lambda$-abstraction}}
for dependent function types is also called the $\lambda$-abstraction rule. Recall that dependent functions are formed from terms $b(x)$ of type $B(x)$ in context $\Gamma,x:A$.
Therefore the \define{$\lambda$-abstraction rule}
\index{lambda-abstraction@{$\lambda$-abstraction}}
\index{rules!for dependent function types!lambda-abstraction@{$\lambda$-abstraction}}
\index{dependent function type!lambda-abstraction@{$\lambda$-abstraction}}
is as follows:
\begin{prooftree}
  \AxiomC{$\Gamma,x:A \vdash b(x) : B(x)$}
  \RightLabel{$\lambda$}
  \UnaryInfC{$\Gamma\vdash \lam{x}b(x) : \prd{x:A}B(x)$}
\end{prooftree}

Just like ordinary mathematicians, we will sometimes write $x\mapsto f(x)$ for a function $f$. The map $n\mapsto n^2$ is an example.

The $\lambda$-abstraction is also required to respect judgmental equality. Therefore we postulate the \define{congruence rule} for $\lambda$-abstraction,
\index{rules!for dependent function types!lambda-congruence@{$\lambda$-congruence}}
\index{lambda-congruence@{$\lambda$-congruence}}
\index{dependent function type!lambda-congruence@{$\lambda$-congruence}}
which asserts that
\begin{prooftree}
  \AxiomC{$\Gamma,x:A \vdash b(x)\jdeq b'(x) : B(x)$}
  \RightLabel{$\lambda$-eq.}
  \UnaryInfC{$\Gamma\vdash \lam{x}b(x)\jdeq \lam{x}b'(x) : \prd{x:A}B(x)$}
\end{prooftree}

\subsubsection{The $\Pi$-elimination rule}

\index{dependent function type!elimination rule|see {evaluation}}
The elimination rule for dependent function types provides us with a way to \emph{use} dependent functions. The way to use a dependent function is to apply it to an argument of the domain type. The $\Pi$-elimination rule is therefore also called the \define{evaluation rule}\index{evaluation}\index{rules!for dependent function types!evaluation}\index{dependent function type!evaluation}. It asserts that given a dependent function $f:\prd{x:A}B(x)$ in context $\Gamma$ we obtain a term $f(x)$ of type $B(x)$ in context $\Gamma,x:A$. More formally:
\begin{prooftree}
\AxiomC{$\Gamma\vdash f:\prd{x:A}B(x)$}
\RightLabel{$ev$}
\UnaryInfC{$\Gamma,x:A\vdash f(x) : B(x)$}
\end{prooftree}
Again we require that evaluation respects judgmental equality:
\begin{prooftree}
  \AxiomC{$\Gamma\vdash f\jdeq f':\prd{x:A}B(x)$}
  \UnaryInfC{$\Gamma,x:A\vdash f(x)\jdeq f'(x):B(x)$}
\end{prooftree}

\subsubsection{The $\Pi$-computation rules}

\index{dependent function type!computation rules|see {$\beta$- and $\eta$-rules}}
The computation rules for dependent function types postulate that $\lambda$-abstraction rule and the evaluation rule are mutual inverses. Thus we have two computation rules.

First we postulate the \define{$\beta$-rule}\index{beta-rule@{$\beta$-rule}!for Pi-types@{for $\Pi$-types}}\index{rules!for dependent function types!beta-rule@{$\beta$-rule}}\index{dependent function type!beta-rule@{$\beta$-rule}}
\begin{prooftree}
\AxiomC{$\Gamma,x:A \vdash b(x) : B(x)$}
\RightLabel{$\beta$.}
\UnaryInfC{$\Gamma,x:A \vdash (\lambda y.b(y))(x)\jdeq b(x) : B(x)$}
\end{prooftree}
Second, we postulate the \define{$\eta$-rule}\index{eta-rule@{$\eta$-rule}!for Pi-types@{for $\Pi$-types}}\index{rules!for dependent function types!eta-rule@{$\eta$-rule}}\index{dependent function type!eta-rule@{$\eta$-rule}}
\begin{prooftree}
\AxiomC{$\Gamma\vdash f:\prd{x:A}B(x)$}
\RightLabel{$\eta$.}
\UnaryInfC{$\Gamma \vdash \lam{x}f(x) \jdeq f : \prd{x:A}B(x)$}
\end{prooftree}
This completes the specification of dependent function types.

\subsection{Ordinary function types}
In the case where both $A$ and $B$ are types in context $\Gamma$, we may first weaken $B$ by $A$, and then apply the formation rule for the dependent function type:
\begin{prooftree}
\AxiomC{$\Gamma\vdash A~\textrm{type}$}
\AxiomC{$\Gamma\vdash B~\textrm{type}$}
\BinaryInfC{$\Gamma,x:A\vdash B~\textrm{type}$}
\UnaryInfC{$\Gamma\vdash \prd{x:A}B~\textrm{type}$}
\end{prooftree}
The result is the type of functions that take an argument of type $A$, and return a term of type $B$. In other words, terms of the type $\prd{x:A}B$ are \emph{ordinary} functions from $A$ to $B$. We write $A\to B$\index{A arrow B@{$A\to B$}|see {function type}} for the \define{type of functions}\index{function type} from $A$ to $B$. Sometimes we will also write $B^A$\index{B^A@{$B^A$}|see {function type}} for the type $A\to B$.

We give a brief summary of the rules specifying ordinary function types, omitting the congruence rules. All of these rules can be derived easily from the corresponding rules for $\Pi$-types.\index{rules!for function types}
\begin{prooftree}
\AxiomC{$\Gamma\vdash A~\textrm{type}$}
\AxiomC{$\Gamma\vdash B~\textrm{type}$}
\RightLabel{$\to$}
\BinaryInfC{$\Gamma\vdash A\to B~\textrm{type}$}
\end{prooftree}%
\begin{center}
\begin{minipage}{.45\textwidth}
\begin{prooftree}
\AxiomC{$\Gamma\vdash B~\textrm{type}$}
\AxiomC{$\Gamma,x:A\vdash b(x):B$}
\RightLabel{$\lambda$}
\BinaryInfC{$\Gamma\vdash \lam{x}b(x):A\to B$}
\end{prooftree}%
\end{minipage}
\begin{minipage}{.45\textwidth}
\begin{prooftree}
\AxiomC{$\Gamma\vdash f:A\to B$}
\RightLabel{$ev$}
\UnaryInfC{$\Gamma,x:A\vdash f(x):B$}
\end{prooftree}%
\end{minipage}
\end{center}
\begin{center}
\begin{minipage}{.45\textwidth}
\begin{prooftree}
\AxiomC{$\Gamma\vdash B~\textrm{type}$}
\AxiomC{$\Gamma,x:A\vdash b(x):B$}
\RightLabel{$\beta$}
\BinaryInfC{$\Gamma,x:A\vdash(\lam{y}b(y))(x)\jdeq b(x):B$}
\end{prooftree}%
\end{minipage}
\begin{minipage}{.45\textwidth}
\begin{prooftree}
\AxiomC{$\Gamma\vdash f:A\to B$}
\RightLabel{$\eta$}
\UnaryInfC{$\Gamma\vdash\lam{x} f(x)\jdeq f:A\to B$}
\end{prooftree}
\end{minipage}
\end{center}

\subsection{The identity function, composition, and their laws}
\begin{defn}
For any type $A$ in context $\Gamma$, we define the \define{identity function}\index{identity function}\index{function type!identity function} $\idfunc[A]:A\to A$\index{id A@{$\idfunc[A]$}} using the variable rule:
\begin{prooftree}
\AxiomC{$\Gamma\vdash A~\textrm{type}$}
\UnaryInfC{$\Gamma,x:A\vdash x:A$}
\UnaryInfC{$\Gamma\vdash \idfunc[A]\defeq \lam{x}x:A\to A$}
\end{prooftree}
\end{defn}

Note that we have used the symbol $\defeq$ in the conclusion to define the identity function. A judgment of the form $\Gamma\vdash a\defeq b:A$ should be read as "$b$ is a well-defined term of type $A$ in context $\Gamma$, and we will refer to it as $a$".

\begin{defn}
For any three types $A$, $B$, and $C$ in context $\Gamma$, there is a \define{composition}\index{function type!composition}\index{composition!of functions} operation
\begin{equation*}
\comp:(B\to C)\to ((A\to B)\to (A\to C)),
\end{equation*}
i.e., we can derive
\begin{prooftree}
\AxiomC{$\Gamma\vdash A~\textrm{type}$}
\AxiomC{$\Gamma\vdash B~\textrm{type}$}
\AxiomC{$\Gamma\vdash C~\textrm{type}$}
\TrinaryInfC{$\Gamma\vdash\comp:(B\to C)\to ((A\to B)\to (A\to C))$}
\end{prooftree}
We will write $g\circ f$\index{g composed with f@{$g\circ f$}} for $\comp(g,f)$\index{comp(g,f)@{$\comp(g,f)$}}.
\end{defn}

\begin{constr}
  The idea of the definition is to define $\comp(g,f)$ to be the function $\lam{x}g(f(x))$. The derivation we use to construct $\comp$ is as follows:
  \begin{prooftree}
    \AxiomC{$\Gamma\vdash A~\type$}
    \AxiomC{$\Gamma\vdash B~\type$}
    \BinaryInfC{$\Gamma,f:B^A,x:A\vdash f(x):B$}
    \UnaryInfC{$\Gamma,g:C^B,f:B^A,x:A\vdash f(x):B$}
    \AxiomC{$\Gamma\vdash B~\type$}
    \AxiomC{$\Gamma\vdash C~\type$}
    \BinaryInfC{$\Gamma,g:C^B,y:B\vdash g(y):C$}
    \UnaryInfC{$\Gamma,g:C^B,f:B^A,y:B\vdash g(y):C$}
    \UnaryInfC{$\Gamma,g:C^B,f:B^A,x:A,y:B\vdash g(y):C$}
    \BinaryInfC{$\Gamma,g:C^B,f:B^A,x:A\vdash g(f(x)) : C$}
    \UnaryInfC{$\Gamma,g:C^B,f:B^A\vdash \lam{x}g(f(x)):C^A$}
    \UnaryInfC{$\Gamma,g:B\to C\vdash \lam{f}\lam{x}g(f(x)):B^A\to C^A$}
    \UnaryInfC{$\Gamma\vdash\comp\defeq \lam{g}\lam{f}\lam{x}g(f(x)):C^B\to (B^A\to C^A)$}
  \end{prooftree}
\end{constr}

The rules of function types can be used to derive the laws of a category\index{category laws!for functions} for functions, i.e., we can derive that function composition is associative and that the identity function satisfies the unit laws. In the remainder of this section we will give these derivations.

\begin{lem}
Composition of functions is associative\index{associativity!of function composition}\index{composition!of functions!associativity}, i.e., we can derive
\begin{prooftree}
\AxiomC{$\Gamma\vdash f:A\to B$}
\AxiomC{$\Gamma\vdash g:B\to C$}
\AxiomC{$\Gamma\vdash h:C\to D$}
\TrinaryInfC{$\Gamma \vdash (h\circ g)\circ f\jdeq h\circ(g\circ f):A\to D$}
\end{prooftree}
\end{lem}

\begin{proof}
  The main idea of the proof is that both $((h\circ g)\circ f)(x)$ and $(h\circ (g\circ f))(x)$ evaluate to $h(g(f(x))$, and therefore $(h\circ g)\circ f$ and $h\circ(g\circ f)$ must be judgmentally equal. This idea is made formal in the following derivation:
  \begin{prooftree}
    \AxiomC{$\Gamma\vdash f:A\to B$}
    \UnaryInfC{$\Gamma,x:A\vdash f(x):B$}
    \AxiomC{$\Gamma\vdash g:B\to C$}
    \UnaryInfC{$\Gamma,y:B\vdash g(y):C$}
    \UnaryInfC{$\Gamma,x:A,y:B\vdash g(y):C$}
    \BinaryInfC{$\Gamma,x:A\vdash g(f(x)):C$}
    \AxiomC{$\Gamma\vdash h:C\to D$}
    \UnaryInfC{$\Gamma,z:C\vdash h(z):D$}
    \UnaryInfC{$\Gamma,x:A,z:C\vdash h(z):D$}
    \BinaryInfC{$\Gamma,x:A\vdash h(g(f(x))):D$}
    \UnaryInfC{$\Gamma,x:A\vdash h(g(f(x)))\jdeq h(g(f(x))):D$}
    \UnaryInfC{$\Gamma,x:A\vdash (h\circ g)(f(x))\jdeq h((g\circ f)(x)):D$}
    \UnaryInfC{$\Gamma,x:A\vdash ((h\circ g)\circ f)(x)\jdeq (h\circ (g \circ f))(x):D$}
    \UnaryInfC{$\Gamma\vdash (h\circ g)\circ f\jdeq h\circ(g\circ f):A\to D$.}
  \end{prooftree}
\end{proof}

\begin{lem}\label{lem:fun_unit}
Composition of functions satisfies the left and right unit laws\index{left unit law|see {unit laws}}\index{right unit law|see {unit laws}}\index{unit laws!for function composition}\index{composition!of functions!unit laws}, i.e., we can derive
\begin{prooftree}
\AxiomC{$\Gamma\vdash f:A\to B$}
\UnaryInfC{$\Gamma\vdash \idfunc[B]\circ f\jdeq f:A\to B$}
\end{prooftree}
and
\begin{prooftree}
\AxiomC{$\Gamma\vdash f:A\to B$}
\UnaryInfC{$\Gamma\vdash f\circ\idfunc[A]\jdeq f:A\to B$}
\end{prooftree}
\end{lem}

\begin{proof}
The derivation for the left unit law is
%\begin{prooftree}
%\AxiomC{$\Gamma\vdash f:A\to B$}
%\UnaryInfC{$\Gamma,x:A\vdash f(x):B$}
%\AxiomC{$\Gamma\vdash B~\type$}
%\UnaryInfC{$\Gamma,y:B\vdash \idfunc[B](y)\jdeq y:B$}
%\UnaryInfC{$\Gamma,x:A,y:B\vdash \idfunc[B](y)\jdeq y:B$}
%\BinaryInfC{$\Gamma,x:A\vdash \idfunc[B](f(x))\jdeq f(x):B$}
%\UnaryInfC{$\Gamma,x:A\vdash (\idfunc[B]\circ f)(x)\jdeq f(x):B$}
%\UnaryInfC{$\Gamma\vdash \idfunc[B]\circ f\jdeq f:A\to B$}
%\end{prooftree}
\begin{prooftree}
  \AxiomC{$\Gamma\vdash f:A\to B$}
  \UnaryInfC{$\Gamma,x:A\vdash f(x):B$}
  \AxiomC{$\Gamma\vdash A~\type$}
  \AxiomC{$\Gamma\vdash B~\type$}
  \UnaryInfC{$\Gamma,y:B\vdash\idfunc(y)\jdeq y:B$}
  \BinaryInfC{$\Gamma,x:A,y:B\vdash\idfunc(y)\jdeq y:B$}
  \BinaryInfC{$\Gamma,x:A\vdash\idfunc(f(x))\jdeq f(x):B$}
  \UnaryInfC{$\Gamma\vdash\lam{x}\idfunc(f(x))\jdeq\lam{x}f(x):A\to B$}
  \AxiomC{$\Gamma\vdash f:A\to B$}
  \UnaryInfC{$\Gamma\vdash\lam{x}f(x)\jdeq f:A\to B$}
  \BinaryInfC{$\Gamma\vdash\idfunc\circ f\jdeq f:A\to B$}
\end{prooftree}
The right unit law is left as \cref{ex:fun_right_unit}.
\end{proof}

\begin{exercises}
\exercise \label{ex:fun_right_unit}Give a derivation for the right unit law of \cref{lem:fun_unit}.\index{unit laws!for function composition}
\exercise Show that the rule
\begin{prooftree}
\AxiomC{$\Gamma,x:A \vdash b(x) : B(x)$}
\RightLabel{$\lambda$-$x'/x$}
\UnaryInfC{$\Gamma\vdash \lam{x}b(x)\jdeq \lam{x'}b(x') : \prd{x:A}B(x)$}
\end{prooftree}
is derivable for any variable $x'$ that does not occur in the context $\Gamma,x:A$.
\exercise 
  \begin{subexenum}
  \item Construct the \define{constant function}\index{constant function}\index{function!constant function}\index{const x@{$\const_x$}}\index{function!const@{$\const$}}
    \begin{prooftree}
      \AxiomC{$\Gamma\vdash A~\textrm{type}$}
      \UnaryInfC{$\Gamma,y:B\vdash \const_y:A\to B$}
    \end{prooftree}
  \item Show that
    \begin{prooftree}
      \AxiomC{$\Gamma\vdash f:A\to B$}
      \UnaryInfC{$\Gamma,z:C\vdash \const_z\circ f\jdeq\const_z : A\to C$}
    \end{prooftree}
  \item Show that
    \begin{prooftree}
      \AxiomC{$\Gamma\vdash A~\textrm{type}$}
      \AxiomC{$\Gamma\vdash g:B\to C$}
      \BinaryInfC{$\Gamma,y:B\vdash g\circ\const_y\jdeq \const_{g(y)}:A\to C$}
    \end{prooftree}
  \end{subexenum}
\exercise In this exercise we generalize the composition operation of non-dependent function types\index{composition!of dependent functions}:
\begin{subexenum}
\item Define a composition operation for dependent function types
  \begin{prooftree}
\AxiomC{$\Gamma\vdash f:\prd{x:A}B(x)$}
\AxiomC{$\Gamma\vdash g:\prd{x:A}\prd{y:B(x)} C(x,y)$}
\BinaryInfC{$\Gamma\vdash g\circ' f:\prd{x:A} C(x,f(x))$}
\end{prooftree}
and show that this operation agrees with ordinary composition when it is specialized to non-dependent function types.
\item Show that composition of dependent functions agrees with ordinary composition of functions:
  \begin{prooftree}
    \AxiomC{$\Gamma\vdash f:A\to B$}
    \AxiomC{$\Gamma\vdash g:B\to C$}
    \BinaryInfC{$\Gamma\vdash (\lam{x}g)\circ' f\jdeq g\circ f:A \to C$}
  \end{prooftree}
\item Show that composition of dependent functions is associative.\index{associativity!of dependent function composition}\index{composition!of dependent functions!associativity}
\item Show that composition of dependent functions satisfies the right unit law\index{unit laws!dependent function composition}:
\begin{prooftree}
\AxiomC{$\Gamma\vdash f:\prd{x:A}B(x)$}
\UnaryInfC{$\Gamma\vdash (\lam{x}f)\circ'\idfunc[A]\jdeq f :\prd{x:A}B(x)$}
\end{prooftree}
\item Show that composition of dependent functions satisfies the left unit law\index{unit laws!dependent function composition}\index{composition!of dependent functions!unit laws}:
\begin{prooftree}
\AxiomC{$\Gamma\vdash f:\prd{x:A}B(x)$}
\UnaryInfC{$\Gamma\vdash (\lam{x}\idfunc[B(x)])\circ' f\jdeq f:\prd{x:A}B(x)$}
\end{prooftree}
\end{subexenum}
\exercise \label{ex:swap}
\begin{subexenum}
\item Given two types $A$ and $B$ in context $\Gamma$, and a type $C$ in context $\Gamma,x:A,y:B$, define the \define{swap function}\index{function!swap}\index{swap function}
\begin{equation*}
\Gamma\vdash \sigma:\Big(\prd{x:A}\prd{y:B}C(x,y)\Big)\to\Big(\prd{y:B}\prd{x:A}C(x,y)\Big)
\end{equation*}
that swaps the order of the arguments.
\item Show that
\begin{equation*}
\Gamma\vdash \sigma\circ\sigma\jdeq\idfunc:\Big(\prd{x:A}\prd{y:B}C(x,y)\Big)\to \Big(\prd{x:A}\prd{y:B}C(x,y)\Big).
\end{equation*}
\end{subexenum}
\end{exercises}
\index{dependent function type|)}

\section{The natural numbers}

\index{inductive type|(}
\index{natural numbers|(}

The set of natural numbers is the most important object in mathematics. We quote Bishop\index{Bishop on the positive integers}, from his Constructivist Manifesto, the first chapter in Foundations of Constructive Analysis \cite{Bishop1967}, where he gives a colorful illustration of its importance to mathematics.

\begin{quote}
  ``The primary concern of mathematics is number, and this means the positive integers. We feel about number the way Kant felt about space. The positive integers and their arithmetic are presupposed by the very nature of our intelligence and, we are tempted to believe, by the very nature of intelligence in general. The development of the theory of the positive integers from the primitive concept of the unit, the concept of adjoining a unit, and the process of mathematical induction carries complete conviction. In the words of Kronecker, the positive integers were created by God. Kronecker would have expressed it even better if he had said that the positive integers were created by God for the benefit of man (and other finite beings). Mathematics belongs to man, not to God. We are not interested in properties of the positive integers that have no descriptive meaning for finite man. When a man proves a positive integer to exist, he should show how to find it. If God has mathematics of his own that needs to be done, let him do it himself.''
\end{quote}

A bit later in the same chapter, he continues:

\begin{quote}
  ``Building on the positive integers, weaving a web of ever more sets and ever more functions, we get the basic structures of mathematics: the rational number system, the real number system, the euclidean spaces, the complex number system, the algebraic number fields, Hilbert space, the classical groups, and so forth. Within the framework of these structures, most mathematics is done. Everything attaches itself to number, and every mathematical statement ultimately expresses the fact that if we perform certain computations within the set of positive integers, we shall get certain results.''
\end{quote}

\subsection{The formal specification of the type of natural numbers}
The type $\N$\index{N@{$\N$}|see {natural numbers}} of \define{natural numbers} is the archetypal example of an inductive type\index{inductive type!natural numbers}. The rules we postulate for the type of natural numbers come in four sets, just as the rules for $\Pi$-types:
\begin{enumerate}
\item The formation rule, which asserts that the type $\N$ can be formed.
\item The introduction rules, which provide the zero element and the successor function.
\item The elimination rule. This rule is the type theoretic analogue of the induction principle for $\N$.
\item The computation rules, which assert that any application of the elimination rule behaves as expected on the constructors $\zeroN$ and $\succN$ of $\N$.
\end{enumerate}

\subsubsection{The formation rule of $\N$}

\index{natural numbers!rules for N@{rules for $\N$}!formation}
\index{rules!for N@{for $\N$}!formation}
The type $\N$ is formed by the \define{$\N$-formation} rule
\begin{prooftree}
  \AxiomC{}
  \RightLabel{$\N$-form}
  \UnaryInfC{$\vdash \N~\type$.}
\end{prooftree}
In other words, $\N$ is postulated to be a closed type.

\subsubsection{The introduction rules of $\N$}
Unlike the set of positive integers in Bishop's remarks, Peano's first axiom postulates that $0$ is a natural number. The introduction rules for $\N$ equip it with the \define{zero term} and the \define{successor function}.
\index{natural numbers!rules for N@{rules for $\N$}!introduction rules}
\index{rules!for N@{for $\N$}!introduction rules}
\index{natural numbers!operations on N@{operations on $\N$}!0 N@{$\zeroN$}}
\index{0 N@{$\zeroN$}}
\index{successor function!on N@{on $\N$}}
\index{function!succ N@{$\succN$}}
\index{natural numbers!operations on N@{operations on $\N$}!succ N@{$\succN$}}
\index{succ N@{$\succN$}}

\bigskip
\begin{minipage}{.45\textwidth}
  \begin{prooftree}
    \AxiomC{}
    \UnaryInfC{$\vdash \zeroN:\N$}
  \end{prooftree}
\end{minipage}
\begin{minipage}{.45\textwidth}
  \begin{prooftree}
    \AxiomC{}
    \UnaryInfC{$\vdash \succN:\N\to\N$}
  \end{prooftree}
\end{minipage}

\bigskip
\begin{rmk}
  We annotate the terms $\zeroN$ and $\succN$ of type $\N$ with their type in the subscript, as a reminder that $\zeroN$ and $\succN$ are declared to be terms of type $\N$, and not of any other type. In the next chapter we will introduce the type $\Z$ of the integers, on which we can also define a zero term $\zeroZ$, and a successor function $\succZ$. These should be distinguished from the terms $\zeroN$ and $\succN$. In general, we will make sure that every term is given a unique name. In libraries of mathematics formalized in a computer proof assistant it is also the case that every type must be given a unique name.
\end{rmk}

\subsubsection{The elimination rule of $\N$}

\index{natural numbers!rules for N@{rules for $\N$}!elimination|see {induction}}
\index{natural numbers!rules for N@{rules for $\N$}!induction principle|(}
\index{induction principle!of N@{for $\N$}|(}
To prove properties about the natural numbers, we postulate an \emph{induction principle} for $\N$. For a typical example, it is easy to show by induction that
\begin{equation*}
  1+\dots+n=\frac{n(n+1)}{2}.
\end{equation*}
Similarly, we can define operations by recursion on the natural numbers: the Fibonacci sequence\index{Fibonacci sequence}\index{natural numbers!operations on N@{operations on $\N$}!Fibonacci sequence} is defined by $F(0)=0$, $F(1)=1$, and
\begin{equation*}
  F(n+2)=F(n)+F(n+1).
\end{equation*}
Needless to say, we want an induction principle to hold for the natural numbers in type theory and we also want it to be possible to construct operations on the natural numbers by recursion.

In dependent type theory we may think of a type family $P$ over $\N$ as a \emph{predicate} over $\N$. Especially after we introduce a few more type-forming operations, such as $\Sigma$-types and identity types, it will become clear that the language of dependent type theory expressive enough to find definitions of all of the standard concepts and operations of elementary number theory in type theory. Many of those definitions, the ordering relations $\leq$ and $<$ for example, will make use of type dependency. Then, to prove that $P(n)$ `holds' for all $n$ we just have to construct a dependent function
\begin{equation*}
  \prd{n:\N}P(n).
\end{equation*}

The induction principle for the natural numbers in type theory exactly states what one has to do in order to construct such a dependent function, via the following inference rule:\index{ind N@{$\ind{\N}$}}\index{rules!for N@{for $\N$}!induction principle}\index{natural numbers!indN@{$\indN$}}
\begin{prooftree}
  \def\fCenter{\Gamma}
  \Axiom$\fCenter, n:\N\vdash P(n)~\type$
  \noLine
  \UnaryInf$\fCenter\ \vdash p_0:P(\zeroN)$
  \noLine
  \UnaryInf$\fCenter\ \vdash p_S:\prd{n:\N}P(n)\to P(\succN(n))$
  \RightLabel{$\N$-ind}
  \UnaryInf$\fCenter\ \vdash \ind{\N}(p_0,p_S):\prd{n:\N} P(n)$
\end{prooftree}
Just like for the usual induction principle of the natural numbers, there are two things to be constructed given a type family $P$ over $\N$: in the \define{base case}\index{base case} we need to construct a term $p_0:P(\zeroN)$, and for the \define{inductive step}\index{inductive step} we need to construct a function of type $P(n)\to P(\succN(n))$ for all $n:\N$. And this comes at one immediate advantage: induction and recursion in type theory are one and the same thing!

\begin{rmk}
  We might alternatively present the induction principle of $\N$ as the following inference rule
  \begin{prooftree}
    \AxiomC{$\Gamma,n:\N\vdash P(n)~\type$}
    \UnaryInfC{$\Gamma\vdash \indN : P(\zeroN)\to \Big(\Big(\prd{n:\N}P(n)\to P(\succN(n))\Big)\to \prd{n:\N}P(n)\Big)$}
  \end{prooftree}
  In other words, for any type family $P$ over $\N$ there is a \emph{function} $\ind{\N}$ that takes two arguments, one for the base case and one for the inductive step, and returns a section of $P$. Now it is justified to wonder: is this slightly different presentation of induction equivalent to the previous presentation?
  
  To see that indeed we get such a function from the induction principle (rule $\N$-ind above), we note that the induction principle is stated to hold in an \emph{arbitrary} context $\Gamma$. So let us wield the power of type dependency: by weakening and the variable rule we have the following well-formed terms:
  \begin{align*}
    \Gamma,~p_0:P(\zeroN),~p_S:\prd{n:\N}P(n)\to P(\succN(n)) & \vdash p_0 : P(\zeroN) \\
    \Gamma,~p_0:P(\zeroN),~p_S:\prd{n:\N}P(n)\to P(\succN(n)) & \vdash p_S : \prd{n:\N}P(n)\to P(\succN(n)).
  \end{align*}
  Therefore, the induction principle of $\N$ provides us with a term
  \begin{equation*}
    \Gamma,~p_0:P(\zeroN),~p_S:\prd{n:\N}P(n)\to P(\succN(n)) \vdash \indN(p_0,p_S) : \prd{n:\N}P(n).
  \end{equation*}
  By $\lambda$-abstraction we now obtain a function
  \begin{equation*}
    \indN : P(\zeroN)\to \Big(\Big(\prd{n:\N}P(n)\to P(\succN(n))\Big) \to \prd{n:\N}P(n)\Big)
  \end{equation*}
  in context $\Gamma$. Therefore we see that it does not really matter whether we present the induction principle of $\N$ in a more verbose way as an inference rule with the base case and the inductive step as hypotheses, or as a function taking variables for the base case and the inductive step as arguments.
\end{rmk}
\index{natural numbers!rules for N@{rules for $\N$}!induction|)}
\index{induction principle!for N@{for $\N$}|)}

\subsubsection{The computation rules of $\N$}

\index{computation rules!for N@{for $\N$}|(}
\index{natural numbers!rules for N@{rules for $\N$}!computation rules|(}
The \define{computation rules} for $\N$ postulate that the dependent function $\ind{\N}(P,p_0,p_S)$ behaves as expected when it is applied to $\zeroN$ or a successor. There is one computation rule for each step in the induction principle, covering the base case and the inductive step.

The computation rule for the base case is\index{rules!for N@{for $\N$}!computation rules|(}
\begin{prooftree}
    \def\fCenter{\Gamma}
  \Axiom$\fCenter, n:\N\vdash P(n)~\type$
  \noLine
  \UnaryInf$\fCenter\ \vdash p_0:P(\zeroN)$
  \noLine
  \UnaryInf$\fCenter\ \vdash p_S:\prd{n:\N}P(n)\to P(\succN(n))$
  \UnaryInf$\fCenter\ \vdash \ind{\N}(p_0,p_S,\zeroN)\jdeq p_0 : P(\zeroN)$
\end{prooftree}
Similarly, with the same hypotheses as for the computation rule for the base case, the computation rule for the inductive step is
\begin{prooftree}
\AxiomC{$\cdots$}
\UnaryInfC{$\Gamma, n:\N \vdash  \ind{\N}(p_0,p_S,\succN(n))\jdeq p_S(n,\ind{\N}(p_0,p_S,n)) : P(\succN(n))$}
\end{prooftree}\index{rules!for N@{for $\N$}!computation rules|)}

This completes the formal specification of $\N$.
\index{computation rules!for N@{for $\N$}|)}
\index{natural numbers!rules for N@{rules for $\N$}!computation rules|)}

\subsection{Addition on the natural numbers}

\index{addition on N@{addition on $\N$}|(}
\index{natural numbers!operations on N@{operations on $\N$}!addition|(}
\index{function!addition on N@{addition on $\N$}|(}
Using the induction principle of $\N$ we can perform many familiar constructions. 
For instance, we can define the \define{addition operation} by induction on $\N$.

\begin{defn}
  We define a function\index{add N@{$\addN$}}\index{natural numbers!operations on N@{operations on $\N$}!add N@{$\addN$}}
  \begin{equation*}
    \addN:\N\to (\N\to\N)
  \end{equation*}
  satisfying $\addN(\zeroN,n)\jdeq n$ and $\addN(\succN(m),n)\jdeq\succN(\addN(m,n))$. Usually we will write $n+m$ for $\addN(n,m)$.
\end{defn}

We first give an informal construction of the addition operation, explaining the ideas behind the construction. This is important, because there are many binary operations on the natural numbers. The correctness of a formal construction of a term
\begin{equation*}
  \vdash \N\to(\N\to\N)
\end{equation*}
only shows us that we have correctly constructed a binary operation on the natural numbers, but this doesn't tell us that the operation we've defined is deserving of the name addition. There are indeed many binary operations on the natural numbers, such as the $\minN$, $\maxN$, and multiplication operations, so we need to be careful to make sure that the binary operation we are constructing really is the addition operation.

\begin{proof}[Informal construction]
  Our goal is to construct a function of type
  \begin{equation*}
    \vdash \addN:\N\to (\N\to\N).
  \end{equation*}
  By $\lambda$-abstraction it therefore suffices to construct a term
  \begin{equation*}
    m:\N\vdash \addN(m):\N\to\N.
  \end{equation*}
  Such a term is constructed by induction. Since we are defining addition, we want our definition of $\addN$ to be such that
  \begin{align*}
    \addN(m,\zeroN) & \jdeq n \\
    \addN(m,\succN(n)) & \jdeq \succN(\addN(m,n)). 
  \end{align*}
  In other words, our definition of addition is such that $m+0\jdeq m$ and $m+\succN(n)\jdeq \succN(m+n)$.

  The inductive proof requires us to define a term
  \begin{equation*}
    n:\N\vdash \addzeroN(n)\defeq n:\N
  \end{equation*}
  in the base case, and a term
  \begin{equation*}
    n:\N \vdash \addsuccN(n):\N\to(\N\to\N)
  \end{equation*}
  in the inductive step. The result of the inductive proof will then be a function $\addN(n):\N\to\N$ satisfying
  \begin{align*}
    n:\N & \vdash \addN(n,\zeroN) \jdeq \addzeroN(n) : \N \\
    n:\N & \vdash \addN(n,\succN(m)) \jdeq \addsuccN(n,m,\addN(n,m)).
  \end{align*}
  Anticipating these computation rules, we see that the following choices result in an addition operation with the expected behavior:
  \begin{align*}
    n:\N & \vdash \addzeroN(n)\defeq n : \N \\
    n:\N & \vdash \addsuccN(n)\defeq\const_{\succN}:\N\to(\N\to\N).\qedhere
  \end{align*}
\end{proof}

\begin{proof}[Formal derivation]
The derivation for the construction of $\addsuccN$ looks as follows:
\begin{prooftree}
  \AxiomC{}
  \UnaryInfC{$\vdash\N~\type$}
  \AxiomC{}
  \UnaryInfC{$\vdash\N~\type$}
  \AxiomC{}
  \UnaryInfC{$\vdash \succN:\N\to\N$}
  \BinaryInfC{$x:\N\vdash \succN:\N\to\N$}
  \BinaryInfC{$n:\N,x:\N \vdash \succN:\N\to\N$}
  \UnaryInfC{$n:\N \vdash \addsuccN(n) \defeq \lam{x}\succN:\N\to (\N \to \N)$}
\end{prooftree}
We combine this derivation with the induction principle of $\N$ to complete the construction of addition:
\begin{prooftree}
  \AxiomC{$\vdots$}
  \UnaryInfC{$n:\N\vdash \addzeroN(n) \defeq n:\N$}
  \AxiomC{$\vdots$}
  \UnaryInfC{$n:\N\vdash \addsuccN(n):\N\to (\N \to \N)$}
  \BinaryInfC{$n:\N\vdash\addN(n)\jdeq\indN(\addzeroN,\addsuccN):\N\to \N$}
\end{prooftree}
The asserted judgmental equalities then hold by the computation rules for $\N$.
\end{proof}

\begin{rmk}
  When we define a function $f:\prd{n:\N} P(n)$, we will often do so just by indicating its definition on $\zeroN$ and its definition on $\succN(n)$, by writing
  \begin{align*}
    f(\zeroN) & \defeq p_0 \\
    f(\succN(n)) & \defeq p_S(n,f(n)).
  \end{align*}
  For example, the definition of addition on the natural numbers could be given as
  \begin{align*}
    \addN(\zeroN,n) & \defeq n \\
    \addN(\succN(m),n) & \defeq \succN(\addN(m,n)).
  \end{align*}
  This way of defining a function is called \emph{pattern matching}\index{pattern matching}. A more formal inductive argument can be obtained from a definition by pattern matching if it is possible to obtain from the expression $p_S(n,f(n))$ a general dependent function
  \begin{equation*}
    p_S : \prd{n:\N} P(n)\to P(\succN(n)).
  \end{equation*}
  In practice this is usually the case. Computer proof assistants such as Agda have sophisticated algorithms to allow for definitions by pattern matching.
\end{rmk}

\begin{rmk}
  By the computation rules for $\N$ it follows that
  \begin{equation*}
    m+\zeroN\jdeq m,\qquad\text{and}\qquad m+\succN(n)\jdeq\succN(m+n).
  \end{equation*}
  A simple consequence of this definition is that $\succN(n)\jdeq n+1$, as one would expect. However, the rules that we provided so far are not sufficient to also conclude that $\zeroN+n\jdeq n$ and $\succN(m) + n\jdeq \succN(m+n)$. In fact, we will not be able to prove such judgmental equalities. Nevertheless, once we have introduced the \emph{identity type} in \cref{chap:identity} we will be able to \emph{identify} $\zeroN+n$ with $n$, and $\succN(m)+n$ with $\succN(m+n)$. See \cref{ex:semi-ring-laws-N}. 
\end{rmk}
\index{addition on N@{addition on $\N$}|)}
\index{natural numbers!operations on N@{operations on $\N$}!addition|)}
\index{function!addition on N@{addition on $\N$}|)}

\begin{exercises}
\exercise Define the binary \define{min} and \define{max} functions
  \index{minimum function}
  \index{maximum function}
  \index{function!minN@{$\minN$}}
  \index{function!maxN@{$\maxN$}}
  \index{natural numbers!operations on N@{operations on $\N$}!minN@{$\minN$}}
  \index{natural numbers!operations on N@{operations on $\N$}!maxN@{$\maxN$}}
  \begin{equation*}
    \minN,\maxN:\N\to(\N\to\N).
  \end{equation*}
\exercise Define the \define{multiplication} operation
  \index{multiplication!on N@{on $\N$}}
  \index{function!mul N@{$\mulN$}}
  \index{natural numbers!operations on N@{operations on $\N$}!mul N@{$\mulN$}}
  \index{mul N@{$\mulN$}}
  \begin{equation*}
    \mulN :\N\to(\N\to\N).
  \end{equation*}
\exercise Define the \define{exponentiation function} $n,m\mapsto m^n$ of type $\N\to (\N\to \N)$.
  \index{exponentiation function on N@{exponentiation function on $\N$}}
  \index{function!exponentiation on N@{exponentiation on $\N$}}
  \index{natural numbers!operations on N@{operations on $\N$}!exponentiation}
\exercise Define the \define{factorial} operation $n\mapsto n!$.
  \index{factorial operation}
  \index{function!factorial operation}
  \index{natural numbers!operations on N@{operations on $\N$}!n factorial@{$n"!$}}
\exercise Define the \define{binomial coefficient} $\binom{n}{k}$ for any $n,k:\N$, making sure that $\binom{n}{k}\jdeq 0$ when $n<k$.
  \index{binomial coefficient}
  \index{function!binomial coefficient}
  \index{natural numbers!operations on N@{operations on $\N$}!binomial coefficient}
  \exercise Use the induction principle of $\N$ to define the \define{Fibonacci sequence} as a function $F:\N\to\N$ that satisfies the equations
  \begin{align*}
    F(\zeroN) & \jdeq \zeroN \\
    F(\oneN) & \jdeq \oneN \\
    F(\succN(\succN(n))) & \jdeq F(n)+F(\succN(n)).
  \end{align*}
  \index{Fibonacci sequence}
  \index{natural numbers!operations on N@{operations on $\N$}!Fibonacci sequence}
\end{exercises}
\index{natural numbers|)}

\section{More inductive types}

Analogous to the type of natural numbers, many types can be specified as inductive types. In this section we introduce some further examples of inductive types: the unit type, the empty type, the booleans, coproducts, dependent pair types, and cartesian products. We also introduce the type of integers.

\subsection{The idea of general inductive types}

Just like the type of natural numbers, other inductive types are also specified by their \emph{constructors}, an \emph{induction principle}, and their \emph{computation rules}: 
\begin{enumerate}
\item The constructors tell what structure the inductive type comes equipped with. There may any finite number of constructors, even no constructors at all, in the specification of an inductive type. 
\item The induction principle specifies the data that should be provided in order to construct a section of an arbitrary type family over the inductive type. 
\item The computation rules assert that the inductively defined section agrees on the constructors with the data that was used to define the section. Thus, there is a computation rule for every constructor.
\end{enumerate}
The induction principle and computation rules can be generated automatically once the constructors are specified, but it goes beyond the scope of our course to describe general inductive types.
%For a more general treatment of inductive types, we refer to Chapter 5 of \cite{hottbook}.


\subsection{The unit type}
\index{unit type|(}
\index{inductive type!unit type|(}
A straightforward example of an inductive type is the \emph{unit type}, which has just one constructor. 
Its induction principle is analogous to just the base case of induction on the natural numbers.

\begin{defn}
We define the \define{unit type}\index{1 @{$\unit$}|see {unit type}}\index{unit type} to be a closed type $\unit$\index{unit type!is a closed type} equipped with a closed term\index{unit type!star@{$\ttt$}}
\begin{equation*}
\ttt:\unit,
\end{equation*}
satisfying the induction principle\index{induction principle!of unit type}\index{unit type!induction principle} that for any type family of types $P(x)$ indexed by $x:\unit$, there is a term\index{ind 1@{$\indunit$}}\index{unit type!indunit@{$\indunit$}}
\begin{equation*}
\indunit : P(\ttt)\to\prd{x:\unit}P(x)
\end{equation*}
for which the computation rule\index{computation rules!of unit type}\index{unit type!computation rules}
\begin{equation*}
\indunit(p,\ttt) \jdeq p
\end{equation*}
holds. Sometimes we write $\lam{\ttt}p$ for $\indunit(p)$.
\end{defn}

The induction principle can also be used to define ordinary functions out of the unit type. Indeed, given a type $A$ we can first weaken it to obtain the constant family over $\unit$, with value $A$. Then the induction principle of the unit type provides a function
\begin{equation*}
  \indunit : A \to (\unit\to A).
\end{equation*}
In other words, by the induction principle for the unit type we obtain for every $x:A$ a function $\pt_x\defeq\indunit(x):\unit\to A$.\index{ptx@{$\pt_x$}}
\index{unit type|)}
\index{inductive type!unit type|)}

\subsection{The empty type}
\index{empty type|(}
\index{inductive type!empty type|(}
The empty type is a degenerate example of an inductive type. It does \emph{not} come equipped with any constructors, and therefore there are also no computation rules. The induction principle merely asserts that any type family has a section. In other words: if we assume the empty type has a term, then we can prove anything.

\begin{defn}
We define the \define{empty type}\index{0 @{$\emptyt$}|see {empty type}} to be a type $\emptyt$ satisfying the induction principle\index{induction principle!of empty type}\index{empty type!induction principle} that for any family of types $P(x)$ indexed by $x:\empty$, there is a term\index{ind 0@{$\indempty$}}\index{empty type!indempty@{$\indempty$}}
\begin{equation*}
\indempty : \prd{x:\emptyt}P(x).
\end{equation*}
\end{defn}

The induction principle for the empty type can also be used to construct a function
\begin{equation*}
  \emptyt\to A
\end{equation*}
for any type $A$. Indeed, to obtain this function one first weakens $A$ to obtain the constant family over $\emptyt$ with value $A$, and then the induction principle gives the desired function.

Thus we see that from the empty type anything follows. Therefore, we we see that anything follows from $A$, if we have a function from $A$ to the empty type. This motivates the following definition.

\begin{defn}
  For any type $A$ we define \define{negation}\index{negation!of types}\index{neg (A)@{$\neg A$}|see {negation}} of $A$ by
  \begin{equation*}
    \neg A\defeq A\to\emptyt.
  \end{equation*}
\end{defn}

Since $\neg A$ is the type of functions from $A$ to $\emptyt$, a proof of $\neg A$ is given by assuming that $A$ holds, and then deriving a contradiction. This proof technique is called \define{proof of negation}\index{proof of negation}. Proofs of negation are not to be confused with \emph{proofs by contradiction}\index{proof by contradiction}. In type theory there is no way of obtaining a term of type $A$ from a term of type $(A\to \emptyt)\to\emptyt$.
\index{empty type|)}
\index{inductive type!empty type|)}

\subsection{The booleans}
\index{booleans}
\index{inductive type!booleans}

\begin{defn}
We define the \define{booleans}\index{2 @{$\bool$}|see {booleans}} to be a type $\bool$ that comes equipped with\index{booleans!btrue@{$\btrue$}}\index{booleans!bfalse@{$\bfalse$}}\index{0 2@{$\bfalse$}}\index{1 2@{$\btrue$}}
\begin{align*}
\bfalse & : \bool \\
\btrue & : \bool
\end{align*}
satisfying the induction principle\index{induction principle!of booleans}\index{booleans!induction principle} that for any family of types $P(x)$ indexed by $x:\bool$, there is a term\index{ind 2@{$\indbool$}}
\begin{equation*}
\indbool : P(\bfalse)\to \Big(P(\btrue)\to \prd{x:\bool}P(x)\Big)
\end{equation*}
for which the computation rules\index{computation rules!of booleans}\index{booleans!computation rules}
\begin{align*}
\indbool(p_0,p_1,\bfalse) & \jdeq p_0 \\
\indbool(p_0,p_1,\btrue) & \jdeq p_1
\end{align*}
hold.
\end{defn}

Just as in the cases for the unit type and the empty type, the induction principle for the booleans can also be used to construct an ordinary function $\bool\to A$, provided that we can construct two terms of type $A$. Indeed, by the induction principle for the booleans there is a function
\begin{equation*}
  \indbool : A \to (A\to A^\bool)
\end{equation*}
for any type $A$.

\begin{eg}\label{eg:boolean-ops}
  \index{boolean operations|(}\index{boolean logic|(}
  Using the induction principle of $\bool$ we can define all the operations of Boolean algebra\index{boolean algebra}. For example, the \define{boolean negation}\index{booleans!negation} operation $\negbool : \bool \to \bool$\index{negation function!on booleans}\index{neg 2@{$\negbool$}}\index{booleans!neg 2@{$\negbool$}} is defined by
  \begin{align*}
    \negbool(\btrue) & \defeq \bfalse & \negbool(\bfalse) & \defeq \btrue.
  \end{align*}
  The \define{boolean conjunction}\index{booleans!conjunction} operation $\blank\land\blank : \bool \to (\bool\to \bool)$ is defined by
  \begin{align*}
    \btrue\land\btrue & \defeq \btrue & \bfalse\land\btrue & \defeq \bfalse \\
    \btrue\land\bfalse & \defeq \bfalse & \bfalse\land\bfalse & \defeq \bfalse.
  \end{align*}
  The \define{boolean disjunction}\index{booleans!disjunction} operation $\blank\lor\blank : \bool \to (\bool\to \bool)$ is defined by
  \begin{align*}
    \btrue\lor\btrue & \defeq \btrue & \bfalse\lor\btrue & \defeq \btrue \\
    \btrue\lor\bfalse & \defeq \btrue & \bfalse\lor\bfalse & \defeq \bfalse.
  \end{align*}  
  We leave the definitions of some of the other boolean operations as \cref{ex:boolean-operation}. Note that the method of defining the boolean operations by the induction principle of $\bool$ is not that different from defining them by truth tables\index{truth tables}.

  Boolean logic is important, but it won't be very prominent in this course. The reason is simple: in type theory it is more natural to use the `logic' of types that is provided by the inference rules.\index{boolean operations|)}\index{boolean logic|)}
\end{eg}
\index{booleans|)}
\index{inductive type!booleans|)}

\subsection{Coproducts and the type of integers}
\index{coproduct|(}
\index{inductive type!coproduct|(}
\begin{defn}
Let $A$ and $B$ be types. We define the \define{coproduct}\index{disjoint sum|see {coproduct}} $A+B$\index{A + B@{$A+B$}|see {coproduct}} to be a type that comes equipped with\index{inl@{$\inl$}}\index{coproduct!inl@{$\inl$}}\index{inr@{$\inr$}}\index{coproduct!inr@{$\inr$}}
\begin{align*}
\inl & : A \to A+B \\
\inr & : B \to A+B
\end{align*}
satisfying the induction principle\index{induction principle!of coproduct}\index{coproduct!induction principle} that for any family of types $P(x)$ indexed by $x:A+B$, there is a term\index{ind +@{$\ind{+}$}}\index{coproduct!ind+@{$\ind{+}$}}
\begin{equation*}
\ind{+} : \Big(\prd{x:A}P(\inl(x))\Big)\to\Big(\prd{y:B}P(\inr(y))\Big)\to\prd{z:A+B}P(z)
\end{equation*}
for which the computation rules\index{computation rules!of coproduct}\index{coproduct!computation rules}
\begin{align*}
\ind{+}(f,g,\inl(x)) & \jdeq f(x) \\
\inr{+}(f,g,\inr(y)) & \jdeq g(y)
\end{align*}
hold. Sometimes we write $[f,g]$ for $\ind{+}(f,g)$.
\end{defn}

The coproduct of two types is sometimes also called the \define{disjoint sum}. By the induction principle of coproducts it follows that we have a function
\begin{equation*}
  (A\to X) \to \big((B\to X) \to (A+B\to X)\big)
\end{equation*}
for any type $X$. Note that this special case of the induction principle of coproducts is very much like the elimination rule of disjunction in first order logic: if $P$, $P'$, and $Q$ are propositions, then we have
\begin{equation*}
  (P\to Q)\to \big((P'\to Q)\to (P\lor P'\to Q)\big).
\end{equation*}
Indeed, we can think of \emph{propositions as types} and of terms as their constructive proofs. Under this interpretation of type theory the coproduct is indeed the disjunction.

\index{integers|(}
An important example of a type that can be defined using coproducts is the type $\Z$ of integers.\index{coproduct!Z@{$\Z$}}

\begin{defn}
  We define the \define{integers}\index{Z@{$\Z$}|see {integers}} to be the type $\Z\defeq\N+(\unit+\N)$. The type of integers comes equipped with inclusion functions of the positive and negative integers\index{integers!in-pos@{$\inpos$}}\index{integers!in-neg@{$\inneg$}}
  \begin{align*}
    \inpos & \defeq \inr\circ\inr \\
    \inneg & \defeq \inl,
  \end{align*}
  which are both of type $\N\to\Z$, and the constants\index{integers!-1 Z@{$-1_\Z$}}\index{integers!0 Z@{$0_\Z$}}\index{integers!1 Z@{$1_\Z$}}\index{-1 Z@{$-1_\Z$}}\index{0 Z@{$0_\Z$}}\index{1 Z@{$1_{\Z}$}}
  \begin{align*}
    -1_\Z & \defeq \inneg(0)\\
    0_\Z & \defeq \inr(\inl(\ttt))\\
    1_\Z & \defeq \inpos(0).
  \end{align*}
\end{defn}

In the following lemma we derive an induction principle\index{induction principle!of Z@{of $\Z$}}\index{integers!induction principle} for $\Z$, which can be used in many familiar constructions on $\Z$, such as in the definitions of addition and multiplication.

\begin{lem}\label{lem:Z_ind}
  Consider a type family $P$ over $\Z$. If we are given
  \begin{align*}
    p_{-1} & :P(-1_\Z) \\
    p_{-S} & : \prd{n:\N}P(\inneg(n))\to P(\inneg(\succN(n)))\\
    p_{0} & : P(0_\Z) \\
    p_{1} & : P(1_\Z) \\
    p_{S} & : \prd{n:\N}P(\inpos(n))\to P(\inpos(\succN(n))),
  \end{align*}
  then we can construct a dependent function $f:\prd{k:\Z}P(k)$ for which the following judgmental equalities hold:\index{integers!computation rules}\index{computation rules!of Z@{of $\Z$}}
  \begin{align*}
    f(-1_\Z) & \jdeq p_{-1} \\
    f(\inneg(\succN(n))) & \jdeq p_{-S}(n,f(\inneg(n))) \\
    f(0_\Z) & \jdeq p_{0} \\
    f(1_\Z) & \jdeq p_{1} \\
    f(\inpos(\succN(n))) & \jdeq p_S(n,f(\inpos(n))).
  \end{align*}
\end{lem}

\begin{proof}
  Since $\Z$ is the coproduct of $\N$ and $\unit+\N$, it suffices to define
  \begin{align*}
    p_{inl} & : \prd{n:\N}P(\inl(n)) \\
    p_{inr} & : \prd{t:\unit+\N}P(\inr(t)).
  \end{align*}
  Note that $\inneg\jdeq\inl$ and $-1_\Z\jdeq \inneg(\zeroN)$. In order to define $p_{inl}$ we use induction on the natural numbers, so it suffices to define
  \begin{align*}
    p_{-1} & : P(-1) \\
    p_{-S} & : \prd{n:\N} P(\inneg(n))\to P(\inneg(\succN(n))).
  \end{align*}
  Similarly, we proceed by coproduct induction, followed by induction on $\unit$ in the left case and induction on $\N$ on the right case, in order to define $p_{inr}$. 
\end{proof}

As an application we define the successor function on the integers.

\begin{defn}
We define the \define{successor function}\index{successor function!on Z@{on $\Z$}}\index{function!succ Z@{$\succZ$}} on the integers $\succZ:\Z\to\Z$\index{succ Z@{$\succZ$}}\index{integers!succ Z@{$\succZ$}} using the induction principle of \cref{lem:Z_ind}, taking
\begin{align*}
\succZ(-1_\Z) & \defeq 0_\N \\
\succZ(\inneg(\succN(n))) & \defeq \inneg(n) \\
\succZ(0_\Z) & \defeq 1_\N \\
\succZ(1_\Z) & \defeq \inpos(1_\N) \\
\succZ(\inpos(\succN(n))) & \defeq \inpos(\succN(\succN(n))).
\end{align*}
\end{defn}
\index{integers|)}
\index{coproduct|)}
\index{inductive type!coproduct|)}

\subsection{Dependent pair types}

\index{dependent pair type|(}
\index{inductive type!dependent pair type|(}

Given a type family $B$ over $A$, we may consider pairs $(a,b)$ of terms, where $a:A$ and $b:B(a)$. Note that the type of $b$ depends on the first term in the pair, so we call such a pair a \define{dependent pair}\index{dependent pair}.

The \emph{dependent pair type} is an inductive type that is generated by the dependent pairs.


\begin{defn}
  Consider a type family $B$ over $A$.
  The \define{dependent pair type} (or $\Sigma$-type) \index{Sigma-type@{$\Sigma$-type}|see {dependent pair type}}is defined to be the inductive type $\sm{x:A}B(x)$ equipped with a \define{pairing function}\index{pairing function}\index{(-,-)@{$(\blank,\blank)$}}\index{dependent pair type!(-,-)@{$(\blank,\blank)$}}
\begin{equation*}
(\blank,\blank):\prd{x:A} \Big(B(x)\to \sm{y:A}B(y)\Big).
\end{equation*}
The induction principle\index{induction principle!of Sigma types@{of $\Sigma$-types}}\index{dependent pair type!induction principle} for $\sm{x:A}B(x)$ asserts that for any family of types $P(p)$ indexed by $p:\sm{x:A}B(x)$, there is a function\index{dependent pair type!indSigma@{$\ind{\Sigma}$}}\index{ind Sigma@{$\ind{\Sigma}$}}
\begin{equation*}
\ind{\Sigma}:\Big(\prd{x:A}\prd{y:B(x)}P(x,y)\Big)\to\Big(\prd{p:\sm{x:A}B(x)}P(p)\Big).
\end{equation*}
satisfying the computation rule\index{computation rules!of Sigma types@{of $\Sigma$-types}}\index{dependent pair type!computation rule}
\begin{equation*}
\ind{\Sigma}(f,(x,y))\jdeq f(x,y).
\end{equation*}
Sometimes we write $\lam{(x,y)}f(x,y)$ for $\ind{\Sigma}(\lam{x}\lam{y}f(x,y))$. 
\end{defn}

\begin{defn}
Given a type $A$ and a type family $B$ over $A$, the \define{first projection map}\index{first projection map}\index{projection maps!first projection}\index{dependent pair type!pr 1@{$\proj 1$}}\index{pr 1@{$\proj 1$}}\index{function!pr 1@{$\proj 1$}}
\begin{equation*}
\proj 1:\Big(\sm{x:A}B(x)\Big)\to A
\end{equation*}
is defined by induction as
\begin{equation*}
\proj 1\defeq \lam{(x,y)}x.
\end{equation*}
The \define{second projection map}\index{second projection map}\index{projection map!second projection}\index{dependent pair type!pr 2@{$\proj 2$}}\index{pr 2@{$\proj 2$}}\index{function!pr 2@{$\proj 2$}} is a dependent function
\begin{equation*}
\proj 2 : \prd{p:\sm{x:A}B(x)} B(\proj 1(p))
\end{equation*}
defined by induction as
\begin{equation*}
\proj 2\defeq \lam{(x,y)}y.
\end{equation*}
By the computation rule we have
\begin{align*}
\proj 1 (x,y) & \jdeq x \\
\proj 2 (x,y) & \jdeq y.
\end{align*}
\end{defn}
\index{dependent pair type|)}
\index{inductive type!dependent pair type|)}

\subsection{Cartesian products}

\index{cartesian product|(}
\index{inductive type!cartesian product|(}
A special case of the $\Sigma$-type occurs when the $B$ is a constant family over $A$, i.e., when $B$ is just a type.
In this case, the inductive type $\sm{x:A}B(x)$ is generated by \emph{ordinary} pairs $(x,y)$ where $x:A$ and $y:B$. In other words, if $B$ does not depend on $A$, then the type $\sm{x:A}B$ is the \emph{(cartesian) product} $A\times B$.
The cartesian product is a very common special case of the dependent pair type, just as the type $A\to B$ of ordinary functions from $A\to B$ is a common special case of the dependent product. Therefore we provide its specification along with the induction principle for cartesian products.

\begin{defn}
Consider two types $A$ and $B$. The \define{(cartesian) product}\index{product of types}\index{A x B@{$A\times B$}|see {cartesian product}} of $A$ and $B$ is defined as the inductive type $A\times B$ with constructor
\begin{equation*}
(\blank,\blank):A\to (B\to A\times B).
\end{equation*}
The induction principle\index{induction principle!of cartesian products}\index{cartesian product!induction principle} for $A\times B$ asserts that for any type family $P$ over $A\times B$, one has\index{ind times@{$\ind{\times}$}}\index{cartesian product!indtimes@{$\ind{\times}$}}
\begin{equation*}
\ind{\times} : \Big(\prd{x:A}\prd{y:B}P(a,b)\Big)\to\Big(\prd{p:A\times B} P(p)\Big)
\end{equation*}
satisfying the computation rule\index{computation rules!of cartesian product}\index{cartesian product!computation rule} that
\begin{align*}
\ind{\times}(f,(x,y)) & \jdeq f(x,y).
\end{align*}
\end{defn}

The projection maps are defined similarly to the projection maps of $\Sigma$-types. When one thinks of types as propositions\index{propositions as types!conjunction}, then $A\times B$ is interpreted as the conjunction of $A$ and $B$.
\index{cartesian product|)}
\index{inductive type!cartesian product|)}

\begin{exercises}
\exercise
  \index{rules!for unit type}\index{unit type!rules}
  \index{rules!for empty type}\index{empty type!rules}
  \index{rules!for booleans}\index{booleans!rules}
  \index{rules!for coproduct}\index{coproduct!rules}
  \index{rules!for dependent pair type}\index{dependent pair type!rules}
  \index{rules!for cartesian product}\index{cartesian product!rules}
  Write the rules for $\unit$, $\emptyt$, $\bool$, $A+B$, $\sm{x:A}B(x)$, and $A\times B$. As usual, present the rules in four sets:
  \begin{enumerate}
  \item A formation rule.
  \item Introduction rules.
  \item An elimination rule.
  \item Computation rules.
  \end{enumerate}
  \exercise Let $P$ and $Q$ be types. Use the fact that $\neg P$\index{negation} is defined as the type $P\to\emptyt$ of functions from $P$ to the empty type\index{empty type}, to give type theoretic proofs of the following taugologies\index{tautologies} of constructive logic\index{constructive logic}.\label{ex:dne-dec}
  \begin{subexenum}
  \item $P\to\neg\neg P$
  \item $(P\to Q)\to(\neg\neg P\to\neg\neg Q)$
  \item $(P+\neg P)\to(\neg\neg P\to P)$
  \item $\neg\neg(P+\neg P)$
  \item $\neg\neg(\neg\neg P \to P)$
  \item $(P\to \neg\neg Q)\to (\neg\neg P \to\neg\neg Q)$
  \item $\neg\neg\neg P \to \neg P$
  \item $\neg\neg(P \to \neg\neg Q)\to (P\to\neg\neg Q)$
  \item $\neg\neg((\neg\neg P)\times(\neg\neg Q))\to (\neg\neg P)\times(\neg\neg Q)$
  \end{subexenum}
\exercise \label{ex:boolean-operation}Define the following operations of Boolean algebra:\index{boolean algebra}\index{booleans!exclusive disjunction}\index{booleans!implication}\index{booleans!if and only if}\index{booleans!Peirce's arrow}\index{booleans!Sheffer stroke}
  \begin{center}
    \begin{tabular}{ll}
      exclusive disjunction & $p \oplus q$ \\
      implication & $p \Rightarrow q$ \\
      if and only if & $p \Leftrightarrow q$ \\
      Peirce's arrow (neither \dots{} nor) & $p \downarrow q$ \\
      Sheffer stroke (not both) & $p\mid q$.
    \end{tabular}
  \end{center}
  Here $p$ and $q$ range over $\bool$. 
\exercise \label{ex:int_pred}\index{integers|(}\index{predecessor function}\index{function!pred Z@{$\predZ$}}\index{integers!pred Z@{$\predZ$}}\index{pred Z@{$\predZ$}}Define the predecessor function $\predZ:\Z\to \Z$.
\exercise \label{ex:int_group_ops}\index{group operations!on Z@{on $\Z$}}Define the group operations\index{add Z@{$\addZ$}}\index{integers!add Z@{$\addZ$}}\index{neg Z@{$\negZ$}}\index{integers!neg Z@{$\negZ$}}\index{mul Z@{$\mulZ$}}\index{integers!mul Z@{$\mulZ$}}
  \begin{align*}
    \addZ & : \Z \to (\Z \to \Z) \\
    \negZ & : \Z \to \Z,
    \intertext{and define the multiplication}
    \mulZ & : \Z \to (\Z \to \Z).
  \end{align*}
\exercise Construct a function $F:\Z\to\Z$ that extends the Fibonacci sequence\index{Fibonacci sequence}\index{integers!Fibonacci sequence} to the negative integers
  \begin{equation*}
    \ldots,5,-3,2,-1,1,0,1,1,2,3,5,8,13,\ldots
  \end{equation*}
  in the expected way.\index{integers|)}
\exercise \label{ex:one_plus_one} Show that $\unit+\unit$ satisfies the same induction principle\index{induction principle!of booleans} as $\bool$, i.e., define
  \begin{align*}
    t_0 & : \unit + \unit \\
    t_1 & : \unit + \unit,
  \end{align*}
  and show that for any type family $P$ over $\unit+\unit$ there is a function
  \begin{align*}
    \ind{\unit+\unit}:P(t_0)\to \Big(P(t_1)\to \prd{t:\unit+\unit}P(t)\Big)
  \end{align*}
  satisfying
  \begin{align*}
    \ind{\unit+\unit}(p_0,p_1,t_0) & \jdeq p_0 \\
    \ind{\unit+\unit}(p_0,p_1,t_1) & \jdeq p_1.
  \end{align*}
  In other words, \emph{type theory cannot distinguish between the types $\bool$ and $\unit+\unit$.}
\exercise \label{ex:lists}For any type $A$ we can define the type $\lst(A)$\index{list A@{$\lst(A)$}|see {lists in $A$}} of \define{lists}\index{lists in A @{lists in $A$}}\index{inductive type!list A@{$\lst(A)$}} elements of $A$ as the inductive type with constructors\index{lists in A@{lists in $A$}!nil@{$\nil$}}\index{nil@{$\nil$}}\index{cons(a,l)@{$\cons(a,l)$}}\index{lists in A@{lists in $A$}!cons@{$\cons$}}
  \begin{align*}
    \nil & : \lst(A) \\
    \cons & : A \to (\lst(A) \to \lst(A)).
  \end{align*}
  \begin{subexenum}
  \item Write down the induction principle and the computation rules for $\lst(A)$.\index{induction principle!list A@{$\lst(A)$}}\index{lists in A@{lists in $A$}!induction principle}
  \item Let $A$ and $B$ be types, suppose that $b:B$, and consider a binary operation $\mu:A\to (B \to B)$. Define a function\index{fold-list@{$\foldlist$}}\index{lists in A@{lists in $A$}!fold-list@{$\foldlist$}}
    \begin{equation*}
      \foldlist(\mu) : \lst(A)\to B
    \end{equation*}
    that iterates the operation $\mu$, starting with $\foldlist(\mu,\nil)\defeq b$.
  \item Define a function $\lengthlist:\lst(A)\to\N$.\index{length-list@{$\lengthlist$}}\index{lists in A@{lists in $A$}!length-list@{$\lengthlist$}}
  \item Define a function\index{sum-list@{$\sumlist$}}\index{lists in A@{lists in $A$}!sum-list@{$\sumlist$}}
    \begin{equation*}
      \sumlist : \lst(\N) \to \N
    \end{equation*}
    that adds all the elements in a list of natural numbers.
  \item Define a function\index{concat-list@{$\concatlist$}}\index{lists in A@{lists in $A$}!concat-list@{$\concatlist$}}\index{concatenation!of lists}
    \begin{equation*}
      \concatlist : \lst(A) \to (\lst(A) \to \lst(A))
    \end{equation*}
    that concatenates any two lists of elements in $A$.
  \item Define a function\index{flatten-list@{$\flattenlist$}}\index{lists in A@{lists in $A$}!flatten-list@{$\flattenlist$}}
    \begin{equation*}
      \flattenlist : \lst(\lst(A)) \to \lst(A)
    \end{equation*}
    that concatenates all the lists in a lists of lists in $A$.
  \item Define a function $\reverselist : \lst(A) \to \lst(A)$ that reverses the order of the elements in any list.\index{reverse-list@{$\reverselist$}}\index{lists in A@{lists in $A$}!reverse-list@{$\reverselist$}}
  \end{subexenum}
\end{exercises}

\input{identity}
\section{Universes}

\index{universe|(}
\index{universal family|(}
To complete our specification of dependent type theory, we introduce type theoretic \emph{universes}. Universes are types that consist of types. In other words, a universe is a type $\UU$ that comes equipped with a type family $\Ty$ over $\UU$, and for any $X:\UU$ we think of $X$ as an \emph{encoding}\index{encoding of a type in a universe} of the type $\Ty(X)$. We call this type family the \emph{universal type family}.

There are several reasons to equip type theory with universes. One reason is that it enables us to define new type families over inductive types, using their induction principle. For example, since the universe is itself a type, we can use the induction principle of $\bool$ to obtain a map $P:\bool\to\UU$ from any two terms $X_0,X_1:\UU$. Then we obtain a type family over $\bool$ by substituting $P$ into the universal type family:
\begin{equation*}
  x:\bool\vdash \Ty(P(x))~\type
\end{equation*}
satisfying $\Ty(P(0_\bool))\jdeq \Ty(X_0)$ and $\Ty(P(1_\bool))\jdeq \Ty(X_1)$.

We use this way of defining type families to define many familiar relations over $\N$, such as $\leq$ and $<$. We also introduce a relation called \emph{observational equality} $\EqN$ on $\N$, which we can think of as equality of $\N$. This relation is reflexive, symmetric, and transitive, and moreover it is the least reflexive relation. Furthermore, one of the most important aspects of observational equality $\EqN$ on $\N$ is that $\EqN(m,n)$ is a type for every $m,n:\N$, unlike judgmental equality. Therefore we can use type theory to reason about observational equality on $\N$. Indeed, in the exercises we show that some very elementary mathematics can already be done at this early stage in our development of type theory.

A second reason to introduce universes is that it allows us to define many types of types equipped with structure. One of the most important examples is the type of groups\index{group}, which is the type of types equipped with the group operations satisfying the group laws, and for which the underlying type is a set\index{set}. We won't discuss the condition for a type to be a set until \cref{chap:hierarchy}, so the definition of groups in type theory will be given much later. Therefore we illustrate this use of the universe by giving simpler examples: pointed types, graphs, and reflexive graphs.

One of the aspects that make universes useful is that they are postulated to be closed under all the type constructors. For example, if we are given $X:\UU$ and $P:\Ty(X)\to \UU$, then the universe is equipped with a term
\begin{equation*}
  \check{\Sigma}(X,P):\UU
\end{equation*}
satisfying the judgmental equality $\Ty(\check{\Sigma}(X,P)\jdeq\sm{x:\Ty(X)}\Ty(P(x))$. We will similarly assume that any universe is closed under $\Pi$-types and the other ways of forming types. However, there is an important restriction: it would be inconsistent to assume that the universe is contained in itself. One way of thinking about this is that universes are types of \emph{small} types, and it cannot be the case that the universe is small with respect to itself. We address this problem by assuming that there are many universes: enough universes so that any type family can be obtained by substituting into the universal type family of some universe.

\subsection{Specification of type theoretic universes}

In the following definition we already state that universes are closed under identity types. Identity types will be introduced in \cref{chap:identity}.

\begin{defn}
  A \define{universe} in type theory is a closed type $\UU$\index{U, V, W@{$\UU$, $\mathcal{V}$, $\mathcal{W}$}|see {universe}} equipped with a type family $\Ty$\index{T@{$\Ty$}|see {universal family}} over $\UU$ called the \define{universal family}\index{family!universal family}, equipped with the following structure:
  \begin{enumerate}
  \item $\UU$ is closed under $\Pi$, in the sense that it comes equipped with a function
    \begin{equation*}
      \check{\Pi} :\prd{X:\UU}(\Ty(X)\to\UU)\to\UU
    \end{equation*}
    for which the judgmental equality
    \begin{equation*}
      \Ty\big(\check{\Pi}(X,P)\big)\jdeq \prd{x:\Ty(X)}\Ty(P(x)).
    \end{equation*}
    holds, for every $X:\UU$ and $P:\Ty(X)\to\UU$.
  \item $\UU$ is closed under $\Sigma$ in the sense that it comes equipped with a function
    \begin{equation*}
      \check{\Sigma} :\prd{X:\UU}(\Ty(X)\to\UU)\to\UU
    \end{equation*}
    for which the judgmental equality
    \begin{equation*}
      \Ty\big(\check{\Sigma}(X,P)\big) \jdeq \sm{x:\Ty(X)}\Ty(P(x))
    \end{equation*}
    holds, for every $X:\UU$ and $P:\Ty(X)\to\UU$.
  \item $\UU$ is closed under identity types, in the sense that it comes equipped with a function
    \begin{equation*}
      \check{\mathrm{I}} : \prd{X:\UU}\Ty(X)\to(\Ty(X)\to\UU)
    \end{equation*}
    for which the judgmental equality
    \begin{equation*}
      \Ty\big(\check{\mathrm{I}}(X,x,y)\big)\jdeq (\id{x}{y})
    \end{equation*}
    holds, for every $X:\UU$ and $x,y:\Ty(X)$.
  \item $\UU$ is closed under coproducts, in the sense that it comes equipped with a function
    \begin{equation*}
      \check{+}:\UU\to (\UU\to\UU)
    \end{equation*}
    that satisfies $\Ty\big(X\check{+}Y\big)\jdeq \Ty(X)+\Ty(Y)$.
  \item $\UU$ contains terms $\check{\emptyt},\check{\unit},\check{\N}:\UU$
    that satisfy the judgmental equalities
    \begin{align*}
      \Ty\big(\check{\emptyt}\big) & \jdeq \emptyt \\
      \Ty\big(\check{\unit}\big) & \jdeq \unit \\
      \Ty(\check{\N}) & \jdeq \N.
    \end{align*}
  \end{enumerate}
  Given a universe $\UU$, we say that a type $A$ in context $\Gamma$ is \define{small}\index{small type}\index{universe!small types} with respect to $\UU$ if it occurs in the universe, i.e., if it comes equipped with a term $\check{A}:\UU$ in context $\Gamma$, for which the judgment
  \begin{equation*}
    \Gamma\vdash\Ty\big(\check{A}\big)\jdeq A~\type
  \end{equation*}
  holds. If $A$ is small with respect to $\UU$, we usually write simply $A$ for $\check{A}$ and also $A$ for $\Ty(\check{A})$. In other words, by $A:\UU$ we mean that $A$ is a small type. 
\end{defn}

\begin{rmk}
  Since ordinary function types are defined as a special case of dependent function types, we don't have to assume that universes are closed under ordinary function types. Similarly, it follows from the assumption that universes are closed under dependent pair types that universes are closed under cartesian product types.
\end{rmk}

\subsection{Assuming enough universes}

\index{enough universes|(}
\index{universe!enough universes|(}
  Most of the time we will get by with assuming one universe $\UU$, and indeed we recommend on a first reading of this text to simply assume that there is one universe $\UU$. However, sometimes we might need a second universe $\mathcal{V}$ that contains $\UU$ as well as all the types in $\UU$. In such situations we cannot get by with a single universe, because the assumption that $\UU$ is a term of itself would lead to inconsistencies like the Russel's paradox.

  Russel's paradox is the famous argument that there cannot be a set of all sets. If there were such a set $S$, then we could consider Russel's subset
  \begin{equation*}
    R:=\{x\in S\mid x\notin x\}.
  \end{equation*}
  Russell then observed that $R\in R$ if and only if $R\notin R$, so we reach a contradiction. A variant of this argument reaches a similar contradiction when we assume that $\UU$ is a universe that contains a term $\check{\UU}:\UU$ such that $\mathcal{T}\big(\check{\UU}\big)\jdeq \UU$. In order to avoid such paradoxes, Russell and Whitehead formulated the \emph{ramified theory of types} in their book \emph{Principia Mathematica}. The ramified theory of types is a precursor of Martin L\"of's type theory that we are studying in this course.  

  Even though the universe is not a term of itself, it is still convenient if every type, including any universe, is small with respect to \emph{some} universe. Therefore we will assume that there are sufficiently many universes: we will assume that for every finite list of types
\begin{align*}
  \Gamma_1 & \vdash A_1~\type \\
  & ~\vdots \\
  \Gamma_n & \vdash A_n~\type,
\end{align*}
there is a universe $\UU$ that contains each $A_i$ in the sense that $\UU$ comes equipped with a term
\begin{align*}
  \Gamma_i\vdash \check{A}_i:\UU
\end{align*}
for which the judgment
\begin{equation*}
  \Gamma_i\vdash \Ty\big(\check{A}_i\big)\jdeq A_i~\type
\end{equation*}
holds. With this assumption it will rarely be necessary to work with more than one universe at the same time.

\begin{rmk}\label{rmk:universe-constructions}
  Using the assumption that for any finite list of types in context there is a universe that contains those types, we obtain many specific universes:
  \begin{enumerate}
  \item There is a \emph{base universe} $\UU_0$ that we obtain using the empty list of types in context. This is a universe, but it isn't specified to contain any further types.
  \item Given a finite list
    \begin{align*}
      \Gamma_1 & \vdash A_1~\type \\
      & ~\vdots \\
      \Gamma_n & \vdash A_n~\type,
    \end{align*}
    of types in context, and a universe $\UU$ that contains them, there is a universe $\UU^+$ that contains all the types in $\UU$ as well as $\UU$. More precisely, it is specified by the finite list
    \begin{align*}
      & \vdash \UU~\type \\
      X:\UU & \vdash \mathcal{T}(X)~\type.
    \end{align*}
    The universe $\UU^+$ therefore contains the type $\UU$ as well as every type in $\UU$, in the following sense
    \begin{align*}
      & \vdash \check{\UU}:\UU^+ & & \vdash \mathcal{T}^+(\check{\UU})\jdeq\UU~\type \\
      X:\UU & \vdash \check{\mathcal{T}}(X) :\UU^+ & X:\UU & \vdash \mathcal{T}^+(\check{\mathcal{T}}(X))\jdeq \mathcal{T}(X)~\type.
    \end{align*}
    In particular, we obtain a function $i:\UU\to\UU^+$ that includes the $\UU$-small types into $\UU^+$.
    
    Note that since the universe $\UU^+$ contains all the types in $\UU$, it also contains the types $A_1,\ldots,A_n$. To see this, we derive that there is a code for $A_i$ in $\UU^+$.
    \begin{prooftree}
      \AxiomC{$\Gamma_i\vdash\check{A}_i:\UU$}
      \AxiomC{$X:\UU\vdash\check{\mathcal{T}}(X):\UU^+$}
      \UnaryInfC{$\Gamma_i,X:\UU\vdash\check{\mathcal{T}}(X):\UU^+$}
      \BinaryInfC{$\Gamma_i\vdash\check{\mathcal{T}}(\check{A}_i):\UU^+$}
    \end{prooftree}
    We leave it as an exercise to derive the judgmental equality
    \begin{equation*}
      \mathcal{T}^+(\check{\mathcal{T}}(\check{A}_i))\jdeq A_i.
    \end{equation*}
  \item Given two finite lists
    \begin{align*}
      \Gamma_1 & \vdash A_1~\type & \Delta_1 & \vdash B_1~\type \\
      & ~\vdots & & ~\vdots \\
      \Gamma_n & \vdash A_n~\type & \Delta_m & \vdash B_m~\type
    \end{align*}
    of types in context, and two universes $\mathcal{U}$ and $\mathcal{V}$ that contain $A_1,\ldots,A_n$ and $B_1,\ldots,B_m$ respectively, there is a universe $\UU\sqcup\mathcal{V}$ that contains the types of both $\UU$ and $\mathcal{V}$. The universe $\UU\sqcup\mathcal{V}$ is specified by the finite list
    \begin{align*}
      X:\UU & \vdash \mathcal{T}_{\mathcal{U}}(X)~\type \\
      Y:\mathcal{V} & \vdash \mathcal{T}_{\mathcal{V}}(Y) ~\type.
    \end{align*}
    With an argument similar to the previous construction of a universe, we see that the universe $\UU\sqcup\mathcal{V}$ contains the types $A_1,\ldots,A_n$ as well as the types $B_1,\ldots,B_m$.

    Note that we could also directly obtain a universe $\mathcal{W}$ that contains the types $A_1,\ldots,A_n$ and $B_1,\ldots,B_m$. However, this universe might not contain all the types in $\UU$ or all the types in $\mathcal{V}$.
  \end{enumerate}
  Since we don't postulate any relations between the universes, there are indeed very few of them. For example, the base universe $\UU_0$ might contain many more types than it is postulated to contain. Nevertheless, there are some relations between the universes. For instance, there is a function $\UU\to\UU^+$, since we can simply derive
  \begin{prooftree}
    \AxiomC{$X:\UU\vdash \check{\mathcal{T}}(X):\UU^+$}
    \UnaryInfC{$\vdash \lam{X}\check{\mathcal{T}}(X) : \UU\to\UU^+$}
  \end{prooftree}
  Similarly, there are functions $\UU\to \UU\sqcup\mathcal{V}$ and $\mathcal{V}\to \UU\sqcup\mathcal{V}$ for any two universes $\UU$ and $\mathcal{V}$.
\end{rmk}
\index{enough universes|)}
\index{universe!enough universes|)}

\subsection{Pointed types}

\index{pointed type|(}
\begin{defn}
  A \define{pointed type} is a pair $(A,a)$ consisting of a type $A$ and a term $a:A$. The type of all pointed types in a universe $\UU$ is defined to be\index{UU*@{$\UU_\ast$}}
  \begin{equation*}
    \UU_\ast \defeq \sm{X:\UU}X.
  \end{equation*}
\end{defn}

\begin{defn}
  Consider two pointed types $(A,a)$ and $(B,b)$. A \define{pointed map}\index{pointed map} from $(A,a)$ to $(B,b)$ is a pair $(f,p)$ consisting of a function $f:A\to B$ and an identification $p:f(a)=b$. We write
  \begin{equation*}
    A\to_\ast B \defeq \sm{f:A\to B}f(a)=b
  \end{equation*}
  for the type of all pointed maps from $(A,a)$ to $(B,b)$, leaving the base point implicit.
\end{defn}

Since we have a type $\UU_\ast$ of \emph{all} pointed types in a universe $\UU$, we can start defining operations on $\UU_\ast$. An important example of such an operation is to take the loop space of a pointed type.

\begin{defn}
  We define the \define{loop space}\index{loop space}\index{Omega@{$\Omega$}|see {loop space}} operation $\Omega : \UU_\ast \to \UU_\ast$
  \begin{equation*}
    \Omega(A,a)\defeq \big((a=a),\refl{a}\big).
  \end{equation*}
\end{defn}

We can even go further and define the \emph{iterated loop space} of a pointed type. Note that this definition could not be given in type theory if we didn't have universes.

\begin{defn}
  Given a pointed type $(A,a)$ and a natural number $n$, we define the $n$-th loop space $\Omega^n(A,a)$ by induction on $n:\N$, taking\index{iterated loop space}\index{Omega^n@{$\Omega^n$}|see {iterated loop space}}
  \begin{align*}
    \Omega^0(A,a) & \defeq (A,a) \\
    \Omega^{n+1}(A,a) & \defeq \Omega(\Omega^n(A,a)).
  \end{align*}
\end{defn}
\index{pointed type|)}

\subsection{Families and relations on the natural numbers}

As we have already seen in the case of the iterated loop space, we can use the universe to define a type family over $\N$ by induction on $\N$. For example, we can define the finite types in this way.

\begin{defn}\label{defn:fin}
We define the type family $\Fin:\N\to\UU$ of finite types\index{Fin@{$\Fin$}}\index{finite types} by induction on $\N$\index{family!of finite types}, taking
\begin{align*}
\Fin(\zeroN) & \defeq \emptyt \\
\Fin(\succN(n)) & \defeq \Fin(n)+\unit
\end{align*}
\end{defn}

Similarly, we can define many relations on the natural numbers using a universe. We give here the example of \emph{observational equality} on $\N$. This inductively defined equivalence relation is very important, as it can be used to show that equality on the natural numbers is \emph{decidable}, i.e., there is a program that decides for any two natural numbers $m$ and $n$ whether they are equal or not.

\begin{defn}\label{defn:obs_nat}
We define the \define{observational equality}\index{observational equality!on N@{on $\N$}}\index{natural numbers!observational equality} on $\N$ as binary relation $\EqN:\N\to(\N\to\UU)$\index{Eq N@{$\EqN$}} satisfying
\begin{align*}
\EqN(\zeroN,\zeroN) & \jdeq \unit & \EqN(\succN(n),\zeroN) & \jdeq \emptyt \\
\EqN(\zeroN,\succN(n)) & \jdeq \emptyt & \EqN(\succN(n),\succN(m)) & \jdeq \EqN(n,m).
\end{align*}
\end{defn}

\begin{constr}
We define $\EqN$ by double induction on $\N$. By the first application of induction it suffices to provide
\begin{align*}
E_0 & : \N\to\UU \\
E_S & : \N\to (\N\to\UU)\to(\N\to\UU)
\end{align*}
We define $E_0$ by induction, taking $E_{00}\defeq \unit$ and $E_{0S}(n,X,m)\defeq \emptyt$. The resulting family $E_0$ satisfies
\begin{align*}
E_0(\zeroN) & \jdeq \unit \\
E_0(\succN(n)) & \jdeq \emptyt.
\end{align*} 
We define $E_S$ by induction, taking $E_{S0}\defeq \emptyt$ and $E_{S0}(n,X,m)\defeq X(m)$. The resulting family $E_S$ satisfies
\begin{align*}
E_S(n,X,\zeroN) & \jdeq \emptyt \\
E_S(n,X,\succN(m)) & \jdeq X(m) 
\end{align*}
Therefore we have by the computation rule for the first induction that the judgmental equality
\begin{align*}
\EqN(\zeroN,m) & \jdeq E_0(m) \\
\EqN(\succN(n),m) & \jdeq E_S(n,\EqN(n),m)
\end{align*}
holds, from which the judgmental equalities in the statement of the definition follow.
\end{constr}

\begin{lem}\label{ex:obs_nat_least}\index{observational equality!on N@{on $\N$}!is least reflexive relation}
  Suppose $R:\N\to(\N\to\UU)$ is a reflexive relation on $\N$\index{reflexive relation}\index{relation!reflexive}, i.e., $R$ comes equipped with
  \begin{equation*}
    \rho : \prd{n:\N}R(n,n).
  \end{equation*}
  Then there is a family of maps
  \begin{equation*}
    \prd{m,n:\N}\EqN(m,n)\to R(m,n).
  \end{equation*}
\end{lem}

\begin{proof}
  We will prove by induction on $m,n:\N$ that there is a term of type
  \begin{equation*}
    f_{m,n}:\prd{e:\EqN(m,n)}\prd{R:\N\to(\N\to\UU)}\Big(\prd{x:\N}R(x,x)\Big)\to R(m,n)
  \end{equation*}
  The dependent function $f_{m,n}$ is defined by
  \begin{align*}
    f_{\zeroN,\zeroN} & \defeq \lam{\ttt}\lam{r}\lam{\rho}\rho(\zeroN) \\
    f_{\zeroN,\succN(n)} & \defeq \indempty \\
    f_{\succN(m),\zeroN} & \defeq \indempty \\
    f_{\succN(m),\succN(n)} & \defeq \lam{e}\lam{R}\lam{\rho}f_{m,n}(e,R',\rho'),
  \end{align*}
  where $R'$ and $\rho'$ are given by
  \begin{align*}
    R'(m,n) & \defeq R(\succN(m),\succN(n)) \\
    \rho'(n) & \defeq \rho(\succN(n)).\qedhere
  \end{align*}
\end{proof}

We can also define observational equality for many other kinds of types, such as $\bool$ or $\Z$. In each of these cases, what sets the observational equality apart from other relations is that it is the \emph{least} reflexive relation. 

\begin{exercises}
\exercise \label{ex:obs_nat_eqrel}Show that observational equality on $\N$\index{observational equality!on N@{on $\N$}!is an equivalence relation} is an equivalence relation\index{equivalence relation!observational equality on N@{observational equality on $\N$}}, i.e., construct terms of the following types:
  \begin{align*}
    & \prd{n:\N} \EqN(n,n) \\
    & \prd{n,m:\N} \EqN(n,m)\to \EqN(m,n) \\
    & \prd{n,m,l:\N} \EqN(n,m)\to (\EqN(m,l)\to \EqN(n,l)).
  \end{align*}
\exercise \index{observational equality!on N@{on $\N$}!is preserved by functions}Show that every function $f:\N\to \N$ preserves observational equality in the sense that
  \begin{equation*}
    \prd{n,m:\N} \EqN(n,m)\to \EqN(f(n),f(m)).
  \end{equation*}
\exercise \label{ex:order_N}
  \begin{subexenum}
  \item Define the \define{order relations}\index{relation!order}\index{order relation!leq on N@{$\leq$ on $\N$}}\index{order relation!le on N@{$<$ on $\N$}}\index{leq@{$\leq$}!on N@{on $\N$}}\index{le@{$<$}!on N@{on $\N$}} $\leq$ and $<$ on $\N$.
  \item Show that $\leq$ is reflexive\index{reflexive!leq on N@{$\leq$ on $\N$}} and that $<$ is \define{irreflexive}\index{irreflexive}\index{relation!irreflexive}, i.e., show that
    \begin{equation*}
      (n\leq n)\qquad\text{and}\qquad (n\nless n).
    \end{equation*}
  \item Show that $n\leq\succN(n)$ and $n<\succN(n)$.
  \item Show that both $\leq$ and $<$ are transitive.
  \item Show that $\leq$ is \define{antisymmetric}, i.e., show that $m=n$ whenever both $m\leq n$ and $n\leq m$ hold. Also show that if $m<n$, then $n\nless m$.
  \item Show that
    \begin{equation*}
      (m\leq n)\leftrightarrow (m=n)+(m<n)
    \end{equation*}
    holds for all $m,n$.
  \item Show that
    \begin{equation*}
      (m\leq n)+(n<m)
    \end{equation*}
    holds for all $m,n$.
  \item Show that $k\leq \min(m,n)$ holds if and only if both $k\leq m$ and $k\leq n$ hold, and show that $\max(m,n)\leq k$ holds if and only if both $m\leq k$ and $n\leq k$ hold.
  \end{subexenum}
\exercise \label{ex:obs_bool}\index{observational equality!on 2@{on $\bool$}}
  \begin{subexenum}
  \item Define observational equality $\EqBool$\index{Eq 2@{$\EqBool$}} on the booleans.
  \item Show that $\EqBool$ is reflexive.\index{observational equality!on 2@{on $\bool$}!is reflexive}
  \item Show that for any reflexive relation $R:\bool\to(\bool\to \UU)$ one has\index{observational equality!on 2@{on $\bool$}!is least reflexive relation}
    \begin{equation*}
      \prd{x,y:\bool} \EqBool(x,y)\to R(x,y).
    \end{equation*}
  \end{subexenum}
\exercise \label{ex:int_order}
  \begin{subexenum}
  \item Define the order relations\index{relation!order}\index{order relation} $\leq$ and $<$ on and $\Z$.
  \item Show that $\leq$ is reflexive, transitive, and antisymmetric.
  \item Show that $<$ is irreflexive and transitive.
  \end{subexenum}
\begin{comment}
\item
  \begin{subexenum}
  \item For each $i:\Fin(\succN(n))$, define a function
    \begin{equation*}
      \skipFin_i : \Fin(n)\to\Fin(\succN(n))
    \end{equation*}
    that includes $\Fin(n)$ in $\Fin(\succN(n))$ by skipping $i$.
  \item For each $i:\Fin(n)$, define a function
    \begin{equation*}
      \doubleFin_i : \Fin(\succN(n))\to\Fin(n)
    \end{equation*}
    that projects $\Fin(\succN(n))$ onto $\Fin(n)$ by doubling at $i$. 
  \end{subexenum}
\end{comment}
\end{exercises}
\index{universe|)}
\index{universal family|)}



\chapter{The univalent foundations for mathematics}

In this chapter we study the foundational concepts of univalent mathematics.
The first concept we study is that of equivalences. Equivalent types are the same for all practical purposes. The concept of equivalence generalizes the concept of isomorphism of sets to type theory. As we delve deeper into synthetic homotopy theory, we will see how vast this generalization really is, even though the type theoretic definition is really simple.

The next topic is that of contractible types and maps. Contractible types that are singletons up to homotopy, i.e., types that come equipped with a point so that every (other) point can be identified with it. Contractible maps are maps $f:A\to B$ such that for every $b:B$ the type $\sm{a:A}f(a)=b$ is contractible. We will show that a map is an equivalence if and only if it is a contractible map in this sense.

It is a very important observation that for any $a:A$, the type
\begin{equation*}
  \sm{x:A}a=x  
\end{equation*}
is contractible. In other words, the total space of the path fibration is contractible. The fundamental theorem of identity types asserts that a type family $B$ over $A$ with $b:B(a)$ has a contractible total space
\begin{equation*}
  \sm{x:A}B(x)
\end{equation*}
if and only if $(a=x)\simeq B(x)$ for all $x:A$. The fundamental theorem of identity types helps us characterizing the identity types of virtually any type that we will encounter. Since types are only fully understood if we also have a clear understanding of their identity types, it is one of the core tasks of a homotopy type theorist to characterize identity types and the fundamental theorem is the main tool.

Not all types have very complicated identity types. For example, some types have the property that all their identity types are contractible. Such types have up to homotopy at most one term, so they are in a sense ``proof irrelevant''. The only thing that matters about such types is whether or not we can construct a term. Since this is also the case for propositions in first order logic, we call such types propositions.

At the next level there are types of which the identity types are propositions. In other words, the identity types of such types have the property of proof irrelevance. We are familiar with this situation from set theory, where equality is a proposition, so we call such types sets. One of our first major theorems is that the type of natural numbers is a set in this sense.

It is now clear that there is a hierarchy arising: first we have the contractible types; then we have the propositions, of which the identity types are contractible; after the propositions we have the sets, of which the identity types are propositions. Defining sets to be of truncation level $0$, we define a type to be of truncation level $n+1$ if its identity types are of truncation level $n$.

The hierarchy of truncation level is due to Voevodsky, who recognized the importance of specifying the level which you are working in. Most mathematics, for example, takes place at truncation level $0$, the level of sets. Groups, rings, posets, and so on are all set-level objects. Categories, on the other hand, are objects of truncation level $1$, the level of groupoids. The study of all these objects will be greatly facilitated by the univalence axiom, so we will postpone a detailed discussion of them until then.

We end the chapter with a section on elementary number theory, the way it is done in type theory. Our goal is to show how to prove in type theory that there are infinitely many prime numbers. Here it will be important to show that the ordering and divisilibility are properties, i.e., that the types $m\leq n$ and $d\mid n$ are propositions. Even more so, we will show that these are \emph{decidable} propositions. Type theory is by its very nature a constructive theory, but that doesn't stop us from showing that either $P$ or $\not P$ holds for some specific propositions $P$, like $m\leq n$ or $d\mid n$. One of the goals in the chapter on elementary number theory is to show how this is done, and how it is used.

\input{equivalences}
\input{contractible}
% !TEX root = hott_intro.tex

\section{The fundamental theorem of identity types}\label{chap:fundamental}
\sectionmark{The fundamental theorem}

\index{fundamental theorem of identity types|(}
\index{characterization of identity type!fundamental theorem of identity types|(}
For many types it is useful to have a characterization of their identity types. For example, we have used a characterization of the identity types of the fibers of a map in order to conclude that any equivalence is a contractible map. The fundamental theorem of identity types is our main tool to carry out such characterizations, and with the fundamental theorem it becomes a routine task to characterize an identity type whenever that is of interest.

In our first application of the fundamental theorem of identity types we show that any equivalence is an embedding. Embeddings are maps that induce equivalences on identity types, i.e., they are the homotopical analogue of injective maps. In our second application we characterize the identity types of coproducts.

Throughout the rest of this book we will encounter many more occasions to characterize identity types. For example, we will show in \cref{thm:eq_nat} that the identity type of the natural numbers is equivalent to its observational equality, and we will show in \cref{thm:eq-circle} that the loop space of the circle is equivalent to $\Z$.

In order to prove the fundamental theorem of identity types, we first prove the basic fact that a family of maps is a family of equivalences if and only if it induces an equivalence on total spaces. 

\subsection{Families of equivalences}

\index{family of equivalences|(}
\begin{defn}
Consider a family of maps
\begin{equation*}
f : \prd{x:A}B(x)\to C(x).
\end{equation*}
We define the map\index{total(f)@{$\tot{f}$}}
\begin{equation*}
\tot{f}:\sm{x:A}B(x)\to\sm{x:A}C(x)
\end{equation*}
by $\lam{(x,y)}(x,f(x,y))$.
\end{defn}

\begin{lem}\label{lem:fib_total}
  For any family of maps $f:\prd{x:A}B(x)\to C(x)$ and any $t:\sm{x:A}C(x)$,
  there is an equivalence\index{fiber!of total(f)@{of $\tot{f}$}}\index{total(f)@{$\tot{f}$}!fiber}
  \begin{equation*}
    \eqv{\fib{\tot{f}}{t}}{\fib{f(\proj 1(t))}{\proj 2(t)}}.
  \end{equation*}
\end{lem}

\begin{proof}
  For any $p:\fib{\tot{f}}{t}$ we define $\varphi(t,p):\fib{\proj 1(t)}{\proj 2(t)}$ by $\Sigma$-induction on $p$. Therefore it suffices to define $\varphi(t,(s,\alpha)):\fib{\proj 1(t)}{\proj 2 (t)}$ for any $s:\sm{x:A}B(x)$ and $\alpha:\tot{f}(s)=t$. Now we proceed by path induction on $\alpha$, so it suffices to define $\varphi(\tot{f}(s),(s,\refl{})):\fib{f(\proj 1(\tot{f}(s)))}{\proj 2(\tot{f}(s))}$. Finally, we use $\Sigma$-induction on $s$ once more, so it suffices to define
  \begin{equation*}
    \varphi((x,f(x,y)),((x,y),\refl{})):\fib{f(x)}{f(x,y)}.
  \end{equation*}
  Now we take as our definition
  \begin{equation*}
    \varphi((x,f(x,y)),((x,y),\refl{}))\defeq(y,\refl{}).
  \end{equation*}

  For the proof that this map is an equivalence we construct a map
  \begin{equation*}
    \psi(t) : \fib{f(\proj 1(t))}{\proj 2(t)}\to\fib{\tot{f}}{t}
  \end{equation*}
  equipped with homotopies $G(t):\varphi(t)\circ\psi(t)\htpy\idfunc$ and $H(t):\psi(t)\circ\varphi(t)\htpy\idfunc$. In each of these definitions we use $\Sigma$-induction and path induction all the way through, until an obvious choice of definition becomes apparent. We define $\psi(t)$, $G(t)$, and $H(t)$ as follows:
  \begin{align*}
    \psi((x,f(x,y)),(y,\refl{})) & \defeq ((x,y),\refl{}) \\
    G((x,f(x,y)),(y,\refl{})) & \defeq \refl{} \\
    H((x,f(x,y)),((x,y),\refl{})) & \defeq \refl{}.\qedhere
  \end{align*}
\end{proof}

\begin{thm}\label{thm:fib_equiv}
  Let $f:\prd{x:A}B(x)\to C(x)$ be a family of maps. The following are equivalent:
  \index{is an equivalence!total(f) of family of equivalences@{$\tot{f}$ of family of equivalences}}
  \index{total(f)@{$\tot{f}$}!of family of equivalences is an equivalence}\index{is family of equivalences!if total(f) is an equivalence@{iff $\tot{f}$ is an equivalence}}
\begin{enumerate}
\item For each $x:A$, the map $f(x)$ is an equivalence. In this case we say that $f$ is a \define{family of equivalences}.
\item The map $\tot{f}:\sm{x:A}B(x)\to\sm{x:A}C(x)$ is an equivalence.
\end{enumerate}
\end{thm}

\begin{proof}
By \cref{thm:equiv_contr,thm:contr_equiv} it suffices to show that $f(x)$ is a contractible map for each $x:A$, if and only if $\tot{f}$ is a contractible map. Thus, we will show that $\fib{f(x)}{c}$ is contractible if and only if $\fib{\tot{f}}{x,c}$ is contractible, for each $x:A$ and $c:C(x)$. However, by \cref{lem:fib_total} these types are equivalent, so the result follows by \cref{ex:contr_equiv}.
\end{proof}

Now consider the situation where we have a map $f:A\to B$, and a family $C$ over $B$. Then we have the map
\begin{equation*}
  \lam{(x,z)}(f(x),z):\sm{x:A}C(f(x))\to\sm{y:B}C(y).
\end{equation*}
We claim that this map is an equivalence when $f$ is an equivalence. The technique to prove this claim is the same as the technique we used in \cref{thm:fib_equiv}: first we note that the fibers are equivalent to the fibers of $f$, and then we use the fact that a map is an equivalence if and only if its fibers are contractible to finish the proof.

The converse of the following lemma does not hold. Why not?

\begin{lem}\label{lem:total-equiv-base-equiv}
  Consider an equivalence $e:A\simeq B$, and let $C$ be a type family over $B$. Then the map
  \begin{equation*}
    \sigma_f(C) \defeq\lam{(x,z)}(f(x),z):\sm{x:A}C(f(x))\to\sm{y:B}C(y)
  \end{equation*}
  is an equivalence.
\end{lem}

\begin{proof}
  We claim that for each $t:\sm{y:B}C(y)$ there is an equivalence
  \begin{equation*}
    \fib{\sigma_f(C)}{t}\simeq \fib{f}{\proj 1(t)}.
  \end{equation*}
  We obtain such an equivalence by constructing the following functions and homotopies:
  \begin{align*}
    \varphi(t) & : \fib{\sigma_f(C)}{t}\to\fib{f}{\proj 1 (t)} & \varphi((f(x),z),((x,z),\refl{})) & \defeq (x,\refl{}) \\
    \psi(t) & : \fib{f}{\proj 1(t)} \to\fib{\sigma_f(C)}{t} & \psi((f(x),z),(x,\refl{})) & \defeq ((x,z),\refl{}) \\
    G(t) & : \varphi(t)\circ\psi(t)\htpy\idfunc & G((f(x),z),(x,\refl{})) & \defeq \refl{} \\
    H(t) & : \psi(t)\circ\varphi(t)\htpy\idfunc & H((f(x),z),((x,z),\refl{})) & \defeq \refl{}.
  \end{align*}
  Now the claim follows, since we see that $\varphi$ is a contractible map if and only if $f$ is a contractible map.
\end{proof}

We now combine \cref{thm:fib_equiv,lem:total-equiv-base-equiv}.

\begin{defn}
  Consider a map $f:A\to B$ and a family of maps
  \begin{equation*}
    g:\prd{x:A}C(x)\to D(f(x)),
  \end{equation*}
  where $C$ is a type family over $A$, and $D$ is a type family over $B$. In this situation we also say that $g$ is a \define{family of maps over $f$}. Then we define\index{total f(g)@{$\tot[f]{g}$}}
  \begin{equation*}
    \tot[f]{g}:\sm{x:A}C(x)\to\sm{y:B}D(y)
  \end{equation*}
  by $\tot[f]{g}(x,z)\defeq (f(x),g(x,z))$.
\end{defn}

\begin{thm}\label{thm:equiv-toto}
  Suppose that $g$ is a family of maps over $f$, and suppose that $f$ is an equivalence. Then the following are equivalent:
  \begin{enumerate}
  \item The family of maps $g$ over $f$ is a family of equivalences.
  \item The map $\tot[f]{g}$ is an equivalence.
  \end{enumerate}
\end{thm}

\begin{proof}
  Note that we have a commuting triangle
  \begin{equation*}
    \begin{tikzcd}[column sep=0]
      \sm{x:A}C(x) \arrow[rr,"{\tot[f]{g}}"] \arrow[dr,swap,"\tot{g}"]& & \sm{y:B}D(y) \\
      & \sm{x:A}D(f(x)) \arrow[ur,swap,"{\lam{(x,z)}(f(x),z)}"]
    \end{tikzcd}
  \end{equation*}
  By the assumption that $f$ is an equivalence, it follows that the map $\sm{x:A}D(f(x))\to \sm{y:B}D(y)$ is an equivalence. Therefore it follows that $\tot[f]{g}$ is an equivalence if and only if $\tot{g}$ is an equivalence. Now the claim follows, since $\tot{g}$ is an equivalence if and only if $g$ if a family of equivalences.
\end{proof}
\index{family of equivalences|)}

\subsection{The fundamental theorem}

\index{identity system|(}
Many types come equipped with a reflexive relation that possesses a similar
structure as the identity type. The observational equality on the natural
numbers is such an example. We have see that it is a reflexive, symmetric, and
transitive relation, and moreover it is contained in any other reflexive
relation. Thus, it is natural to ask whether observational equality on the natural numbers is equivalent to the identity type.

The fundamental theorem of identity types (\cref{thm:id_fundamental}) is a general theorem that can be used to answer such questions. It describes a necessary and sufficient condition on a type family $B$ over a type $A$ equipped with a point $a:A$, for there to be a family of equivalences $\prd{x:A}(a=x)\simeq B(x)$. In other words, it tells us when a family $B$ is a characterization of the identity type of $A$.

Before we state the fundamental theorem of identity types we introduce the notion of \emph{identity systems}. Those are families $B$ over a $A$ that satisfy an induction principle that is similar to the path induction principle, where the `computation rule' is stated with an identification.

\begin{defn}
  Let $A$ be a type equipped with a term $a:A$. A \define{(unary) identity system} on $A$ at $a$ consists of a type family $B$ over $A$ equipped with $b:B(a)$, such that for any family of types $P(x,y)$ indexed by $x:A$ and $y:B(x)$,
  the function
  \begin{equation*}
    h\mapsto h(a,b):\Big(\prd{x:A}\prd{y:B(x)}P(x,y)\Big)\to P(a,b)
  \end{equation*}
  has a section.
\end{defn}

The most important implication in the fundamental theorem is that (ii) implies (i). Occasionally we will also use the third equivalent statement. We note that the fundamental theorem also appears as Theorem 5.8.4 in \cite{hottbook}.

\begin{thm}\label{thm:id_fundamental}
Let $A$ be a type with $a:A$, and let $B$ be be a type family over $A$ with $b:B(a)$.
Then  the following are logically equivalent for any family of maps
\begin{equation*}
  f:\prd{x:A}(a=x)\to B(x).
\end{equation*}
\begin{enumerate}
\item The family of maps $f$ is a family of equivalences.
\item The total space\index{is contractible!total space of an identity system}
\begin{equation*}
\sm{x:A}B(x)
\end{equation*}
is contractible.
\item The family $B$ is an identity system.
\end{enumerate}
In particular the canonical family of maps
\begin{equation*}
\pathind_a(b):\prd{x:A} (a=x)\to B(x)
\end{equation*}
is a family of equivalences if and only if $\sm{x:A}B(x)$ is contractible.
\end{thm}

\begin{proof}
  First we show that (i) and (ii) are equivalent.
  By \cref{thm:fib_equiv} it follows that the family of maps $f$ is a family of equivalences if and only if it induces an equivalence
  \begin{equation*}
    \eqv{\Big(\sm{x:A}a=x\Big)}{\Big(\sm{x:A}B(x)\Big)}
  \end{equation*}
  on total spaces. We have that $\sm{x:A}a=x$ is contractible. Now it follows by \cref{ex:contr_equiv}, applied in the case
  \begin{equation*}
    \begin{tikzcd}[column sep=3em]
      \sm{x:A}a=x \arrow[rr,"\tot{f}"] \arrow[dr,swap,"\eqvsym"] & & \sm{x:A}B(x) \arrow[dl] \\
      & \unit & \phantom{\sm{x:A}a=x}
    \end{tikzcd}
  \end{equation*}
  that $\tot{f}$ is an equivalence if and only if $\sm{x:A}B(x)$ is contractible.

  Now we show that (ii) and (iii) are equivalent. Note that we have the following commuting triangle
  \begin{equation*}
    \begin{tikzcd}[column sep=0]
      \prd{t:\sm{x:A}B(x)}P(t) \arrow[rr,"\evpair"] \arrow[dr,swap,"{\evpt(a,b)}"] & & \prd{x:A}\prd{y:B(x)}P(x,y) \arrow[dl,"{\lam{h}h(a,b)}"] \\
      \phantom{\prd{x:A}\prd{y:B(x)}P(x,y)} & P(a,b)
    \end{tikzcd}
  \end{equation*}
  In this diagram the top map has a section. Therefore it follows by \cref{ex:3_for_2} that the left map has a section if and only if the right map has a section. Notice that the left map has a section for all $P$ if and only if $\sm{x:A}B(x)$ satisfies singleton induction, which is by \cref{thm:contractible} equivalent to $\sm{x:A}B(x)$ being contractible.
\end{proof}
\index{identity system|)}

\subsection{Embeddings}
\index{embedding|(}
As an application of the fundamental theorem we show that equivalences are embeddings. The notion of embedding is the homotopical analogue of the set theoretic notion of injective map.

\begin{defn}
An \define{embedding} is a map $f:A\to B$\index{is an embedding} satisfying the property that\index{is an equivalence!action on paths of an embedding}
\begin{equation*}
\apfunc{f}:(\id{x}{y})\to(\id{f(x)}{f(y)})
\end{equation*}
is an equivalence for every $x,y:A$. We write $\isemb(f)$\index{is-emb(f)@{$\isemb(f)$}} for the type of witnesses that $f$ is an embedding.
\end{defn}

Another way of phrasing the following statement is that equivalent types have equivalent identity types.

\begin{thm}
\label{cor:emb_equiv} 
Any equivalence is an embedding.\index{is an embedding!equivalence}\index{equivalence!is an embedding}
\end{thm}

\begin{proof}
Let $e:\eqv{A}{B}$ be an equivalence, and let $x:A$. Our goal is to show that
\begin{equation*}
\apfunc{e} : (\id{x}{y})\to (\id{e(x)}{e(y)})
\end{equation*}
is an equivalence for every $y:A$. By \cref{thm:id_fundamental} it suffices to show that 
\begin{equation*}
\sm{y:A}e(x)=e(y)
\end{equation*}
is contractible for every $y:A$. Now observe that there is an equivalence
\begin{samepage}
\begin{align*}
\sm{y:A}e(x)=e(y) & \eqvsym \sm{y:A}e(y)=e(x) \\
& \jdeq \fib{e}{e(x)}
\end{align*}
\end{samepage}
by \cref{thm:fib_equiv}, since for each $y:A$ the map
\begin{equation*}
\invfunc : (e(x)=e(y))\to (e(y)= e(x))
\end{equation*}
is an equivalence by \cref{ex:equiv_grpd_ops}.
The fiber $\fib{e}{e(x)}$ is contractible by \cref{thm:contr_equiv}, so it follows by \cref{ex:contr_equiv} that the type $\sm{y:A}e(x)=e(y)$ is indeed contractible.
\end{proof}
\index{embedding|)}

\subsection{Disjointness of coproducts}

\index{disjointness of coproducts|(}
\index{characterization of identity type!coproduct|(}
\index{identity type!coproduct|(}
\index{coproduct!identity type|(}
\index{coproduct!disjointness|(}
To give a second application of the fundamental theorem of identity types, we characterize the identity types of coproducts. Our goal in this section is to prove the following theorem.

\begin{thm}\label{thm:id-coprod-compute}
Let $A$ and $B$ be types. Then there are equivalences
\begin{align*}
(\inl(x)=\inl(x')) & \eqvsym (x = x')\\
(\inl(x)=\inr(y')) & \eqvsym \emptyt \\
(\inr(y)=\inl(x')) & \eqvsym \emptyt \\
(\inr(y)=\inr(y')) & \eqvsym (y=y')
\end{align*}
for any $x,x':A$ and $y,y':B$.
\end{thm}

In order to prove \cref{thm:id-coprod-compute}, we first define
a binary relation $\Eqcoprod_{A,B}$ on the coproduct $A+B$.

\begin{defn}
Let $A$ and $B$ be types. We define 
\begin{equation*}
\Eqcoprod_{A,B} : (A+B)\to (A+B)\to\UU
\end{equation*}
by double induction on the coproduct, postulating
\begin{align*}
\Eqcoprod_{A,B}(\inl(x),\inl(x')) & \defeq (x=x') \\
\Eqcoprod_{A,B}(\inl(x),\inr(y')) & \defeq \emptyt \\
\Eqcoprod_{A,B}(\inr(y),\inl(x')) & \defeq \emptyt \\
\Eqcoprod_{A,B}(\inr(y),\inr(y')) & \defeq (y=y')
\end{align*}
The relation $\Eqcoprod_{A,B}$ is also called the \define{observational equality of coproducts}\index{observational equality!of coproducts}.
\end{defn}

\begin{lem}
The observational equality relation $\Eqcoprod_{A,B}$ on $A+B$ is reflexive, and therefore there is a map
\begin{equation*}
\Eqcoprodeq:\prd{s,t:A+B} (s=t)\to \Eqcoprod_{A,B}(s,t)
\end{equation*}
\end{lem}

\begin{constr}
The reflexivity term $\rho$ is constructed by induction on $t:A+B$, using
\begin{align*}
\rho(\inl(x))\defeq \refl{\inl(x)}  & : \Eqcoprod_{A,B}(\inl(x)) \\
\rho(\inr(y))\defeq \refl{\inr(y)} & : \Eqcoprod_{A,B}(\inr(y)).\qedhere
\end{align*}
\end{constr}

To show that $\Eqcoprodeq$ is a family of equivalences, we will use the fundamental theorem, \cref{thm:id_fundamental}. Moreover, we will use the functoriality of coproducts (established in \cref{ex:coproduct_functor}), and the fact that any total space over a coproduct is again a coproduct:
\begin{align*}
\sm{t:A+B}P(t) & \eqvsym \Big(\sm{x:A}P(\inl(x))\Big)+\Big(\sm{y:B}P(\inr(y))\Big)
\end{align*}
All of these equivalences are straightforward to construct, so we leave them as an exercise to the reader. 

\begin{lem}\label{lem:is-contr-total-eq-coprod}
For any $s:A+B$ the total space
\begin{equation*}
\sm{t:A+B}\Eqcoprod_{A,B}(s,t)
\end{equation*}
is contractible.
\end{lem}

\begin{proof}
We will do the proof by induction on $s$. The two cases are similar, so we only show that the total space
\begin{equation*}
\sm{t:A+B}\Eqcoprod_{A,B}(\inl(x),t)
\end{equation*}
is contractible. Note that we have equivalences
\begin{samepage}
\begin{align*}
& \sm{t:A+B}\Eqcoprod_{A,B}(\inl(x),t) \\
& \eqvsym \Big(\sm{x':A}\Eqcoprod_{A,B}(\inl(x),\inl(x'))\Big)+\Big(\sm{y':B}\Eqcoprod_{A,B}(\inl(x),\inr(y'))\Big) \\
& \eqvsym \Big(\sm{x':A}x=x'\Big)+\Big(\sm{y':B}\emptyt\Big) \\
& \eqvsym \Big(\sm{x':A}x=x'\Big)+\emptyt \\
& \eqvsym \sm{x':A}x=x'.
\end{align*}%
\end{samepage}%
In the last two equivalences we used \cref{ex:unit-laws-coprod}. This shows that the total space is contractible, since the latter type is contractible by \cref{thm:total_path}.
\end{proof}

\begin{proof}[Proof of \cref{thm:id-coprod-compute}]
The proof is now concluded with an application of \cref{thm:id_fundamental}, using \cref{lem:is-contr-total-eq-coprod}.
\end{proof}
\index{disjointness of coproducts|)}
\index{characterization of identity type!coproduct|)}
\index{identity type!coproduct|)}
\index{coproduct!identity type|)}
\index{coproduct!disjointness|)}

\begin{exercises}
  \exercise
  \begin{subexenum}
  \item \label{ex:is-emb-empty}Show that the map $\emptyt\to A$ is an embedding for every type $A$.\index{is an embedding!0 to A@{$\emptyt\to A$}}
  \item \label{ex:is-emb-inl-inr}Show that $\inl:A\to A+B$ and $\inr:B\to A+B$ are embeddings for any two types $A$ and $B$.
    \index{is an embedding!inl (for coproducts)@{$\inl$ (for coproducts)}}
    \index{is an embedding!inr (for coproducts)@{$\inr$ (for coproducts)}}
    \index{inl@{$\inl$}!is an embedding}
    \index{inr@{$\inr$}!is an embedding}
  \end{subexenum}
  \exercise Consider an equivalence $e:A\simeq B$. Construct an equivalence
  \begin{equation*}
    (e(x)=y)\simeq(x=e^{-1}(y))
  \end{equation*}
  for every $x:A$ and $y:B$.
  \exercise Show that\index{embedding!closed under homotopies}
  \begin{equation*}
    (f\htpy g)\to (\isemb(f)\leftrightarrow\isemb(g))
  \end{equation*}
  for any $f,g:A\to B$.
  \exercise \label{ex:emb_triangle}Consider a commuting triangle
  \begin{equation*}
    \begin{tikzcd}[column sep=tiny]
      A \arrow[rr,"h"] \arrow[dr,swap,"f"] & & B \arrow[dl,"g"] \\
      & X
    \end{tikzcd}
  \end{equation*}
  with $H:f\htpy g\circ h$. 
  \begin{subexenum}
  \item Suppose that $g$ is an embedding. Show that $f$ is an embedding if and only if $h$ is an embedding.\index{is an embedding!composite of embeddings}\index{is an embedding!right factor of embedding if left factor is an embedding}
  \item Suppose that $h$ is an equivalence. Show that $f$ is an embedding if and only if $g$ is an embedding.\index{is an embedding!left factor of embedding if right factor is an equivalence}
  \end{subexenum}
  \exercise \label{ex:is-equiv-is-equiv-functor-coprod}Consider two maps $f:A\to A'$ and $g:B \to B'$.
  \begin{subexenum}
  \item Show that if the map
    \begin{equation*}
      f+g:(A+B)\to (A'+B')
    \end{equation*}
    is an equivalence, then so are both $f$ and $g$ (this is the converse of \cref{ex:coproduct_functor_equivalence}).
  \item \label{ex:is-emb-coprod}Show that $f+g$ is an embedding if and only if both $f$ and $g$ are embeddings.
  \end{subexenum}
  \exercise \label{ex:htpy_total} 
  \begin{subexenum}
  \item Let $f,g:\prd{x:A}B(x)\to C(x)$ be two families of maps. Show that
    \begin{equation*}
      \Big(\prd{x:A}f(x)\htpy g(x)\Big)\to \Big(\tot{f}\htpy \tot{g}\Big). 
    \end{equation*}
  \item Let $f:\prd{x:A}B(x)\to C(x)$ and let $g:\prd{x:A}C(x)\to D(x)$. Show that
    \begin{equation*}
      \tot{\lam{x}g(x)\circ f(x)}\htpy \tot{g}\circ\tot{f}.
    \end{equation*}
  \item For any family $B$ over $A$, show that
    \begin{equation*}
      \tot{\lam{x}\idfunc[B(x)]}\htpy\idfunc.
    \end{equation*}
  \end{subexenum}
  \exercise \label{ex:id_fundamental_retr}Let $a:A$, and let $B$ be a type family over $A$. 
  \begin{subexenum}
  \item Use \cref{ex:htpy_total,ex:contr_retr} to show that if each $B(x)$ is a retract of $\id{a}{x}$, then $B(x)$ is equivalent to $\id{a}{x}$ for every $x:A$.
    \index{fundamental theorem of identity types!formulation with retractions}
  \item Conclude that for any family of maps
    \index{fundamental theorem of identity types!formulation with sections}
    \begin{equation*}
      f : \prd{x:A} (a=x) \to B(x),
    \end{equation*}
    if each $f(x)$ has a section, then $f$ is a family of equivalences.
  \end{subexenum}
  \exercise Use \cref{ex:id_fundamental_retr} to show that for any map $f:A\to B$, if
  \begin{equation*}
    \apfunc{f} : (x=y) \to (f(x)=f(y))
  \end{equation*}
  has a section for each $x,y:A$, then $f$ is an embedding.\index{is an embedding!if the action on paths have sections}
  \exercise \label{ex:path-split}We say that a map $f:A\to B$ is \define{path-split}\index{path-split} if $f$ has a section, and for each $x,y:A$ the map
  \begin{equation*}
    \apfunc{f}(x,y):(x=y)\to (f(x)=f(y))
  \end{equation*}
  also has a section. We write $\pathsplit(f)$\index{path-split(f)@{$\pathsplit(f)$}} for the type
  \begin{equation*}
    \sections(f)\times\prd{x,y:A}\sections(\apfunc{f}(x,y)).
  \end{equation*}
  Show that for any map $f:A\to B$ the following are equivalent:
  \begin{enumerate}
  \item The map $f$ is an equivalence.
  \item The map $f$ is path-split.
  \end{enumerate}
  \exercise \label{ex:fiber_trans}Consider a triangle
  \begin{equation*}
    \begin{tikzcd}[column sep=small]
      A \arrow[rr,"h"] \arrow[dr,swap,"f"] & & B \arrow[dl,"g"] \\
      & X
    \end{tikzcd}
  \end{equation*}
  with a homotopy $H:f\htpy g\circ h$ witnessing that the triangle commutes. 
  \begin{subexenum}
  \item Construct a family of maps
    \begin{equation*}
      \fibtriangle(h,H):\prd{x:X}\fib{f}{x}\to\fib{g}{x},
    \end{equation*}
    for which the square
    \begin{equation*}
      \begin{tikzcd}[column sep=8em]
        \sm{x:X}\fib{f}{x} \arrow[r,"\tot{\fibtriangle(h,H)}"] \arrow[d] & \sm{x:X}\fib{g}{x} \arrow[d] \\
        A \arrow[r,swap,"h"] & B
      \end{tikzcd}
    \end{equation*}
    commutes, where the vertical maps are as constructed in \cref{ex:fib_replacement}.
  \item Show that $h$ is an equivalence if and only if $\fibtriangle(h,H)$ is a family of equivalences.
  \end{subexenum}
\end{exercises}
\index{fundamental theorem of identity types|)}
\index{characterization of identity type!fundamental theorem of identity types|)}

\endinput

  \begin{comment}
    \exercise \label{ex:eqv_sigma_mv}Consider a map
    \begin{equation*}
      f:A \to \sm{y:B}C(y).
    \end{equation*}
    \begin{subexenum}
    \item Construct a family of maps
      \begin{equation*}
        f':\prd{y:B} \fib{\proj 1\circ f}{y}\to C(y).
      \end{equation*}
    \item Construct an equivalence
      \begin{equation*}
        \eqv{\fib{f'(b)}{c}}{\fib{f}{(b,c)}}
      \end{equation*}
      for every $(b,c):\sm{y:B}C(y)$.
    \item Conclude that the following are equivalent:
      \begin{enumerate}
      \item $f$ is an equivalence.
      \item $f'$ is a family of equivalences.
      \end{enumerate}
    \end{subexenum}
    \exercise \label{ex:coh_intro}Consider a type $A$ with base point $a:A$, and let $B$ be a type family on $A$ that implies the identity type, i.e., there is a term
    \begin{equation*}
      \alpha : \prd{x:A} B(x)\to (a=x).
    \end{equation*}
    Show that the \define{coherence reduction map}
    \begin{equation*}
      \cohreduction : \Big(\sm{y:B(a)}\alpha(a,y)=\refl{a}\Big) \to \Big(\sm{x:A}B(x)\Big)
    \end{equation*}
    defined by $\lam{(y,q)}(a,y)$ is an equivalence.
  \end{comment}

\input{hierarchy}
\input{funext}
\section{The univalence axiom}

\subsection{Equivalent forms of the univalence axiom}

The univalence axiom characterizes the identity type of the universe. Roughly speaking, it asserts that equivalent types are equal. It is considered to be an \emph{extensionality principle}\index{extensionality principle!for types} for types.

\begin{axiom}[Univalence]\label{axiom:univalence}
  The \define{univalence axiom}\index{univalence axiom} on a universe $\UU$ is the statement that for any $A:\UU$ the family of maps\index{equiv-eq@{$\equiveq$}}
\begin{equation*}
\equiveq : \prd{B:\UU} (\id{A}{B})\to(\eqv{A}{B}).
\end{equation*}
that sends $\refl{A}$ to the identity equivalence $\idfunc:\eqv{A}{A}$ is a family of equivalences.\index{identity type!of a universe} A universe satisfying the univalence axiom is referred to as a \define{univalent universe}\index{univalent universe}. If $\UU$ is a univalent universe we will write
$\eqequiv$\index{eq-equiv@{$\eqequiv$}}
for the inverse of $\equiveq$.
\end{axiom}

The following theorem is a special case of the fundamental theorem of identity types (\cref{thm:id_fundamental})\index{fundamental theorem of identity types}. Subsequently we will assume that any type is contained in a univalent universe.\index{axiom!univalence}

\begin{thm}\label{thm:univalence}
The following are equivalent:
\begin{enumerate}
\item The univalence axiom holds.
\item The type
\begin{equation*}
\sm{B:\UU}\eqv{A}{B}
\end{equation*}
is contractible for each $A:\UU$.
\item The principle of \define{equivalence induction}\index{equivalence induction}\index{induction principle!for equivalences} holds: for every $A:\UU$ and for every type family
\begin{equation*}
P:\prd{B:\UU} (\eqv{A}{B})\to \UU,
\end{equation*}
the map
\begin{equation*}
\Big(\prd{B:\UU}{e:\eqv{A}{B}}P(B,e)\Big)\to P(A,\idfunc[A])
\end{equation*}
given by $f\mapsto f(A,\idfunc[A])$ has a section.
\end{enumerate}
\end{thm}

\subsection{Univalence implies function extensionality}
One of the first applications of the univalence axiom was Voevodsky's theorem that the univalence axiom on a universe $\UU$ implies function extensionality for types in $\UU$. The proof uses the fact that weak function extensionality implies function extensionality.

We will also make use of the following lemma. Note that this statement was also part of \cref{lem:postcomp_equiv}. That exercise is solved using function extensionality. Since our present goal is to derive function extensionality from the univalence axiom, we cannot make use of that exercise.

\begin{lem}\label{lem:postcomp-equiv}
  For any equivalence $e:\eqv{X}{Y}$ in a univalent universe $\UU$, and any type $A$, the post-composition map
  \begin{equation*}
    e\circ\blank : (A \to X) \to (A\to Y)
  \end{equation*}
  is an equivalence.
\end{lem}

\begin{proof}
  The statement is obvious for the identity equivalence $\idfunc : \eqv{X}{X}$. Therefore the claim follows by equivalence induction, which is by \cref{thm:univalence} one of the equivalent forms of the univalence axiom.
\end{proof}

\begin{thm}\label{thm:funext-univalence}\index{univalence axiom!implies function extensionality}
  For any universe $\UU$, the univalence axiom on $\UU$ implies function extensionality on $\UU$.
\end{thm}

\begin{proof}
  Note that by \cref{thm:funext_wkfunext}\index{weak function extensionality} it suffices to show that univalence implies weak function extensionality, where we note that \cref{thm:funext_wkfunext} also holds when it is restricted to small types.
  
Suppose that $B:A\to \UU$ is a family of contractible types. Our goal is to show that the product $\prd{x:A}B(x)$ is contractible.
Since each $B(x)$ is contractible, the projection map $\proj 1:\big(\sm{x:A}B(x)\big)\to A$ is an equivalence by \cref{ex:proj_fiber}.

Now it follows by \cref{lem:postcomp-equiv} that $\proj1\circ\blank$ is an equivalence. Consequently, it follows from \cref{thm:contr_equiv} that the fibers of
\begin{equation*}
\proj 1\circ\blank : \Big(A\to \sm{x:A}B(x)\Big)\to (A\to A)
\end{equation*}
are contractible. In particular, the fiber at $\idfunc[A]$ is contractible. Therefore it suffices to show that $\prd{x:A}B(x)$ is a retract of $\sm{f:A\to\sm{x:A}B(x)}\proj 1\circ f=\idfunc[A]$. In other words, we will construct
\begin{equation*}
\begin{tikzcd}
\Big(\prd{x:A}B(x)\Big) \arrow[r,"i"] & \Big(\sm{f:A\to\sm{x:A}B(x)}\proj 1\circ f=\idfunc[A]\Big) \arrow[r,"r"] & \Big(\prd{x:A}B(x)\Big),
\end{tikzcd}
\end{equation*}
and a homotopy $r\circ i\htpy \idfunc$.

We define the function $i$ by
\begin{equation*}
  i(f) \defeq (\lam{x}(x,f(x)),\refl{\idfunc}).
\end{equation*}
To see that this definition is correct, we need to know that
\begin{equation*}
  \lam{x}\proj 1(x,f(x))\jdeq \idfunc.
\end{equation*}
This is indeed the case, by the $\eta$-rule\index{eta-rule@{$\eta$-rule}} for $\Pi$-types.

Next, we define the function $r$. Let $h:A\to \sm{x:A}B(x)$, and let $p:\proj 1 \circ h = \idfunc$. Then we have the homotopy $H\defeq\htpyeq(p):\proj 1 \circ h \htpy \idfunc$. Then we have $\proj 2(h(x)):B(\proj 1(h(x)))$ and we have the identification $H(x):\proj 1(h(x))=x$. Therefore we define $r$ by
\begin{equation*}
  r((h,p),x)\defeq \tr_B(H(x),\proj 2(h(x))).
\end{equation*}

We note that if $p\jdeq \refl{\idfunc}$, then $H(x)\jdeq\refl{x}$. In this case we have the judgmental equality $r((h,\refl{}),x)\jdeq\proj 2 (h(x))$. Thus we see that $r\circ i\jdeq \idfunc$ by another application of the $\eta$-rule for $\Pi$-types.
\end{proof}

\subsection{Propositional extensionality and posets}

\begin{thm}\label{thm:propositional-extensionality}
  Propositions satisfy \define{propositional extensionality}\index{propositional extensionality}\index{extensionality principle!for propositions}:
  for any two propositions $P$ and $Q$, the canonical map\index{bi-implication}\index{iff-eq@{$\iffeq$}}
  \begin{equation*}
    \iffeq:(P=Q)\to (P\leftrightarrow Q)
  \end{equation*}
  that sends $\refl{P}$ to $(\idfunc,\idfunc)$ is an equivalence. It follows that the type $\prop$ of propositions in $\UU$ is a set.\index{Prop@{$\prop$}!is a set}
\end{thm}

Note that for any $P:\prop$, we usually also write $P$ for the underlying type of the proposition $P$. If we would be more formal about it we would have to write $\proj 1(P)$ for the underlying type, since $\prop$ is the $\Sigma$-type $\sm{X:\UU}\isprop(X)$. In the following proof it is clearer if we use the more formal notation $\proj 1(P)$ for the underlying type of a proposition $P$.

\begin{proof}
  We note that the identity type $P=Q$ is an identity type in $\prop$. However, since $\isprop(X)$ is a proposition for any type $X$, it follows that the map
  \begin{equation*}
    \apfunc{\proj 1} : (P = Q) \to (\proj 1(P) = \proj 1(Q))
  \end{equation*}
  is an equivalence. Now we observe that we have a commuting square
  \begin{equation*}
    \begin{tikzcd}[column sep=huge]
      (P=Q) \arrow[d,swap,"\apfunc{\proj 1}"] \arrow[r] & (P\leftrightarrow Q) \\
      (\proj 1(P)=\proj 1(Q)) \arrow[r,swap,"\equiveq"] & (\proj 1(P)\simeq\proj 1(Q)) \arrow[u,swap,"\simeq"]
    \end{tikzcd}
  \end{equation*}
  Since the left, bottom, and right map are equivalences, it follows that the top map is an equivalence.
\end{proof}

\begin{defn}
  A \define{partially ordered set (poset)}\index{partially ordered set|see poset}\index{poset} is a set $P$ equipped with a relation
  \begin{equation*}
    \blank\leq\blank : P \to (P \to \prop)
  \end{equation*}
  that is \define{reflexive}\index{reflexive!poset} (for every $x:P$ we have $x\leq x$), \define{transitive}\index{transitive!poset} (for every $x,y,z:P$ such that $x\leq y$ and $y\leq z$ we have $x\leq z$), and \define{anti-symmetric}\index{anti-symmetric!poset} (for every $x,y:P$ such that $x\leq y$ and $y\leq x$ we have $x=y$).
\end{defn}

\begin{rmk}
  The condition that $X$ is a set can be omitted from the definition of a poset. Indeed, if $X$ is any type that comes equipped with a $\prop$-valued ordering relation $\leq$ that is reflexive and anti-symmetric, then $X$ is a set by \cref{lem:prop_to_id}.
\end{rmk}

\begin{eg}
  The type $\prop$ is a poset\index{Prop@{$\prop$}!is a poset}, where the ordering relation is given by implication: $P$ is less than $Q$ if $P\to Q$. The fact that $P\to Q$ is a proposition is a special case of \cref{cor:funtype_trunc}. The relation $P\to Q$ is reflexive by the identity function, and transitive by function composition. Moreover, the relation $P\to Q$ is anti-symmetric by \cref{thm:propositional-extensionality}\index{propositional extensionality}.
\end{eg}

\begin{eg}
  The type of natural numbers\index{natural numbers!is a poset with leq@{is a poset with $\leq$}}\index{natural numbers!is a poset with divisibility}\index{poset!N with leq@{$\N$ with $\leq$}}\index{poset!N with divisibility@{$\N$ with divisibility}} comes equipped with at least two important poset structures. The first is given by the usual ordering relation $\leq$, and the second is given by the relation $d\mid n$ that $d$ divides $n$.
\end{eg}

\begin{thm}
  For any poset $P$ and any type $X$, the set $P^X$ is a poset.\index{poset!closed under exponentials} In particular the type of subtypes of any type is a poset.\index{poset!type of subtypes}\index{subtype!poset}
\end{thm}

\begin{proof}
  Let $P$ be a poset with ordering $\leq$, and let $X$ be a type. Then $P^X$ is a set by \cref{cor:funtype_trunc}. For any $f,g:X\to P$ we define
  \begin{equation*}
    (f\leq g) \defeq \prd{x:X}f(x)\leq g(x).
  \end{equation*}
  Reflexivity and transitivity follow immediately from reflexivity and transitivity of the original relation. Moreover, by the anti-symmetry of the original relation it follows that
  \begin{equation*}
    (f\leq g)\times (g\leq f) \to (f\htpy g). 
  \end{equation*}
  Therefore we obtain an identification $f=g$ by function extensionality. The last claim follows immediately from the fact that a subtype of $X$ is a map $X\to\prop$, and the fact that $\prop$ is a poset.
\end{proof}

\begin{exercises}
\exercise \label{ex:istrunc_UUtrunc}
\begin{subexenum}
\item Use the univalence axiom to show that the type $\sm{A:\UU}\iscontr(A)$ of all contractible types in $\UU$ is contractible.\index{universe!of contractible types}
\item Use \cref{cor:emb_into_ktype,cor:funtype_trunc,ex:isprop_isequiv} to show that if $A$ and $B$ are $(k+1)$-types, then the type $\eqv{A}{B}$ is also a $(k+1)$-type.\index{A simeq B@{$\eqv{A}{B}$}!truncatedness}
\item Use univalence to show that the universe of $k$-types\index{universe!of k-types@{of $k$-types}}\index{U leq k@{$\UU^{\leq k}$}}\index{k-type@{$k$-type}!universe of k-types@{universe of $k$-types}}\index{truncated type!universe of k-types@{universe of $k$-types}}
\begin{equation*}
\UU^{\leq k}\defeq \sm{X:\UU}\istrunc{k}(X)
\end{equation*}
is a $(k+1)$-type, for any $k\geq -2$.
\item Show that $\UU^{\leq-1}$ is not a proposition.\index{universe!of propositions}
\item Show that $\eqv{(\eqv{\bool}{\bool})}{\bool}$, and conclude by the univalence axiom that the universe of sets\index{universe!of sets} $\UU^{\leq 0}$ is not a set. 
\end{subexenum}
\exercise Use the univalence axiom to show that the type $\sm{P:\prop}P$ is contractible.
\exercise Let $A$ and $B$ be small types. 
\begin{subexenum}
\item Construct an equivalence
\begin{equation*}
\eqv{(A\to (B\to\UU))}{\Big(\sm{S:\UU} (S\to A)\times (S\to B)\Big)}
\end{equation*}
\item We say that a relation $R:A\to (B\to\UU)$ is \define{functional}\index{relation!functional} if it comes equipped with a term of type\index{is-function(R)@{$\isfunction(R)$}}
\begin{equation*}
\isfunction(R) \defeq \prd{x:A}\iscontr\Big(\sm{y:B}R(x,y)\Big)
\end{equation*}
For any function $f:A\to B$, show that the \define{graph}\index{graph!of a function} of $f$ 
\begin{equation*}
\graph_f:A\to (B\to \UU)
\end{equation*}
given by $\graph_f(a,b)\defeq (f(a)=b)$ is a functional relation from $A$ to $B$.
\item Construct an equivalence
\begin{equation*}
\eqv{\Big(\sm{R:A\to (B\to\UU)}\isfunction(R)\Big)}{(A\to B)}
\end{equation*}
\item Given a relation $R:A\to (B\to \UU)$ we define the \define{opposite relation}\index{relation!opposite relation}\index{opposite relation}\index{op R@{$\opp{R}$}}
\begin{equation*}
\opp{R} : B\to (A\to\UU)
\end{equation*}
by $\opp{R}(y,x)\defeq R(x,y)$. Construct an equivalence\index{A simeq B@{$\eqv{A}{B}$}!as relation}
\begin{equation*}
\eqv{\Big(\sm{R:A\to (B\to \UU)}\isfunction(R)\times\isfunction(\opp{R})\Big)}{(\eqv{A}{B})}.
\end{equation*}
\end{subexenum}
\exercise
  \begin{subexenum}
  \item Show that $\isdecidable(P)$ is a proposition\index{is-decidable@{$\isdecidable$}!is a proposition}, for any proposition $P$.
  \item Show that $\classicalprop$%
    \index{classical-Prop@{$\classicalprop$}!classical-Prop bool@{$\classicalprop\eqvsym\bool$}} is equivalent to $\bool$.
  \end{subexenum}
\exercise Recall that $\UU_\ast$ is the universe of pointed types\index{UU*@{$\UU_\ast$}}.
  \begin{subexenum}
  \item For any $(A,a)$ and $(B,b)$ in $\UU_\ast$, write $(A,a)\simeq_\ast(B,b)$ for the type of \define{pointed equivalences}\index{pointed equivalence}\index{equivalence!pointed equivalence} from $A$ to $B$, i.e.,
    \begin{equation*}
      (A,a)\simeq_\ast (B,b)\defeq \sm{e:A\simeq B}e(a)=b.
    \end{equation*}
    Show that the canonical map\index{UU*@{$\UU_\ast$}!identity type}\index{identity type!of UU*@{of $\UU_\ast$}}
    \begin{equation*}
      \big((A,a)=(B,b)\big)\to \Big((A,a)\simeq (B,b)\Big)
    \end{equation*}
    sending $\refl{(A,a)}$ to the pair $(\idfunc,\refl{a})$, is an equivalence.
  \item Construct for any pointed type $(X,x_0)$ an equivalence
    \begin{equation*}
      \Big(\sm{P:X\to \UU}P(x_0)\Big)\simeq \sm{(A,a_0):\UU_\ast}(A,a_0)\to_\ast(X,x_0).
    \end{equation*}
  \end{subexenum}
\exercise Show that any subuniverse\index{subuniverse!closed under equivalences} is closed under equivalences, i.e., show that there is a map
  \begin{equation*}
    (X\simeq Y) \to (P(X)\to P(Y))
  \end{equation*}
  for any subuniverse $P:\UU\to\prop$, and any $X,Y:\UU$.
  \exercise Show that the universe inclusions
  \begin{equation*}
    \UU \to \UU^+\qquad\text{and}\qquad \UU\to \UU\sqcup\VV
  \end{equation*}
  defined in \cref{rmk:universe-constructions}, are embeddings.
\end{exercises}

\section{Propositional truncations}\label{sec:propositional-truncation}

The propositional truncation operation is a universal way of turning type a type $A$ into a proposition $\brck{A}$. Informally, the proposition $\brck{A}$ is the proposition that $A$ is inhabited. More precisely, the propositional truncation of $A$ comes equipped with a map $A\to\brck{A}$ and it is characterized by its universal property, which asserts that any map $A\to P$ into a proposition $P$ extends uniquely to a map $\brck{A}\to P$, as indicated in the diagram
\begin{equation*}
  \begin{tikzcd}
    A \arrow[dr] \arrow[d] \\
    \brck{A} \arrow[r,densely dotted] & P.
  \end{tikzcd}
\end{equation*}
Using the propositional truncation operation we can define many important mathematical concepts, including the image of a map, surjectivity, and connected components. We will discuss those topics in \cref{chap:image}.

\subsection{The universal property of propositional truncations}\label{sec:propositional-truncation-up}

\begin{defn}
Let $A$ be a type, and let $f:A\to P$ be a map into a proposition $P$. We say that $f$ is a \define{propositional truncation of $A$} if for every proposition $Q$, the precomposition map
\begin{equation*}
\blank\circ f:(P\to Q)\to (A\to Q)
\end{equation*}
is an equivalence. This property of $f$ is also called the \define{universal property of propositional truncation of $A$}\index{universal property!of propositional truncation}
\end{defn}

In other words, a map $f:A\to P$ into a proposition $P$ is a propositional truncation of $A$ if every map $g:A\to Q$ into a proposition extends uniquely along $f$, as indicated in the diagram
\begin{equation*}
  \begin{tikzcd}
    A \arrow[d,swap,"f"] \arrow[dr,"g"] \\
    P \arrow[r,densely dotted] & Q.
  \end{tikzcd}
\end{equation*}
Indeed, this unique extension property asserts that the type
\begin{equation*}
  \sm{h:P\to Q}h\circ f=g
\end{equation*}
is contractible for every $g:A\to Q$. In other words, the unique extension property asserts that the precomposition function $\blank\circ f:(P\to Q)\to (A\to Q)$ is a contractible map, which is the case if and only if it is an equivalence.

\begin{rmk}
  Note that if $Q$ is a proposition, then the type $X\to Q$ is a proposition for any type $X$. Furthermore, recall from \cref{ex:equiv-bi-implication} that the map $(P\to Q)\to (A\to Q)$ is an equivalence as soon as there is a map in the converse direction. Therefore, in order to prove the universal property of the propositional truncation it suffices to show that
  \begin{equation*}
    (A\to Q)\to (P\to Q).
  \end{equation*}
  We also note that the universal property of the propositional truncation of a type is formulated with respect to all propositions, regardless of the universe they live in. 
\end{rmk}

\begin{eg}
  Suppose $A$ is a type that comes equipped with a point $a:A$, such as the booleans, the type of natural numbers, or the loop space $\loopspace{A}$ of a pointed type. Then the constant map
  \begin{equation*}
    \const_\ttt: A\to\unit
  \end{equation*}
  is a propositional truncation of $A$. To see this, let $Q$ be an arbitrary proposition. It suffices to show that
  \begin{equation*}
    (A\to Q)\to (\unit\to Q).
  \end{equation*}
  To see this, let $f:A\to Q$. Then we have $f(a):Q$, so we define $\const_{f(a)}:\unit\to Q$. Thus we see that we have
  \begin{equation*}
    \lam{f}\const_{f(a)}:(A\to Q)\to (\unit\to Q).
  \end{equation*}
  This proves that $\const_\ttt:A\to \unit$ satisfies the universal property of the propositional truncation of $A$. 
\end{eg}

\begin{eg}
  If the type $A$ is already a proposition, then the identity function
  \begin{equation*}
    \idfunc:A\to A
  \end{equation*}
  is a propositional truncation of $A$. To see this, simply note that the precomposittion function with the identity function
  \begin{equation*}
    \blank\circ\idfunc : (A\to Q)\to (A\to Q)
  \end{equation*}
  is itself just the identity function. In particular, it is an equivalence.

  Similarly, any equivalence $e:P\simeq P'$ between propositions satisfies the universal property of the propositional truncation of $P$, since precomposing by an equivalence is an equivalence by \cref{ex:equiv_precomp}.
\end{eg}

The universal property of the propositional truncation determines the propositional truncation up to equivalence. Such proofs of uniqueness from a universal property always follow the same pattern.

\begin{prp}\label{prp:propositional-truncation-3-for-2}
  Let $A$ be a type, and consider a commuting triangle
  \begin{equation*}
    \begin{tikzcd}[column sep=tiny]
      \phantom{P'} & A \arrow[dl,swap,"f"] \arrow[dr,"{f'}"] \\
      P \arrow[rr,swap,"h"] & & P'
    \end{tikzcd}
  \end{equation*}
  where $P$ and $P'$ are propositions. If any two of the following three assertions hold, so does the third:
  \begin{enumerate}
  \item The map $f$ satisfies the universal property of the propositional truncation of $A$.
  \item The map $f'$ satisfies the universal propertyof the propositional truncation of $A$.
  \item The map $h$ is an equivalence.
  \end{enumerate}
\end{prp}

\begin{proof}
  Note that the map $h:P\to P'$ is an equivalence if and only if for every proposition $Q$, the precomposition map
  \begin{equation*}
    \blank\circ h:(P'\to Q)\to (P\to Q)
  \end{equation*}
  is an equivalence. Thus, the claim follows by observing that for every proposition $Q$ we have the triangle
  \begin{equation*}
    \begin{tikzcd}[column sep=-1em]
      (P'\to Q) \arrow[rr,"\blank\circ h"] \arrow[dr,swap,"\blank\circ {f'}"] & & (P\to Q) \arrow[dl,"\blank\circ f"] \\
      & (A\to Q). & \phantom{(P'\to Q)}
    \end{tikzcd}
  \end{equation*}
\end{proof}

\begin{cor}\label{cor:uniquely-unique-brck}
  Consider two maps $f:A\to P$ and $f':A\to P'$ into propositions $P$ and $P'$, both satisfying the universal property of the propositional truncation of $A$. Then the type of equivalences $e:P \simeq P'$ for which the triangle
  \begin{equation*}
    \begin{tikzcd}[column sep=tiny]
      \phantom{P'} & A \arrow[dl,swap,"f"] \arrow[dr,"{f'}"] \\
      P \arrow[rr,swap,"e"] & & P'
    \end{tikzcd}
  \end{equation*}
  commutes, is contractible.
\end{cor}

\begin{rmk}
  Note that the triangles in \cref{prp:propositional-truncation-3-for-2,cor:uniquely-unique-brck} always commutes, since $P$ and $P'$ are assumed to be propositions.
\end{rmk}

Now that we have shown that propositional truncations are determined uniquely, we will assume that any universe is closed under propositional truncations.

\begin{axiom}
  Any universe $\UU$ is closed under propositional truncations: for any type $A:\UU$ there is a proposition $\brck{A}:\UU$ equipped with a map $\eta:A\to\brck{A}$ that satisfies the universal property of the propositional truncation.
\end{axiom}

The propositional truncation is therefore an operation
\begin{equation*}
  \brck{\blank}:\UU\to\UU
\end{equation*}
on the universe. One simple application of the universal property of the propositional truncation is that $\brck{\blank}$ also acts on functions in a functorial way.

\begin{prp}
  There is a map
  \begin{equation*}
    \brck{\blank}:(A\to B)\to (\brck{A}\to\brck{B})
  \end{equation*}
  for any two types $A$ and $B$, such that
  \begin{align*}
    \brck{\idfunc} & \htpy \idfunc \\
    \brck{g\circ f} & \htpy \brck{g}\circ\brck{f}.
  \end{align*}
\end{prp}

\begin{proof}
  For any $f:A\to B$, the map $\brck{f}:\brck{A}\to\brck{B}$ is defined to be the unique extension
  \begin{equation*}
    \begin{tikzcd}
      A \arrow[d,swap,"\eta"] \arrow[r,"f"] & B \arrow[d,"\eta"] \\
      \brck{A} \arrow[r,densely dotted,swap,"\brck{f}"] & \brck{B}.
    \end{tikzcd}
  \end{equation*}
  To see that $\brck{\blank}$ preserves identity maps and compositions, simply note that $\idfunc[\brck{A}]$ is an extension of $\idfunc[A]$, and that $\brck{g}\circ\brck{f}$ is an extension of $g\circ f$. Hence the homotopies are obtained by uniqueness.
\end{proof}

\subsection{Propositional truncations as higher inductive types}

The idea of higher inductive types is that types can be generated inductively not only by point constructors, such as $\zeroN:\N$ and $\succN:\N\to\N$, but also by path constructors. One of the first examples of a higher inductive type was the propositional truncation of a type $A$. This is a type $\brck{A}$ equipped with one point constructor
\begin{equation*}
  \eta : A \to \brck{A},
\end{equation*}
one path constructor
\begin{equation*}
  \alpha : \prd{x,y:\brck{A}}x=y.
\end{equation*}
Note that the path constructor $\alpha$ immediately proves that $\brck{A}$ is a proposition. Now we should formulate the induction principle for the propositional truncation.

Just as we did with the universal property, we will formulate the induction principle of the propositional truncation for an arbitrary map $f:A\to P$ into a proposition $P$. When the induction principle is formulated in this way, we will be able to show that $f$ satisfies the universal property if and only if it satisfes the induction principle.

Consider a map $f:A\to P$ into a type equipped with a family of paths
\begin{equation*}
  \alpha : \prd{p,q:P}p=q
\end{equation*}
witnessing that $P$ is a proposition, and consider a type family $B$ over $P$. The induction principle of the propositional truncation of $A$ tells us what we have to do in order to construct a dependent function $h:\prd{p:P}B(p)$.

In order to figure out what the induction principle has to be, we first note that if we start with a dependent function $h:\prd{p:P}B(p)$, then we also obtain the function $h\circ f : \prd{x:A}B(f(x))$. In other words, if we think of $f:A\to P$ as the point constructor of $P$, then the function $h\circ f$ describes the action of $h$ on the points of $P$. The first requirement in the induction principle is therefore that $B$ must come equipped with a dependent function
\begin{equation*}
  g:\prd{x:A}B(f(x)).
\end{equation*}
Furthermore, the function $h$ acts on the paths in $P$ via its dependent action on paths, which we constructed in \cref{defn:apd}. The paths in $P$ are generated by $\alpha$, so we obtain a function
\begin{equation*}
  \lam{p}{q}\apd{h}{\alpha(p,q)} : \prd{p,q:P} \tr_B(\alpha(p,q),h(p))=h(q).
\end{equation*}
The induction principle must ensure that any function $h$ defined via the induction principle, satisfies this law. Therefore, the second condition in the induction principle is that we must have a family of identifications
\begin{equation*}
  \prd{p,q:P}{y:B(p)}{z:B(q)}\tr_B(\alpha(p,q),y)= z.
\end{equation*}
We now formulate the induction principle for propositional truncation.

\begin{defn}
  Consider a map $f:A\to P$ into a type $P$ equipped with a family of paths
  \begin{equation*}
    \alpha : \prd{p,q:P}p=q,
  \end{equation*}
  witnessing that $P$ is a proposition. We say that $f$ satisfies the induction principle for the propositional truncation of $A$ if for any family $B$ over $P$ that comes equipped with
  \begin{align*}
    g & : \prd{x:A}B(f(x)) \\
    \beta & : \prd{p,q:P}{y:B(p)}{z:B(q)}\tr_B(\alpha(p,q),y)=z,
  \end{align*}
  there is a dependent function $h:\prd{p:P}B(p)$ equipped with a homotopy
  \begin{equation*}
    \prd{x:A}h(f(x))= g(x).
  \end{equation*}
\end{defn}

In the following lemma we show that if a family $B$ over $\brck{A}$ comes equipped with a family of paths
\begin{equation*}
  \prd{x,y:\brck{A}}{u:B(x)}{v:B(y)}\tr_B(\alpha(x,y),u)=v,
\end{equation*}
then $B$ must be a family of propositions.

\begin{lem}\label{lem:case-paths-induction-principle-propositional-truncation}
  Let $P$ be a type equipped with a family of paths
  \begin{equation*}
    \alpha : \prd{p,q:P}p=q,
  \end{equation*}
  showing that $P$ is a proposition, and consider a type family $B$ over $P$. The following are equivalent:
  \begin{enumerate}
  \item The family $B$ comes equipped with a family of identifications
    \begin{equation*}
      \beta:\prd{p,q:P}{x:B(p)}{y:B(q)}\tr_B(\alpha(p,q),x)=y,
    \end{equation*}
  \item The family $B$ is a family of propositions.
  \end{enumerate}
\end{lem}

\begin{proof}
  Assuming that (i) holds, we will show that each $B(p)$ is a proposition by showing that
  \begin{equation*}
    B(p)\to\iscontr(B(p)).
  \end{equation*}
  Let $x:B(p)$. We have to construct a center of contraction and a contraction. Our plan is to use $\beta$ to define the contraction, so it is natural to define the center as $\tr_B(\alpha(p,p),x)$. Now we take
  \begin{equation*}
    \beta(p,p,x):\prd{y:B(p)}\tr_B(\alpha(p,p),x)=y
  \end{equation*}
  as our contraction. This completes the proof that $B$ is a family of propositions.

  The converse is immediate: if $B$ is a family of propositions, then any two terms in any $B(q)$ can be identified.
\end{proof}

\begin{defn}
  Consider a map $f:A\to P$ into a proposition $P$. We say that $f$ satisfies the dependent universal property of the propositional truncation of $A$, if for any family $Q$ of propositions over $P$, the precomposition map
  \begin{equation*}
    \blank\circ f : \Big(\prd{p:P}Q(p)\Big)\to\Big(\prd{x:A}Q(f(x))\Big)
  \end{equation*}
  is an equivalence.
\end{defn}

\begin{thm}
  Consider a map $f:A\to P$ into a proposition $P$. Then the following are equivalent:
  \begin{enumerate}
  \item The map $f$ is a propositional truncation.
  \item The map $f$ satisfies the dependent universal property of the propositional truncation.
  \item The map $f$ satisfies the induction principle of the propositional truncation.
  \end{enumerate}
\end{thm}

\begin{proof}
  We will first show that (i) and (ii) are equivalent. Of course, the universal property is a special case of the dependent universal property, so the fact that (ii) implies (i) is immediate. We now show that (i) implies (ii). Let $Q$ be a family of propositions over $P$, and consider the following commuting diagram:
  \begin{equation*}
    \begin{tikzcd}[column sep=8em]
      \Big(\sm{h:P\to P}\prd{p:P}Q(h(p))\Big) \arrow[r,"{\tot[\blank\circ f]{\blank\circ f}}"] \arrow[d,swap,"\choice^{-1}"] & \Big(\sm{g:A\to P}\prd{x:A}Q(g(x))\Big) \arrow[d,"\choice^{-1}"] \\
      \Big(P\to \sm{p:P}Q(p)\Big) \arrow[r,"\blank\circ f"] \arrow[d,swap,"\proj 1\circ\blank"]  & \Big(A \to \sm{p:P}Q(p)\Big) \arrow[d,"\proj 1\circ\blank"] \\
      \Big(P\to P\Big) \arrow[r,swap,"\blank\circ f"] & \Big(A\to P\Big)
    \end{tikzcd}
  \end{equation*}
  In this diagram the bottom map is an equivalence by the universal property of the propositional truncation of $A$. Note also that the type $\sm{p:P}Q(p)$ is a proposition by \cref{ex:istrunc_sigma}, so it follows that also the middle map is an equivalence. Furthermore, the type theoretic choice maps are equivalences by \cref{thm:choice}, so it also follows that the top map is an equivalence. Now we use \cref{thm:equiv-toto} to conclude that the family of maps
  \begin{equation*}
    \blank\circ f: \Big(\prd{p:P}Q(h(p))\Big)\to\Big(\prd{x:A}Q(h(f(x)))\Big)
  \end{equation*}
  indexed by $h:P\to P$ is a family of equivalences. The dependent universal property is now just a special case: take $h\jdeq\idfunc$. This completes the proof that (i) is equivalent to (ii).

  It remains to show that (ii) is equivalent to (iii). By \cref{lem:case-paths-induction-principle-propositional-truncation} it follows that the induction principle is equivalent to the property that for each family $Q$ of propositions over $P$, the precomposition map
  \begin{equation*}
    \blank\circ f : \Big(\prd{p:P}Q(p)\Big)\to\Big(\prd{x:A}Q(f(x))\Big)
  \end{equation*}
  has a section. Since the domain and codomain of this map are propositions by \cref{thm:trunc_pi}, we see that this precomposition map has a section if and only if it is an equivalence.
\end{proof}

\begin{exercises}
  \exercise Let $A$ be a type and let $P$ be a proposition, and suppose that $P$ is a retract of $A$. Show that the retraction $A\to P$ is a propositional truncation.
%  \exercise Show that the relation $x,y\mapsto\brck{x=y}$ is an equivalence relation, on any type.
  \exercise Consider two maps $f:A\to P$ and $g:B\to Q$ into propositions $P$ and $Q$. Recall from \cref{ex:istrunc_sigma} that the type $P\times Q$ is also a proposition. Show that if both $f$ and $g$ are propositional truncations then the map $f\times g : A\times B\to P\times Q$ is also a propositional truncation. Conclude that
  \begin{equation*}
    \brck{A\times B}\simeq \brck{A}\times\brck{B}. 
  \end{equation*}
  \exercise Consider two propositions $P$ and $Q$, and define
  \begin{align*}
    P\land Q & \defeq P\times Q\\
    P\vee Q & \defeq \brck{P+Q}.
  \end{align*}
  \begin{subexenum}
  \item Construct maps $i:P\to P\vee Q$ and $j:Q\to P\vee Q$.
  \item Prove the universal property of disjunction, i.e., show that for any proposition $R$, the map
    \begin{equation*}
      (P\vee Q\to R) \to (P\to R)\land (Q\to R) 
    \end{equation*}
    given by $h\mapsto (h\circ i,h\circ j)$ is an equivalence.
  \end{subexenum}
  \exercise Consider a family $P$ of propositions over a type $A$, and define
  \begin{align*}
    \forall_{(x:A)}P(x) & \defeq \prd{x:A}P(x) \\
    \exists_{(x:A)}P(x) & \defeq \Brck{\sm{x:A}P(x)}
  \end{align*}
  \begin{subexenum}
  \item Construct a map $i_a : P(a)\to \exists_{(x:A)}P(x)$ for each $a:A$.
  \item Prove the universal property of the existential quantification, i.e. show that for any proposition $Q$, the map
    \begin{equation*}
      \Big(\Big(\exists_{(x:A)}P(x)\Big)\to Q\Big)\to \Big(\forall_{(x:A)}(P(x)\to Q)\Big)
    \end{equation*}
    given by $h\mapsto \lam{x}h\circ i_x$, is an equivalence.
  \end{subexenum}
  \exercise Show that
  \begin{equation*}
    \eqv{\brck{A}}{\prd{P:\prop}(A\to P)\to P}
  \end{equation*}
  for any type $A:\UU$. This is called the \define{impredicative encoding} of the propositional truncation.
  % \exercise For any $B:A\to\UU$, construct an equivalence
  % \begin{equation*}
  %   \eqv{\Big(\exists_{(a:A)}\brck{B(a)}\Big)}{\brck{\sm{a:A}B(a)}}
  % \end{equation*}
\end{exercises}
\section{The image of a map and the replacement axiom}\label{chap:image}

The idea of the image of a map $f:A\to X$ is that it is, in a way, the least subtype of $X$ that contains all the values of $f$. More precisely, the image of $f$ is an embedding $i:\im(f)\hookrightarrow X$ that fits in a commuting triangle
\begin{equation*}
  \begin{tikzcd}[column sep=tiny]
    A \arrow[rr,"q"] \arrow[dr,swap,"f"] & & \im(f) \arrow[dl,hook,"i"] \\
    \phantom{\im(f)} & X
  \end{tikzcd}
\end{equation*}
and satisfies the \emph{universal property} of the image of $f$. The universal property of the image of $f$ asserts that if a subtype $B\hookrightarrow X$ contains all the values of $f$, then it contains the image of $f$.
%In other words, for  asserts that there is a unique map $h:\im(f)\to B$ for which the tetrahedron
%\begin{equation*}
%  \begin{tikzcd}[column sep=large]
%    A \arrow[rr] \arrow[dr,"q"] \arrow[dddr,swap,"f"] & & B \arrow[dddl,"m"] \\
%    & \im(f) \arrow[ur,densely dotted,"h"] \arrow[dd,"i"] \\ \\
%    & X
%  \end{tikzcd}
%\end{equation*}
%commutes.
The image of a map can be constructed using the propositional truncation operation. In fact, we can also go the other way around: The propositional truncation of a type $A$ is the image of the map $A\to\unit$.

The final topic of this section is the type theoretic replacement axiom. A specific instance of the replacement axiom asserts that the image of any map $f:A\to\UU$ is equivalent to a type in $\UU$, provided that $A$ is equivant to a type in $\UU$. This property will be used to construct quotients in type theory, much in the same way as quotients are constructed in set theory.

We should note that the existence of the propositional truncation operation and the replacement axiom will be assumed for now. However, once we assume that universes are closed under pushouts, we will be able to construct the propositional truncations and we will be able to prove the replacement axiom. These constructions will be given in \cref{sec:join-construction}.

\subsection{The image of a map}\label{sec:image-construction}
 %Note that there is quite a lot of information in this diagram: not only are there the three small commuting triangles; there is also the large commuting triange in the back, and there is a three-dimensional solid filling the space between the four triangles. We make the following definition, in order to express the universal property of the image efficiently.

\begin{defn}
  Let $f:A\to X$ and $g:B\to X$ be maps. A \define{morphism} from $f$ to $g$ over $X$ consists of a map $h:A\to B$ equipped with a homotopy $H:f\htpy g\circ h$ witnessing that the triangle
\begin{equation*}
\begin{tikzcd}[column sep=tiny]
A \arrow[rr,"h"] \arrow[dr,swap,"f"] & & B \arrow[dl,"g"] \\
& X
\end{tikzcd}
\end{equation*}
commutes. Thus, we define the type
\begin{equation*}
\mathrm{hom}_X(f,g)\defeq\sm{h:A\to B}f\htpy g\circ h.
\end{equation*}
Composition of morphisms over $X$ is defined by
\begin{equation*}
  (k,K)\circ (h,H) \defeq (k\circ h,\ct{H}{(K\cdot h)}).
\end{equation*}
\end{defn}

\begin{defn}
Consider a commuting triangle
\begin{equation*}
\begin{tikzcd}[column sep=tiny]
A \arrow[rr,"q"] \arrow[dr,swap,"f"] & & I \arrow[dl,"i"] \\
& X
\end{tikzcd}
\end{equation*}
with $H:f\htpy i\circ q$, where $i$ is an embedding\index{embedding}.
We say that $i$ has the \define{universal property of the image of $f$}\index{universal property!of the image} if the map
\begin{equation*}
\blank\circ(q,H) : \mathrm{hom}_X(i,m)\to\mathrm{hom}_X(f,m)
\end{equation*}
is an equivalence for every embedding $m:B\to X$. 
\end{defn}

\begin{rmk}
  Consider a commuting triangle
\begin{equation*}
\begin{tikzcd}[column sep=tiny]
A \arrow[rr,"q"] \arrow[dr,swap,"f"] & & I \arrow[dl,"i"] \\
& X
\end{tikzcd}
\end{equation*}
with $H:f\htpy i\circ q$, where $i$ is an embedding. Then it is not hard to see that the embedding $i$ satisfies the universal property of the image inclusion if and only if for every commuting triangle
\begin{equation*}
  \begin{tikzcd}[column sep=tiny]
    A \arrow[dr,swap,"f"] \arrow[rr,"g"] & & B \arrow[dl,"m"] \\
    & X
  \end{tikzcd}
\end{equation*}
with $G:f\htpy m\circ g$, where $m$ is an embedding, the type of quadruples $(h,K,L,M)$ consisting of
\begin{enumerate}
\item a map $h:I\to B$,
\item a homotopy $K:i\htpy m\circ h$ witnessing that the triangle
  \begin{equation*}
    \begin{tikzcd}[column sep=tiny]
      I \arrow[rr,"h"] \arrow[dr,swap,"i"] & & B \arrow[dl,"m"] \\
      & X
    \end{tikzcd}
  \end{equation*}
  commutes,
\item a homotopy $L:g\htpy h\circ q$ witnessing that the triangle
  \begin{equation*}
    \begin{tikzcd}[column sep=tiny]
      A \arrow[rr,"q"] \arrow[dr,swap,"g"] & & I \arrow[dl,"h"] \\
      & B
    \end{tikzcd}
  \end{equation*}
  commutes,
\item a homotopy $M:\ct{H}{(K\cdot q)}\htpy\ct{G}{(m\cdot L)}$ witnessing that the square
  \begin{equation*}
    \begin{tikzcd}
      f \arrow[d,swap,"H"] \arrow[r,"G"] & m\circ g \arrow[d,"m\cdot L"] \\
      i\circ q \arrow[r,swap,"K\cdot q"] & m\circ h\circ g
    \end{tikzcd}
  \end{equation*}
  commutes,
\end{enumerate}
is contractible. However, the situation is in fact much simpler, because the type $\mathrm{hom}_X(f,m)$ is a proposition whenever $m$ is an embedding.
\end{rmk}

\begin{rmk}
  Suppose that the map $f:A\to X$ has a section. Then the identity function
  \begin{equation*}
    \idfunc:X\to X
  \end{equation*}
  satisfies the universal property of the image of $f$. 
\end{rmk}

\begin{rmk}
  Suppose that $f:A\to X$ is already an embedding. Then $f$ itself satisfies the universal property of the image of $f$.
\end{rmk}

\begin{lem}
For any $f:A\to X$ and any embedding\index{embedding} $m:B\to X$, the type $\mathrm{hom}_X(f,m)$ is a proposition.
\end{lem}

\begin{proof}
  Recall from \cref{ex:triangle_fib} that the type $\mathrm{hom}_X(f,m)$ is equivalent to the type
  \begin{equation*}
    \prd{x:X}\fib{f}{x}\to\fib{m}{x}.
  \end{equation*}
  Therefore it suffices to show that this type is a proposition. Recall from \cref{cor:prop_emb} that a map is an embedding if and only if its fibers are propositions.
  Thus we see that the type $\prd{x:X}\fib{f}{x}\to\fib{m}{x}$ is a product of propositions, hence it is a proposition by \cref{thm:trunc_pi}.
\end{proof}

\begin{prp}\label{prp:simplifly-universal-property-image}
  Consider a commuting triangle
  \begin{equation*}
    \begin{tikzcd}[column sep=tiny]
      A \arrow[rr,"q"] \arrow[dr,swap,"f"] & & I \arrow[dl,"i"] \\
      & X
\end{tikzcd}
  \end{equation*}
  with $H:f\htpy i\circ q$, where $i$ is an embedding. Then the following are equivalent:
  \begin{enumerate}
  \item The embedding $i$ satisfies the universal property of the image inclusion of $f$.
  \item For every embedding $m:B\to X$ there is a map
    \begin{equation*}
      \mathrm{hom}_X(f,m)\to\mathrm{hom}_X(i,m).
    \end{equation*}
  \end{enumerate}
\end{prp}

\begin{proof}
Since $\mathrm{hom}_X(f,m)$ is a proposition for every every embedding $m:B\to X$, the claim follows immediately by \cref{ex:prop_equiv}.
\end{proof}

Just as in the cases for pullbacks and pushouts, the universal property of the image implies that the image is determined uniquely. We will show here that the type of image factorizations of any map is a proposition. In \cref{sec:image-construction} we will construct the image, after constructing the propositional truncation.

\begin{prp}
  Let $f$ be a map, and consider two commuting triangles
  \begin{equation*}
    \begin{tikzcd}[column sep=tiny]
      A \arrow[dr,swap,"f"] \arrow[rr,"q"] & & B \arrow[dl,"i"] &[2em] A \arrow[dr,swap,"f"] \arrow[rr,"{q'}"] & & B' \arrow[dl,"{i'}"] \\
      & X & & \phantom{B'} & X
    \end{tikzcd}
  \end{equation*}
  with $I:f\htpy i\circ q$ and $I':f\htpy i'\circ q'$, in which $i$ and $i'$ are assumed to be embeddings. Moreover, consider
  \begin{equation*}
    (h,H):\mathrm{hom}_X(i,i')
  \end{equation*}
  equipped with an identification $(h,H)\circ(q,I)=(q',I')$ in $\mathrm{hom}_X(f,i')$. Then, if any two of the following properties hold, so does the third:
  \begin{enumerate}
  \item The embedding $i$ satisfies the universal property of the image inclusion of $f$.
  \item The embedding $i'$ satisfies the universal property of the image inclusion of $f$.
  \item The map $h$ is an equivalence.
  \end{enumerate}
\end{prp}

\begin{proof}
  Consider an embedding $m:C\to X$. Then we have a commuting triangle
  \begin{equation*}
    \begin{tikzcd}[column sep=-1em]
      \mathrm{hom}_X(i',m) \arrow[rr,"{\blank\circ(h,H)}"] \arrow[dr,swap,"{\blank\circ(q',I')}"] & & \mathrm{hom}_X(i,m) \arrow[dl,"{\blank\circ(q,I)}"] \\
      & \mathrm{hom}_X(f,m), & \phantom{\mathrm{hom}_X(i',m)}
    \end{tikzcd}
  \end{equation*}
  so it follows that if any two of these maps are equivalences, then so is the third. The claim now follows by the observation that $\blank\circ(h,H)$ is an equivalence for every embedding $m:C\to X$ if and only if $h$ is an equivalence.
\end{proof}

\begin{cor}\label{cor:uniqueness-image}
  Consider two image factorizations
  \begin{equation*}
    \begin{tikzcd}[column sep=tiny]
      A \arrow[dr,swap,"f"] \arrow[rr,"q"] & & B \arrow[dl,"i"] &[2em] A \arrow[dr,swap,"f"] \arrow[rr,"{q'}"] & & B' \arrow[dl,"{i'}"] \\
      & X & & \phantom{B'} & X
    \end{tikzcd}
  \end{equation*}
  of a map $f$, with $I:f\htpy i\circ q$ and $I':f\htpy i'\circ q'$. Then the type of $(e,H):\mathrm{hom}_X(i,i')$ in which $e$ is an equivalence, equipped with an identification
  \begin{equation*}
    (e,H)\circ(q,I)=(q',I')
  \end{equation*}
  in $\mathrm{hom}_X(f,i')$, is contractible.
\end{cor}

The image of a map $f:A\to X$ can now be defined using the propositional truncation:

\begin{defn}
For any map $f:A\to X$ we define the \define{image}\index{image} of $f$ to be the type
\begin{equation*}
\im(f) \defeq \sm{x:X}\brck{\fib{f}{x}}.
\end{equation*}
Furthermore, we define:
\begin{enumerate}
\item The \define{image inclusion}
  \begin{equation*}
    i_f:\im(f)\to X
  \end{equation*}
  to be the projection $\proj 1$.
\item The map
  \begin{equation*}
    q_f:A\to\im(f)
  \end{equation*}
  to be the map given by $q_f(x)\defeq(f(x),\eta(x,\refl{f(x)}))$.
\item The homotopy $I_f:f\htpy i_f\circ q_f$ witnessing that the triangle
  \begin{equation*}
    \begin{tikzcd}[column sep=tiny]
      A \arrow[rr,"q_f"] \arrow[dr,swap,"f"] & & \im(f) \arrow[dl,"i_f"] \\
      \phantom{\im(f)} & X
    \end{tikzcd}
  \end{equation*}
  commutes, to be given by $I_f(x)\defeq\refl{f(x)}$.
\end{enumerate}
\end{defn}

\begin{prp}
  The image inclusion $i_f:\im(f)\to X$ of any map $f:A\to X$ is an embedding.
\end{prp}

\begin{proof}
  The fiber of $i_f$ at $x:X$ is equivalent to the type $\brck{\fib{f}{x}}$. In particular we see that the fibers are propositions, so $i_f$ is an embedding.
\end{proof}

\begin{thm}
  The image inclusion $i_f:\im(f)\to X$ of any map $f:A\to X$ satisfies the universal property of the image inclusion of $f$.
\end{thm}

\begin{proof}
  Consider an embedding $m:B\to X$. Note that we have a commuting square
  \begin{equation*}
    \begin{tikzcd}[column sep=6em]
      \mathrm{hom}_X(i_f,m) \arrow[d] \arrow[r] & \mathrm{hom}_X(f,m) \arrow[d] \\
      \Big(\prd{x:X}\fib{i_f}{x}\to\fib{m}{x}\Big) \arrow[r,swap,"h\mapsto{\lam{x}h_x\circ\varphi_x}"] & \Big(\prd{x:X}\fib{f}{x}\to\fib{m}{x}\Big)
    \end{tikzcd}
  \end{equation*}
  The vertical maps are of the form
  \begin{equation*}
    (h,H) \mapsto \lam{x}{(y,p)}(h(y),\ct{H(y)^{-1}}{p}),
  \end{equation*}
  and they are both equivalences. The map
  \begin{equation*}
    \varphi_x:\fib{f}{x}\to\fib{i_f}{x}
  \end{equation*}
  given by $\varphi_x(a,p)\defeq((h(a),\eta(a,p)),p)$ is a propositional truncation for every $x:X$. Therefore it follows that the map
  \begin{equation*}
    (\fib{i_f}{x}\to\fib{m}{x})\to(\fib{f}{x}\to\fib{m}{x})
  \end{equation*}
  is an equivalence, for every $x:X$. Thus we conclude that the bottom map in the above square is an equivalence, which implies that the top map is an equivalence. 
\end{proof}

\begin{eg}
  An important special case of the homotopy image of a map is the image of the terminal projection
\begin{equation*}
  \const_\ttt : A \to \unit,
\end{equation*}
which results in an embedding $I\hookrightarrow \unit$. Embeddings into the unit type are in fact just propositions. To see this, note that
\begin{align*}
\sm{A:\UU}{f:A\to\unit}\isemb(f)
& \eqvsym \sm{A:\UU}\isemb(\const_\ttt) \\
& \eqvsym \sm{A:\UU}\prd{x:\unit}\isprop(\fib{\const_\ttt}{x}) \\
& \eqvsym \sm{A:\UU}\isprop(\fib{\const_\ttt}{\ttt}) \\
& \eqvsym \sm{A:\UU}\isprop(A).
\end{align*}
Therefore, the universal property of the image of the map $A\to\unit$ is equivalently described as a proposition $P$ satisfying the universal property of the propositional truncation.
\end{eg}

\subsection{Surjective maps}

Another application of the propositional truncation is the notion of surjective map.

\begin{defn}
A map $f:A\to B$ is said to be \define{surjective} if there is a term of type
\begin{equation*}
\issurj(f)\defeq \prd{y:B}\brck{\fib{f}{b}}.
\end{equation*}
\end{defn}

\begin{eg}
Any equivalence is a surjective map, and so is any map that has a section (those are sometimes called \define{split epimorphisms}). Other examples include the base point inclusion $\unit\to\sphere{n}$ for any $n\geq 1$. 
\end{eg}

\begin{prp}\label{prp:surjective}
  Consider a map $f:A\to B$. Then the following are equivalent:
  \begin{enumerate}
  \item The map $f:A\to B$ is surjective.
  \item For any family $P$ of propositions over $B$, the precomposition map
    \begin{equation*}
      \blank\circ f : \Big(\prd{y:B}P(y)\Big)\to\Big(\prd{x:A}P(f(x))\Big)
    \end{equation*}
    is an equivalence.
  \end{enumerate}
\end{prp}

\begin{proof}
  Suppose first that $f$ is surjective, and consider the commuting square
  \begin{equation*}
    \begin{tikzcd}[column sep=6em]
      \Big(\prd{y:B}P(y)\Big) \arrow[r,"\blank\circ f"] \arrow[d,swap,"h\mapsto\lam{y}\const_{h(y)}"] & \Big(\prd{x:A}P(f(x))\Big)  \\
      \Big(\prd{y:B}\brck{\fib{f}{y}}\to P(y)\Big) \arrow[r,swap,"h\mapsto\lam{y}h(y)\circ\eta"] & \Big(\prd{y:B}\fib{f}{y}\to P(y)\Big) \arrow[u,swap,"{h\mapsto\lam{x}h(f(x),(x,\refl{f(x)}))}"]
    \end{tikzcd}
  \end{equation*}
  In this square, the bottom map is an equivalence by the universal property of the propositional truncation of $\fib{f}{y}$. The map on the right is also easily seen to be an equivalence. Furthermore, the map on the left is an equivalence by the assumption that $f$ is surjective, from which it follows that the types $\brck{\fib{f}{y}}$ are contractible. Therefore it follows that the top map is an equivalence, which completes the proof that (i) implies (ii).

  For the converse, it follows immediately from the assumption (ii) that
  \begin{equation*}
    \blank\circ f : \Big(\prd{y:B}\brck{\fib{f}{y}}\Big)\to\Big(\prd{x:A}\brck{\fib{f}{f(x)}}\Big)
  \end{equation*}
  is an equivalence. Hence it suffices to construct a term of type $\brck{\fib{f}{f(x)}}$ for each $x:A$. This is easy, because we have
  \begin{equation*}
    \eta(x,\refl{f(x)}):\brck{\fib{f}{f(x)}}.\qedhere.
  \end{equation*}
\end{proof}

\begin{thm}\label{thm:surjective}
Consider a commuting triangle
\begin{equation*}
\begin{tikzcd}[column sep=tiny]
A \arrow[rr,"q"] \arrow[dr,swap,"f"] & & B \arrow[dl,"m"] \\
& X
\end{tikzcd}
\end{equation*}
in which $m$ is an embedding. Then the following are equivalent:
\begin{enumerate}
\item The embedding $m$ satisfies the universal property of the image inclusion of $f$.
\item The map $q$ is surjective.
\end{enumerate}
\end{thm}

\begin{proof}
  First assume that $m$ satisfies the universal property of the image inclusion of $f$, and consider the composite function
  \begin{equation*}
    \begin{tikzcd}
      \Big(\sm{y:B}\brck{\fib{q}{y}}\Big) \arrow[r,"\proj 1"] & B \arrow[r,"m"] & X.
    \end{tikzcd}
  \end{equation*}
  Note that $m\circ\proj 1$ is a composition of embeddings, so it is an embedding. By the universal property of $m$ there is a unique map $h$ for which the triangle
  \begin{equation*}
    \begin{tikzcd}[column sep=0]
      B \arrow[dr,swap,"m"] \arrow[rr,densely dotted,"h"] & & \sm{y:B}\brck{\fib{q}{y}} \arrow[dl,"m\circ\proj 1"] \\
      \phantom{\sm{y:B}\brck{\fib{q}{y}}} & X
    \end{tikzcd}
  \end{equation*}
  commutes. Now note that $\proj 1\circ h$ is a map such that $m\circ (\proj 1\circ h)\htpy m$. The identity function is another map for which we have $m\circ\idfunc\htpy m$, so it follows by uniqueness that $\proj 1\circ h\htpy \idfunc$. In other words, the map $h$ is a section of the projection map. Therefore we obtain by \cref{ex:pi_sec} a dependent function
  \begin{equation*}
    \prd{b:B}\brck{\fib{q}{b}},
  \end{equation*}
  showing that $q$ is surjective.

  For the converse, suppose that $q$ is surjective. To prove that $m$ satisfies the universal property of the image factorization of $f$, it suffices to construct an equivalence
  \begin{equation*}
    \mathrm{hom}_X(f,m')\to\mathrm{hom}_X(m,m'),
  \end{equation*}
  for any embedding $m':B'\to X$. To see that there is such an equivalence, we make the following calculation
  \begin{align*}
    \mathrm{hom}_X(m,m') & \simeq \prd{x:X}\fib{m}{x}\to\fib{m'}{x} \\
                         & \simeq \prd{b:B}\fib{m'}{m(b)} \\
                         & \simeq \prd{a:A}\fib{m'}{m(q(a))} \\
                         & \simeq \prd{a:A}\fib{m'}{f(a)} \\
                         & \simeq \prd{x:X}\fib{f}{x}\to\fib{m'}{x} \\
                         & \simeq \mathrm{hom}_X(f,m').
  \end{align*}
  In this calculation, the first and last equivalence hold by \cref{ex:triangle_fib}. The second and second to last equivalences hold by \cref{ex:pi-fib}. The third equivalence holds by \cref{prp:surjective}, since $q$ is assumed to be surjective, and the fourth equivalence holds since we have a homotopy $f\htpy m\circ f$.
\end{proof}

\begin{cor}
  Every map factors uniquely as a surjective map followed by an embedding.
\end{cor}

\begin{proof}
  Consider a map $f:A\to X$, and two factorizations
  \begin{equation*}
    \begin{tikzcd}[column sep=tiny]
      A \arrow[rr,"q"] \arrow[dr,swap,"f"] & & B \arrow[dl,"i"] &[3em] A \arrow[rr,"{q'}"] \arrow[dr,swap,"f"] & & B' \arrow[dl,"{i'}"] \\
      & X & & & X
    \end{tikzcd}
  \end{equation*}
  of $f$ where $m$ and $m'$ are embeddings, and $q$ and $q'$ are surjective. Then both $m$ and $m'$ satisfy the universal property of the image factorization of $f$ by \cref{thm:surjective}. Now it follows by \cref{cor:uniqueness-image} that the type of $(e,H):\mathrm{hom}_X(i,i')$ in which $e$ is an equivalence, equipped with an identification
  \begin{equation*}
    (e,H)\circ(q,I)=(q',I')
  \end{equation*}
  in $\mathrm{hom}_X(f,i')$, is contractible.
\end{proof}

\subsection{Type theoretic replacement}

\begin{comment}
We have constructed the set quotient $A/R$ as the image of the equivalence relation
\begin{equation*}
  R:A\to \UU^A.
\end{equation*}
However, the type $\UU^A$ is itself in the next universe $\UU^+$. Hence the quotient is also in the universe $\UU^+$. We prove in this section that $A/R$ is nevertheless equivalent to a type in $\UU$. In other words, we show that $A/R$ is \emph{essentially} small.
\end{comment}

\begin{defn}\label{defn:ess_small}
\begin{enumerate}
\item A type $A$ is said to be \define{essentially small}\index{essentially small!type} if there is a type $X:\UU$ and an equivalence $\eqv{A}{X}$. We write\index{ess_small(A)@{$\mathsf{ess\usc{}small}(A)$}}
\begin{equation*}
\mathsf{ess\usc{}small}(A)\defeq\sm{X:\UU}\eqv{A}{X}.
\end{equation*}
\item A map $f:A\to B$ is said to be \define{essentially small}\index{essentially small!map} if for each $b:B$ the fiber $\fib{f}{b}$ is essentially small.
We write\index{ess_small(f)@{$\mathsf{ess\usc{}small}(f)$}}
\begin{equation*}
\mathsf{ess\usc{}small}(f)\defeq\prd{b:B}\mathsf{ess\usc{}small}(\fib{f}{b}).
\end{equation*}
\item A type $A$ is said to be \define{locally small}\index{locally small!type} if for every $x,y:A$ the identity type $x=y$ is essentially small.
We write\index{loc_small(A)@{$\mathsf{loc\usc{}small}(A)$}}
\begin{equation*}
\mathsf{loc\usc{}small}(A)\defeq \prd{x,y:A}\mathsf{ess\usc{}small}(x=y).
\end{equation*}
\end{enumerate}
\end{defn}

\begin{eg}
  \begin{enumerate}
  \item Any essentially $\UU$-small type is also locally $\UU$-small.
  \item Any univalent universe $\UU$ is locally $\UU$-small, because by the univalence axiom we have equivalences
    \begin{equation*}
      (A=B)\simeq (A\simeq B)
    \end{equation*}
    for each $A,B:\UU$, and the type $A\simeq B$ is in $\UU$.
  \item Any proposition is locally small with respect to any universe $\UU$.
  \item For any family $P$ of locally $\UU$-small types over a essentially $\UU$-small type $A$, the dependent product $\prd{x:A}P(x)$ is locally $\UU$-small. In particular, any type $A\to B$ of functions from an essentially small type into a locally small type is again locally small.
  \end{enumerate}
\end{eg}

\begin{lem}\label{lem:isprop_ess_small}
The type $\mathsf{ess\usc{}small}(A)$ is a proposition for any type $A$.\index{essentially small!is a proposition}
\end{lem}

\begin{proof}
Let $A$ be a type, not necessarily in $\UU$. In order to show that $\mathsf{ess\usc{}small}(A)$ is a proposition, we will use \cref{lem:isprop_eq} and show that for any $X:\UU$ and any equivalence $e:A\simeq X$, the type
\begin{equation*}
\sm{Y:\UU}\eqv{A}{Y}
\end{equation*}
is contractible. Note that we have an equivalence
\begin{equation*}
\eqv{\Big(\sm{Y:\UU}\eqv{X}{Y}\Big)}{\Big(\sm{Y:\UU}\eqv{A}{Y}\Big)}
\end{equation*}
because precomposing with the equivalence $e:A \simeq X$ is an equivalence. However, the type $\sm{Y:\UU}\eqv{X}{Y}$ is contractible by \cref{thm:univalence}. This shows that $\mathsf{ess\usc{}small}(A)$ is equivalent to a contractible type, assuming that $A$ is essentially small.
\end{proof}

\begin{cor}
For each function $f:A\to B$, the type $\mathsf{ess\usc{}small}(f)$ is a proposition, and for each type $X$ the type $\mathsf{loc\usc{}small}(X)$ is a proposition.
\end{cor}

\begin{proof}
This follows from the fact that propositions are closed under dependent products, established in \cref{thm:trunc_pi}.
\end{proof}

Recall that in set theory, the replacement axiom asserts that for any family of sets $\{X_i\}_{i\in I}$ indexed by a set $I$, there is a set $X[I]$ consisting of precisely those sets $x$ for which there exists an $i\in I$ such that $x\in X_i$. In other words: the image of a set-indexed family of sets is again a set. Without the replacement axiom, $X[I]$ would be a class. In the following corollary we establish a type-theoretic analogue of the replacement axiom: the image of a family of small types indexed by a small type is again (essentially) small.

\begin{axiom}\label{axiom:replacement}
  For any map $f:A\to B$ from an essentially small type $A$ into a locally small type $B$, the image of $f$ is again essentially small.
\end{axiom}

\begin{eg}
  For any type $A:\UU$, the image of the constant map $\const_A:\unit\to \UU$ is essentially small. This image is called the \define{connected component} of the universe at $A$. To see why, let us calculate
  \begin{align*} 
    \im(\const_A) & \jdeq \sm{X:\UU}\Brck{\fib{\const_A}{X}} \\
                  & \jdeq \sm{X:\UU}\Brck{\sm{t:\unit}A=X} \\
                  & \simeq \sm{X:\UU}\brck{A=X}.
  \end{align*}
  We see that the image of $\const_A:\unit\to\UU$ is the type of all types that are \emph{merely} equal to $A$. In other words, they are equal to $A$ in an unspecified way.
\end{eg}

\begin{eg}
  The type $\F$ of all finite types is defined to be the image of the map
  \begin{equation*}
    \Fin : \N\to\UU_0
  \end{equation*}
  By the replacement axiom, this type is essentially small. 
\end{eg}

\begin{exercises}
  \exercise Consider a map $f:A\to P$ into a proposition $P$. Show that the following are equivalent:
  \begin{enumerate}
  \item The map $f$ is a propositional truncation of $A$.
  \item The map $f$ is surjective.
  \end{enumerate}
  \exercise Consider a map $f:A\to B$. Show that the following are equivalent:
  \begin{enumerate}
  \item $f$ is an equivalence.
  \item $f$ is both surjective and an embedding.
  \end{enumerate}
  \exercise Consider a commuting triangle
  \begin{equation*}
    \begin{tikzcd}[column sep=tiny]
      A \arrow[rr,"h"] \arrow[dr,swap,"f"] & & B \arrow[dl,"g"] \\
      & X
    \end{tikzcd}
  \end{equation*}
  with $H:f\htpy g\circ h$, and assume that $h$ is surjective. Show that the following are equivalent:
  \begin{enumerate}
  \item The map $f$ is surjective.
  \item The map $g$ is surjective.
  \end{enumerate}
  \exercise \label{ex:surjective-precomp}Consider a map $f:A\to B$. Show that the following are equivalent:
  \begin{enumerate}
  \item The map $f$ is surjective.
  \item For every set $C$, the precomposition function
    \begin{equation*}
      \blank\circ f:(B\to C)\to (A\to C)
    \end{equation*}
    is an embedding.
  \end{enumerate}
  Hint: To show that (ii) implies (i), use the assumption with the set $C\jdeq\prop_\UU$, where $\UU$ is a univalent universe containing both $A$ and $B$.
  \exercise Let us say that a type family $B$ over $A$ is \define{univalent} if the map
  \begin{equation*}
    (x=y)\to (B(x)\simeq B(y))
  \end{equation*}
  is an equivalence, for every $x,y:A$.
  \begin{subexenum}
  \item Show that a family $B:A\to\UU$ is univalent if and only if the map $B:A\to\UU$ is an embedding.
  \item For any family $B:A\to\UU$, show that the type family $\hat{B}:\hat{A}\to\UU$ defined by
    \begin{align*}
      \hat{A} & \defeq \im(B) \\
      \hat{B}(X,p) & \defeq X
    \end{align*}
    is univalent.
  \item For any two families $B:A\to\UU$ and $D:C\to\mathcal{V}$, define the type of \define{cartesian morphisms}
    \begin{equation*}
      \carthomFam((A,B),(C,D)) \defeq \sm{f:A\to C}\prd{x:A}B(x)\simeq D(f(x)).
    \end{equation*}
    Construct a cartesian morphism
    \begin{equation*}
      (\eta,\alpha) : \carthomFam((A,B),(\hat{A},\hat{B})).
    \end{equation*}
  \item Show that for any family $B:A\to\UU$ and any \emph{univalent} family $D:C\to\mathcal{V}$, the map
    \begin{equation*}
      \carthomFam((\hat{A},\hat{B}),(C,D))\to\carthomFam((A,B),(C,D))
    \end{equation*}
    given by
    \begin{equation*}
      (f,e)\mapsto (f\circ\eta,\lam{x}e_x\circ \alpha_x)
    \end{equation*}
    is an equivalence. This is the \define{universal property} of the univalent completion of $A$.
  \end{subexenum}
  %\exercise \label{also}(Mart\'in Escard\'o) For any two propositions $P$ and $Q$, define
  %\begin{equation*}
  %P\boxplus Q \defeq ((P\to Q)\to Q)\times ((Q\to P)\to P).
  %\end{equation*}
  %\begin{subexenum}
  %\item Show that $P\lor Q\to P\boxplus Q$ and $P\boxplus Q\to\neg(\neg P\land \neg Q)$.
  %\end{subexenum}
  %\item \label{ex:brck_comp} Formulate the computation rule corresponding to the path constructor $\mu$. That is, compute the type of $\apd{\rec{\brck{\blank}}(f,g)}{\mu(x,y)}$, and find a canonical element in it.
  %\exercise Let $f:A\to X$ be a map. Construct an equivalence
  %\begin{equation*}
  %\eqv{\Big(\sm{y:\mathsf{join\usc{}power}_X(n,A)}f(x)=f^{\ast n}(y)\Big)}{\Big(\sm{y:A}f(x)=f(y)\Big)^{\ast n}}
  %\end{equation*}
  %for any $x:A$.
\end{exercises}

\endinput

\begin{thm}
Consider a commuting triangle
\begin{equation*}
\begin{tikzcd}[column sep=small]
A \arrow[rr,"i"] \arrow[dr,swap,"f"] & & B \arrow[dl,"m"] \\
& X
\end{tikzcd}
\end{equation*}
with $I:f\htpy m\circ i$, where $m$ is an embedding. The following are equivalent:
\begin{enumerate}
\item $m$ satisfies the universal property of the image of $f$.
\item for each $x:X$, the proposition $\fib{m}{x}$ satisfies the universal property of the propositional truncation of $\fib{f}{x}$.
\end{enumerate}
\end{thm}


\chapter{Topics in univalent mathematics}\label{chap:univalent-mathematics}
\input{number-theory}
\input{sets}
\input{groups}
\input{circle}
\section{The fundamental cover of the circle}
\index{circle!fundamental cover|(}
\index{fundamental cover!of the circle|(}

In this section we show that the loop space of the circle is equivalent to $\mathbb{Z}$ by constructing the universal cover of the circle as an application of the univalence axiom. 

\subsection{Families over the circle}

The type of small families over $\sphere{1}$ is just the function type $\sphere{1}\to\UU$, so in fact we may use the universal property of the circle to construct small dependent types over the circle. 
By the universal property, small type families over $\sphere{1}$ are equivalently described as pairs $(X,p)$ consisting of a type $X:\UU$ and an identification $p:X=X$.
This is where the univalence axiom\index{univalence axiom!families over $\sphere{1}$} comes in. By the map
\begin{equation*}
\mathsf{eq\usc{}equiv}_{X,X}:(\eqv{X}{X})\to (X=X)
\end{equation*}
it suffices to provide an equivalence $\eqv{X}{X}$.

\begin{defn}\label{defn:circle_descent}
Consider a type $X$ and every equivalence $e:\eqv{X}{X}$.
We will construct a dependent type $\mathcal{D}(X,e):\sphere{1}\to\UU$ with an equivalence $x\mapsto x_{\mathcal{D}}:\eqv{X}{\mathcal{D}(X,e,\base)}$ for which the square
\begin{equation*}
\begin{tikzcd}
X \arrow[r,"\eqvsym"] \arrow[d,swap,"e"] & \mathcal{D}(X,e,\base) \arrow[d,"\mathsf{tr}_{\mathcal{D}(X,e)}(\lloop)"] \\
X \arrow[r,swap,"\eqvsym"] & \mathcal{D}(X,e,\base)
\end{tikzcd}
\end{equation*}
commutes. We also write $d\mapsto d_{X}$ for the inverse of this equivalence, so that the relations
\begin{samepage}%
\begin{align*}
(x_{\mathcal{D}})_X & =x & (e(x)_{\mathcal{D}}) & = \mathsf{tr}_{\mathcal{D}(X,e)}(\lloop,x_{\mathcal{D}}) \\
(d_X)_{\mathcal{D}} & =d & (\mathsf{tr}_{\mathcal{D}(X,e)}(d))_X & = e(d_X)
\end{align*}
\end{samepage}%
hold.

The type $\sm{X:\UU}\eqv{X}{X}$ is also called the type of \define{descent data}\index{descent data!for the circle} for the circle.
\end{defn}

\begin{constr}
  An easy path induction argument reveals that
\begin{equation*}
\mathsf{equiv\usc{}eq}(\ap{P}{\lloop})=\mathsf{tr}_P(\lloop)
\end{equation*}
for each dependent type $P:\sphere{1}\to\UU$. Therefore we see that the triangle\index{desc_S1@{$\mathsf{desc}_{\sphere{1}}$}}
\begin{equation*}
\begin{tikzcd}
& (\sphere{1}\to \UU) \arrow[dl,swap,"\mathsf{gen}_{\sphere{1}}"] \arrow[dr,"\mathsf{desc}_{\sphere{1}}"] \\
\sm{X:\UU}X=X \arrow[rr,swap,"\tot{\lam{X}\mathsf{equiv\usc{}eq}_{X,X}}"] & & \sm{X:\UU}\eqv{X}{X}
\end{tikzcd}
\end{equation*}
commutes, where the map $\mathsf{desc}_{\sphere{1}}$ is given by $P\mapsto\pairr{P(\base),\mathsf{tr}_P(\lloop)}$ and the bottom map is an equivalence by the univalence axiom and \cref{thm:fib_equiv}.
Now it follows by the 3-for-2 property that $\mathsf{desc}_{\sphere{1}}$ is an equivalence, since $\mathsf{gen}_{\sphere{1}}$ is an equivalence by \cref{thm:circle_up}.
This means that for every type $X$ and every $e:\eqv{X}{X}$ there is a type family $\mathcal{D}(X,e):\sphere{1}\to\UU$ such that
\begin{equation*}
\pairr{\mathcal{D}(X,e,\base),\mathsf{tr}_{\mathcal{D}(X,e)}(\lloop)}=\pairr{X,e}.
\end{equation*}
Equivalently, we have $p:\id{\mathcal{D}(X,e,\base)}{X}$ and $\mathsf{tr}(p,{\mathsf{tr}_{\mathcal{D}(X,e)}(\lloop)})=e$. Thus, we obtain $\mathsf{equiv\usc{}eq}(p):\eqv{\mathcal{D}(X,e,\base)}{X}$, for which the square
\begin{equation*}
\begin{tikzcd}[column sep=huge]
\mathcal{D}(X,e,\base)\arrow[r,"\mathsf{equiv\usc{}eq}(p)"] \arrow[d,swap,"\mathsf{tr}_{\mathcal{D}(X,e)}(\lloop)"] & X \arrow[d,"e"] \\
\mathcal{D}(X,e,\base)\arrow[r,swap,"\mathsf{equiv\usc{}eq}(p)"] & X
\end{tikzcd}
\end{equation*}
commutes.
\end{constr}

\begin{comment}
\begin{defn}\label{defn:fiber_sequence}
A \define{fiber sequence} 
\begin{equation*}
F \hookrightarrow E \twoheadrightarrow B
\end{equation*}
consists of a \define{base type} $B$ with a base point $b_0$ and a dependent type $P:B\to\type$, a type $F$ called the \define{fiber} with an equivalence $\eqv{P(b_0)}{F}$, and a type $E$ called the \define{total space} with a map $p:E\to B$ and an equivalence $e:\eqv{(\sm{b:B}P(b))}{E}$ such that the triangle
\begin{equation*}
\begin{tikzcd}
\Big(\sm{b:B}P(b)\Big) \arrow[rr,"e"] \arrow[dr,swap,"\proj 1"] & & E \arrow[dl,"p"] \\
& B
\end{tikzcd}
\end{equation*}
commutes.
\end{defn}
\end{comment}

\subsection{The fundamental cover of the circle}

The \emph{fundamental cover}\index{fundamental cover!of the circle} of the circle is a family of sets over the circle with contractible total space.
Classically, the fundamental cover is described as a map $\mathbb{R}\to\sphere{1}$ that winds the real line around the circle.
In homotopy type theory there is no analogue of such a construction.

Recall from \cref{ex:succ_equiv} that the successor function $\mathsf{succ}:\Z\to \Z$ is an equivalence. Its inverse is the predecessor function defined in \cref{ex:int_pred}. 

\begin{defn}
The \define{fundamental cover}\index{fundamental cover!of the circle} of the circle is the dependent type $\mathcal{E}_{\sphere{1}}\defeq\mathcal{D}(\Z,\mathsf{succ}):\sphere{1}\to\UU$.\index{Z@{$\Z$}!fundamental cover of S1@{fundamental cover of $\sphere{1}$}}\index{E_S1@{$\mathcal{E}_{\sphere{1}}$}}
\end{defn}

\begin{rmk}
  The fundamental cover of the circle comes equipped with an equivalence
  \begin{equation*}
    e:\mathbb{Z} \simeq \mathcal{E}_{\sphere{1}}(\mathsf{base})
  \end{equation*}
  and a homotopy witnessing that the square
  \begin{equation*}
    \begin{tikzcd}
      \mathbb{Z} \arrow[r,"e"] \arrow[d,swap,"\mathsf{succ}"] & \mathcal{E}_{\sphere{1}}(\mathsf{base}) \arrow[d,"\mathsf{tr}_{\mathcal{E}_{\sphere{1}}}(\mathsf{loop})"] \\
      \mathbb{Z} \arrow[r,swap,"e"] & \mathcal{E}_{\sphere{1}}(\mathsf{base})
    \end{tikzcd}
  \end{equation*}
  commutes.

  For convenience, we write $k_{\mathcal{E}}$ for the term $e(k):\mathcal{E}_{\sphere{1}}(\mathsf{base})$, for any $k:\mathbb{Z}$. 
\end{rmk}

The picture of the fundamental cover is that of a helix\index{helix} over the circle. This picture emerges from the path liftings of $\mathsf{loop}$ in the total space. The segments of the helix connecting $k$ to $k+1$ in the total space of the helix, are constructed in the following lemma.

\begin{lem}
For any $k:\Z$, there is an identification
\begin{equation*}
\mathsf{segment\usc{}helix}_k:(\base,k_{\mathcal{E}})=(\base,\mathsf{succ}(k)_{\mathcal{E}})
\end{equation*}
in the total space $\sm{t:\sphere{1}}\mathcal{E}(t)$.
\end{lem}

\begin{proof}
By \cref{thm:eq_sigma} it suffices to show that
\begin{equation*}
\prd{k:\Z} \sm{\alpha:\base=\base} \mathsf{tr}_{\mathcal{E}}(\alpha,k_{\mathcal{E}})= \mathsf{succ}(k)_{\mathcal{E}}.
\end{equation*}
We just take $\alpha\defeq\lloop$. Then we have $\mathsf{tr}_{\mathcal{E}}(\alpha,k_{\mathcal{E}})= \mathsf{succ}(k)_{\mathcal{E}}$ by the commuting square provided in the definition of $\mathcal{E}$.
\end{proof}

\subsection{Contractibility of general total spaces}
Consider a type $X$, a family $P$ over $X$, and a term $c:\sm{x:X}P(x)$, and suppose our goal is to construct a contraction
\begin{equation*}
  \prd{t:\sm{x:X}P(x)}c=t.
\end{equation*}
Of course, the first step is to apply the induction principle of $\Sigma$-types, so it suffices to construct a term of type
\begin{equation*}
\prd{x:X}{y:P(x)} c = (x,y).
\end{equation*}
In the case where $P$ is the fundamental cover of the circle, we are given an equivalence $e:\eqv{\Z}{\mathcal{E}(\base)}$. Using this equivalence, we obtain an equivalence
\begin{equation*}
  \Big(\prd{y:\mathcal{E}(y)}c=(\mathsf{base},y)\Big)\to \Big(\prd{k:\Z}c=(\mathsf{base},k_{\mathcal{E}})\Big).
\end{equation*}
More generally, if we are given an equivalence $e:\eqv{F}{P(x)}$ for some $x:X$, then we have an equivalence
\begin{equation}
\Big(\prd{y:P(x)}c=(x,y)\Big) \to \Big(\prd{y:F}c=(x,e(y))\Big)
\end{equation}
by precomposing with the equivalence $e$. Therefore we can construct a term of type $\prd{y:P(x)}c=(x,y)$ by constructing a term of type $\prd{y:F}c=(x,e(y))$. 

Furthermore, if we consider a path $p:x=x'$ in $X$ and a commuting square
  \begin{equation*}
    \begin{tikzcd}
      F \arrow[r,"e"] \arrow[d,swap,"f"] & P(x) \arrow[d,"\mathsf{tr}_P(p)"] \\
      F' \arrow[r,"{e'}"] & P(x')
    \end{tikzcd}
  \end{equation*}
  where $e$, $e'$, and $f$ are all equivalences, then we obtain a function
  \begin{equation*}
    \psi : \Big(\prd{y:F}c=(x,e(y))\Big)\to \Big(\prd{y':F'}c=(x,e'(y'))\Big).
  \end{equation*}
  The function $\psi$ is constructed as follows. Given $h:\prd{y:F}c=(x,e(y))$ and $y':F'$ we have the path $h(f^{-1}(y')):c=(x,e(f^{-1}(y')))$. Moreover, writing $G$ for the homotopy $f\circ f^{-1} \htpy\idfunc$, we have the path
  \begin{equation*}
    \begin{tikzcd}[column sep=huge]
      {\mathsf{tr}_P(p,e(f^{-1}(y')))} \arrow[r,equals,"{H(f^{-1}(y'))}"] &
      {e'(f(f^{-1}(y')))} \arrow[r,equals,"\ap{e'}{G(y')}"] &
      {e'(y')}.
    \end{tikzcd}
  \end{equation*}
  From this concatenated path we obtain the path
  \begin{equation*}
    \begin{tikzcd}[column sep=14em]
      {(x,e(f^{-1}(y')))} \arrow[r,equals,"{\mathsf{eq\usc{}pair}(p,\ct{H(f^{-1}(y'))}{\ap{e'}{G(y')}})}"] & {(x',e'(y'))}.
    \end{tikzcd}
  \end{equation*}
  Now we define the function $\psi$ by
  \begin{equation*}
    h\mapsto \lam{y'}\ct{h(f^{-1}(y'))}{\mathsf{eq\usc{}pair}(p,\ct{H(f^{-1}(y'))}{\ap{e'}{G(y')}})}.
  \end{equation*}
  Note that $\psi$ is an equivalence, since it is given as precomposition by the equivalence $f^{-1}$, followed by postcomposition by concatenation, which is also an equivalence. Now we state the main technical result of this section, which will help us prove the contractibility of the total space of the fundamental cover of the circle by computing transport in the family $x\mapsto \prd{y:P(x)}c=(x,y)$.

  \begin{defn}
    Consider a path $p:x=x'$ in $X$ and a commuting square
    \begin{equation*}
      \begin{tikzcd}
        F \arrow[r,"e"] \arrow[d,swap,"f"] & P(x) \arrow[d,"\mathsf{tr}_P(p)"] \\
        F' \arrow[r,"{e'}"] & P(x')
      \end{tikzcd}
    \end{equation*}
    with $H:e'\circ f ~ \mathsf{tr}_P(p)\circ e$, where $e$, $e'$, and $f$ are all equivalences. Then there is for any $y:F$ an identification
    \begin{equation*}
      \mathsf{segment\usc{}tot}(y):(x,e(y))=(x',e'(f(y)))
    \end{equation*}
    defined as $\mathsf{segment\usc{}tot}(y)\defeq\mathsf{eq\usc{}pair}(p,H(y)^{-1})$.
  \end{defn}

  \begin{lem}\label{lem:compute-tr-contraction}
    Consider a path $p:x=x'$ in $X$ and a commuting square
    \begin{equation*}
      \begin{tikzcd}
        F \arrow[r,"e"] \arrow[d,swap,"f"] & P(x) \arrow[d,"\mathsf{tr}_P(p)"] \\
        F' \arrow[r,"{e'}"] & P(x')
      \end{tikzcd}
    \end{equation*}
    with $H:e'\circ f ~ \mathsf{tr}_P(p)\circ e$, where $e$, $e'$, and $f$ are all equivalences. Furthermore, let
    \begin{align*}
      h & : \prd{y:F}c=(x,e(y)) \\
      h' & : \prd{y':F'}c=(x',e'(y')).
    \end{align*}
    Then there is an equivalence
    \begin{equation*}
      \Big(\prd{y:F} h'(f(y))=\ct{h(y)}{\mathsf{segment\usc{}tot}(y)}\Big)
      \simeq \Big(\mathsf{tr}_C(p,\varphi(h))= \varphi'(h')\Big).
    \end{equation*}
  \end{lem}

  \begin{proof}
    We first note that we have a commuting square
    \begin{equation*}
      \begin{tikzcd}
        \prd{y:B(x)}c=(x,y) \arrow[r,"\blank\circ e"] \arrow[d,swap,"\mathsf{tr}_C(p)"] & \prd{y:F}c=(x,e(y)) \\
        \prd{y':B(x')}c=(x',y') \arrow[r,swap,"\blank\circ {e'}"] & \prd{y':F'}c=(x',e'(y')) \arrow[u,swap,"\psi"]
      \end{tikzcd}
    \end{equation*}
    where $\psi(h')=\lam{y}\ct{h'(f(y))}{\mathsf{segment\usc{}tot}(y)^{-1}}$. All the maps in this square are equivalences. In particular, the inverses of the top and bottom maps are $\varphi$ and $\varphi'$, respectively. The claim follows from this observation, but we will spell out the details.

    Since any equivalence is an embedding, we see immediately that the type $\mathsf{tr}_C(p)(\varphi(h))=\varphi'(h')$ is equivalent to the type
    \begin{equation*}
      \psi(\mathsf{tr}_C(p)(\varphi(h))\circ e')=\psi(\varphi'(h')\circ e').
    \end{equation*}
    By the commutativity of the square, the left hand side is $h$. The right hand side is $\psi(h')$. Therefore it follows that
    \begin{align*}
      \Big(\mathsf{tr}_C(p)(\varphi(h))=\varphi'(h')\Big)
      & \simeq \Big(h= \lam{y}\ct{h'(f(y))}{\mathsf{segment\usc{}tot}(y)^{-1}}\Big) \\
      & \simeq \Big(h'\circ f \htpy (\lam{y}\ct{h(y)}{\mathsf{segment\usc{}tot}(y)}\Big).\qedhere
    \end{align*}
  \end{proof}
  
  Applying these observations to the fundamental cover of the circle, we obtain the following lemma that we will use to prove that the total space of $\mathcal{E}$ is contractible.
  
  \begin{cor}\label{cor:construct-contraction-fundamental-cover}
    In order to show that the total space of $\mathcal{E}$ is contractible, it suffices to construct a function
    \begin{equation*}
      h : \prd{k:\Z}(\base,0_{\mathcal{E}})=(\base,k_{\mathcal{E}})
    \end{equation*}
    equipped with a homotopy
    \begin{equation*}
      H : \prd{k:\Z}h(\mathsf{succ}(k)_{\mathcal{E}})=\ct{h(k)}{\mathsf{segment\usc{}helix}(k)}.
    \end{equation*}
  \end{cor}

  In the next section we establish the dependent universal property of the integers, which we will use with \cref{cor:construct-contraction-fundamental-cover} to show that the total space of the fundamental cover is contractible.
  

\subsection{The dependent universal property of the integers}
\begin{lem}\label{lem:elim-Z}
Let $B$ be a family over $\Z$, equipped with a term $b_0:B(0)$, and an equivalence
\begin{equation*}
e_k : B(k)\eqvsym B(\mathsf{succ}(k))
\end{equation*}
for each $k:\Z$. Then there is a dependent function $f:\prd{k:\Z}B(k)$ equipped with identifications $f(0)=b_0$ and
\begin{equation*}
f(\mathsf{succ}(k))=e_k(f(k))
\end{equation*}
for any $k:\Z$.
\end{lem}

\begin{proof}
The map is defined using the induction principle for the integers, stated in \cref{lem:Z_ind}. First we take
\begin{align*}
f(-1) & \defeq e^{-1}(b_0) \\
f(0) & \defeq b_0 \\
f(1) & \defeq e(b_0).
\end{align*}
For the induction step on the negative integers we use
\begin{equation*}
\lam{n}e_{\mathsf{neg}(S(n))}^{-1} : \prd{n:\N} B(\mathsf{neg}(n))\to B(\mathsf{neg}(S(n)))
\end{equation*}
For the induction step on the positive integers we use
\begin{equation*}
\lam{n}e(\mathsf{pos}(n)) : \prd{n:\N} B(\mathsf{pos}(n))\to B(\mathsf{pos}(S(n))).
\end{equation*}
The computation rules follow in a straightforward way from the computation rules of $\Z$-induction and the fact that $e^{-1}$ is an inverse of $e$. 
\end{proof}

\begin{eg}
For any type $A$, we obtain a map $f:\Z\to A$ from any $x:A$ and any equivalence $e:\eqv{A}{A}$, such that $f(0)=x$ and the square
\begin{equation*}
\begin{tikzcd}
\Z \arrow[d,swap,"\mathsf{succ}"] \arrow[r,"f"] & A \arrow[d,"e"] \\
\Z \arrow[r,swap,"f"] & A
\end{tikzcd}
\end{equation*}
commutes. In particular, if we take $A\jdeq (x=x)$ for some $x:X$, then for any $p:x=x$ we have the equivalence $\lam{q}\ct{p}{q}:(x=x)\to (x=x)$. This equivalence induces a map
\begin{equation*}
k\mapsto p^k : \Z \to (x=x),
\end{equation*}
for any $p:x=x$. This induces the \define{degree $k$ map} on the circle
\begin{equation*}
\mathsf{deg}(k) : \sphere{1}\to\sphere{1},
\end{equation*}
for any $k:\mathbb{Z}$, see \cref{ex:degk}.
\end{eg}

In the following theorem we show that the dependent function constructed in \cref{lem:elim-Z} is unique.

\begin{thm}
  Consider a type family $B:\mathbb{Z}\to\UU$ equipped with $b:B(0)$ and a family of equivalences
  \begin{equation*}
    e:\prd{k:\Z} \eqv{B(k)}{B(\mathsf{succ}(k))}.
  \end{equation*}
  Then the type
  \begin{equation*}
    \sm{f:\prd{k:\Z}B(k)}(f(0)=b)\times\prd{k:\Z}f(\mathsf{succ}(k))=e_k(f(k))
  \end{equation*}
  is contractible.
\end{thm}

\begin{proof}
  In \cref{lem:elim-Z} we have already constructed a term of the asserted type.
  Therefore it suffices to show that any two terms of this type can be identified.
  Note that the type $(f,p,H)=(f',p',H')$ is equivalent to the type
  \begin{equation*}
    \sm{K:f\htpy f'} (K(0)= \ct{p}{(p')^{-1}})\times \prd{k:\Z}K(\mathsf{succ}(k))=\ct{(\ct{H(k)}{\ap{e_k}{K(k)}})}{H'(k)^{-1}}. 
  \end{equation*}
  We obtain a term of this type by applying \cref{lem:elim-Z} to the family $C$ over $\Z$ given by $C(k)\defeq f(k)=f'(k)$, which comes equipped with a base point
  \begin{equation*}
    \ct{p}{(p')^{-1}} : C(0),
  \end{equation*}
  and the family of equivalences
  \begin{equation*}
    \lam{\alpha:f(k)=f'(k)}\ct{(\ct{H(k)}{\ap{e_k}{\alpha}})}{H'(k)^{-1}}:\prd{k:\Z}\eqv{C(k)}{C(\mathsf{succ}(k))}.\qedhere
  \end{equation*}
\end{proof}

One way of phrasing the following corollary, is that $\Z$ is the `initial type equipped with a point and an automorphism'.

\begin{cor}
  For any type $X$ equipped with a base point $x_0:X$ and an automorphism $e:\eqv{X}{X}$, the type
  \begin{equation*}
    \sm{f:\Z\to X}(f(0)=x_0)\times ((f \circ \mathsf{succ})\htpy(e\circ f))
  \end{equation*}
  is contractible.
\end{cor}



\subsection{The identity type of the circle}

\begin{lem}\label{thm:circle_fundamental}
The total space $\sm{t:\sphere{1}}\mathcal{E}(t)$ of the fundamental cover of $\sphere{1}$ is contractible.\index{circle!fundamental cover!total space is contractible}
\end{lem}

\begin{proof}
  By \cref{cor:construct-contraction-fundamental-cover} it suffices to construct
  a function
  \begin{equation*}
    h : \prd{k:\Z}(\base,0_{\mathcal{E}})=(\base,k_{\mathcal{E}})
  \end{equation*}
  equipped with a homotopy
  \begin{equation*}
    H : \prd{k:\Z}h(\mathsf{succ}(k)_{\mathcal{E}})=\ct{h(k)}{\mathsf{segment\usc{}helix}(k)}.
  \end{equation*}
  We obtain $h$ and $H$ by the elimination principle of \cref{lem:elim-Z}. Indeed, the family $P$ over the integers given by $P(k)\defeq (\base,0_{\mathcal{E}})=(\base,k_{\mathcal{E}})$ comes equipped with a term $\refl{(\base,0_{\mathcal{E}})}:P(0)$, and a family of equivalences
  \begin{equation*}
    \prd{k:\Z}P(k) \simeq P(\mathsf{succ}(k))
  \end{equation*}
  given by $k,p\mapsto \ct{p}{\mathsf{segment\usc{}helix}(k)}$. 
\end{proof}

\begin{comment}
\begin{proof}
We show that the total space satisfies singleton induction (i.e., we apply \cref{thm:contractible}). Let $P$ be a family over the total space of the fundamental cover, and let $p_0:P(\base,0_{\mathcal{E}})$. Our goal is to construct a term of type
\begin{equation*}
\prd{t:\sphere{1}}{x:\mathcal{E}(t)} P(t,x).
\end{equation*}
We do this by induction. For the base case we must construct a term of type
\begin{equation*}
\prd{k:\Z}P(\base,k_{\mathcal{E}}).
\end{equation*}
Since we have the identifications $s_k: (\base,k_{\mathcal{E}})=(\base,\mathsf{succ}(k)_{\mathcal{E}})$, we have the equivalences
\begin{equation*}
\mathsf{tr}_P(s_k) : \eqv{P(\base,k_{\mathcal{E}})}{P(\base,\mathsf{succ}(k)_{\mathcal{E}})}
\end{equation*}
for each $k:\Z$. Thus we obtain a dependent function $f:\prd{x:\mathcal{E}(\base)}P(\base,x)$ satisfying $f(0_{\mathcal{E}})=p_0$ and $f(\mathsf{succ}(k)_{\mathcal{E}})=\mathsf{tr}_P(s_k,f(k_{\mathcal{E}}))$, for each $k:\Z$. 

For the loop case we must show that
\begin{equation*}
\mathsf{tr}_Q(\lloop,f)=f,
\end{equation*}
where $Q$ is the family over $\sphere{1}$ given by $Q(t)\defeq \prd{x:\mathcal{E}(t)} P(t,x)$. By function extensionality it suffices to construct a homotopy, and the transport along $\lloop$ in $Q$ computes as
\begin{equation*}
\mathsf{tr}_Q(\lloop,f)(k_{\mathcal{E}})= \mathsf{tr}_P(s_k,f(\mathsf{succ}^{-1}(k)_{\mathcal{E}})). 
\end{equation*}
Therefore the following computation completes the proof:
\begin{align*}
\mathsf{tr}_Q(\lloop,f)(k_{\mathcal{E}})
& = \mathsf{tr}_P(s_k,f(\mathsf{succ}^{-1}(k)_{\mathcal{E}})) \\
& = f(\mathsf{succ}(\mathsf{succ}^{-1}(k))_{\mathcal{E}}) \\
& = f(k_{\mathcal{E}}).\qedhere
\end{align*}
\end{proof}
\end{comment}

\begin{thm}\label{thm:eq-circle}
  The family of maps
  \begin{equation*}
    \prd{t:\sphere{1}} (\base=t)\to \mathcal{E}(t)
  \end{equation*}
  sending $\refl{\base}$ to $0_{\mathcal{E}}$ is a family of equivalences. In particular, the loop space of the circle is equivalent to $\Z$.
\end{thm}

\begin{proof}
  This is a direct corollary of \cref{thm:circle_fundamental,thm:id_fundamental}. 
\end{proof}

\begin{cor}
  The circle is a $1$-type and not a $0$-type.\index{circle!is a 1-type@{is a $1$-type}}
\end{cor}

\begin{proof}
  To see that the circle is a $1$-type we have to show that $s=t$ is a $0$-type for every $s,t:\sphere{1}$. By \cref{ex:circle-connected} it suffices to show that the loop space of the circle is a $0$-type. This is indeed the case, because $\Z$ is a $0$-type, and we have an equivalence $(\base=\base)\simeq \Z$.

  Furthermore, since $\Z$ is a $0$-type and not a $(-1)$-type, it follows that the circle is a $1$-type and not a $0$-type.
\end{proof}

\begin{exercises}
\exercise Show that the map
  \begin{equation*}
    \Z\to\loopspace{\sphere{1}}
  \end{equation*}
  is a group homomorphism. Conclude that the loop space $\loopspace{\sphere{1}}$ as a group is isomorphic to $\Z$.
  \exercise
  \begin{subexenum}
  \item Show that
    \begin{equation*}
      \prd{x:\sphere{1}}\neg\neg(\base=x).
    \end{equation*}
  \item On the other hand, use the fundamental cover of the circle to show that
    \begin{equation*}
      \neg\Big(\prd{x:\sphere{1}}\base=x\Big).
    \end{equation*}
  \item Conclude that
    \begin{equation*}
      \neg\Big(\prd{X:\UU} \neg\neg X\to X\Big)
    \end{equation*}
    for any univalent universe $\UU$ containing the circle.
  \end{subexenum}
  \exercise \label{ex:circle_degk}
\begin{subexenum}
\item Show that for every $x:X$, we have an equivalence
\begin{equation*}
\eqv{\Big(\sm{f:\sphere{1}\to X}f(\base)= x \Big)}{(x=x)}
\end{equation*}
\item Show that for every $t:\sphere{1}$, we have an equivalence
\begin{equation*}
\eqv{\Big(\sm{f:\sphere{1}\to \sphere{1}}f(\base)= t \Big)}{\Z}
\end{equation*}
The base point preserving map $f:\sphere{1}\to\sphere{1}$ corresponding to $k:\Z$ is called the \define{degree $k$ map} on the circle, and is denoted by $\mathsf{deg}(k)$.
\item Show that for every $t:\sphere{1}$, we have an equivalence
\begin{equation*}
\eqv{\Big(\sm{e:\eqv{\sphere{1}}{\sphere{1}}}e(\base)= t \Big)}{\bool}
\end{equation*}
\end{subexenum}
\exercise \label{ex:circle_double_cover} The \define{(twisted) double cover} of the circle is defined as the type family $\mathcal{T}\defeq\mathcal{D}(\bool,\mathsf{neg}):\sphere{1}\to\UU$, where $\mathsf{neg}:\eqv{\bool}{\bool}$ is the negation equivalence of \cref{ex:neg_equiv}.
\begin{subexenum}
\item Show that $\neg(\prd{t:\sphere{1}}\mathcal{T}(t))$.
\item Construct an equivalence $e:\eqv{\sphere{1}}{\sm{t:\sphere{1}}\mathcal{T}(t)}$ for which the triangle
\begin{equation*}
\begin{tikzcd}[column sep=tiny]
\sphere{1} \arrow[rr,"e"] \arrow[dr,swap,"\mathsf{deg}(2)"] & & \sm{t:\sphere{1}}\mathcal{T}(t) \arrow[dl,"\proj 1"] \\
\phantom{\sm{t:\sphere{1}}\mathcal{T}(t)} & \sphere{1}
\end{tikzcd}
\end{equation*}
commutes.
\end{subexenum}
\exercise Construct an equivalence $\eqv{(\eqv{\sphere{1}}{\sphere{1}})}{\sphere{1}+\sphere{1}}$ for which the triangle
\begin{equation*}
  \begin{tikzcd}
    (\eqv{\sphere{1}}{\sphere{1}}) \arrow[rr,"\simeq"] \arrow[dr,swap,"\evbase"] & & (\sphere{1}+\sphere{1}) \arrow[dl,"\fold"] \\
    & \sphere{1}
  \end{tikzcd}
\end{equation*}
commutes. Conclude that a univalent universe containing a circle is not a $1$-type.
\exercise \label{ex:is_invertible_id_S1}
\begin{subexenum}
\item Construct a family of equivalences
\begin{equation*}
\prd{t:\sphere{1}} \big(\eqv{(t=t)}{\Z}\big).
\end{equation*}
\item Use \cref{ex:circle_connected} to show that $\eqv{(\idfunc[\sphere{1}]\htpy\idfunc[\sphere{1}])}{\Z}$.
\item Use \cref{ex:idfunc_autohtpy} to show that
\begin{equation*}
\eqv{\mathsf{has\usc{}inverse}(\idfunc[\sphere{1}])}{\Z},
\end{equation*}
and conclude that ${\mathsf{has\usc{}inverse}}(\idfunc[\sphere{1}])\not\simeq{\isequiv(\idfunc[\sphere{1}])}$. 
\end{subexenum}
\exercise Consider a map $i:A \to \sphere{1}$, and assume that $i$ has a retraction. Construct a term of type
  \begin{equation*}
    \iscontr(A)+\isequiv(i).
  \end{equation*}
  \exercise
  \begin{subexenum}
  \item Show that the multiplicative operation on the circle is associative, i.e.~construct an identification
    \begin{equation*}
      \assoc_{\sphere{1}}(x,y,z) :
      \mulcircle(\mulcircle(x,y),z)=\mulcircle(x,\mulcircle(y,z))
    \end{equation*}
    for any $x,y,z:\sphere{1}$.
  \item Show that the associator satisfies unit laws, in the sense that the following triangles commute:
    \begin{equation*}
      \begin{tikzcd}[column sep=-1em]
        \mulcircle(\mulcircle(\base,x),y) \arrow[rr,equals] \arrow[dr,equals] & & \mulcircle(\base,\mulcircle(x,y)) \arrow[dl,equals] \\
        & \mulcircle(x,y)
      \end{tikzcd}
    \end{equation*}
    \begin{equation*}
      \begin{tikzcd}[column sep=-1em]
        \mulcircle(\mulcircle(x,\base),y) \arrow[rr,equals] \arrow[dr,equals] & & \mulcircle(x,\mulcircle(\base,y)) \arrow[dl,equals] \\
        & \mulcircle(x,y)
      \end{tikzcd}
    \end{equation*}
    \begin{equation*}
      \begin{tikzcd}[column sep=-1em]
        \mulcircle(\mulcircle(x,y),\base) \arrow[rr,equals] \arrow[dr,equals] & & \mulcircle(x,\mulcircle(y,\base)) \arrow[dl,equals] \\
        & \mulcircle(x,y).
      \end{tikzcd}
    \end{equation*}
  \item State the laws that compute
    \begin{align*}
      & \assoc_{\sphere{1}}(\base,\base,x) \\
      & \assoc_{\sphere{1}}(\base,x,\base) \\
      & \assoc_{\sphere{1}}(x,\base,\base) \\
      & \assoc_{\sphere{1}}(\base,\base,\base).
    \end{align*}
    Note: the first three laws should be $3$-cells and the last law should be a $4$-cell. The laws are automatically satisfied, since the circle is a $1$-type.
  \end{subexenum}
  \exercise Construct the \define{Mac Lane pentagon} for the circle, i.e.~show that the pentagon
  \begin{equation*}
    \begin{tikzcd}[column sep=-6em]
      &[-6em] \mulcircle(\mulcircle(\mulcircle(x,y),z),w) \arrow[rr,equals] \arrow[dl,equals] & & \mulcircle(\mulcircle(x,y),\mulcircle(z,w)) \arrow[dr,equals] &[-6em] \\
      \mulcircle(\mulcircle(x,\mulcircle(y,z)),w) \arrow[drr,equals] & & & & \mulcircle(x,\mulcircle(y,\mulcircle(z,w))) \\
      & & \mulcircle(x,\mulcircle(\mulcircle(y,z),w)) \arrow[urr,equals]
    \end{tikzcd}
  \end{equation*}
  commutes for every $x,y,z,w:\sphere{1}$.
  \exercise Recall from \cref{ex:surjective-precomp} that if $f:A\to B$ is a surjective map, then the precomposition map
  \begin{equation*}
    \blank\circ f : (B\to C)\to (A\to C)
  \end{equation*}
  is an embedding for every set $C$. 
  Give an example of a surjective map $f:A\to B$, such that the precomposition function
    \begin{equation*}
      \blank\circ f:(B\to \sphere{1})\to (A\to \sphere{1})
    \end{equation*}
    is \emph{not} an embedding, showing that the condition that $C$ is a set is essential.
\end{exercises}

\index{circle!fundamental cover|)}
\index{fundamental cover!of the circle|)}
\index{circle|)}
\index{inductive type!circle|)}

\input{classifying}

\chapter{Concepts of higher category theory in type theory}
% !TEX root = hott_intro.tex

\section{Homotopy pullbacks}

Suppose we are given a map $f:A\to B$, and type families $P$ over $A$, and $Q$ over $B$.
Then any family of maps
\begin{equation*}
g:\prd{x:A}P(x)\to Q(f(x))
\end{equation*}
gives rise to a commuting square
\begin{equation*}
\begin{tikzcd}[column sep=large]
\sm{x:A}P(x) \arrow[r,"{\tot[f]{g}}"] \arrow[d,swap,"\proj 1"] & \sm{y:B}Q(y) \arrow[d,"\proj 1"] \\
A \arrow[r,swap,"f"] & B
\end{tikzcd}
\end{equation*}
where $\tot[f]{g}$ is defined as $\lam{(x,y)}(f(x),g(x,y))$. In the main theorem of this chapter we show that $g$ is a family of equivalences if and only if this square satisfies a certain universal property: the universal property of \emph{pullback squares}.

Pullback squares are of interest because they appear in many situations. Cartesian products, fibers of maps, and substitutions can all be presented as pullbacks. Moreover, the fact that a family of maps $g:\prd{x:A}P(x)\to Q(f(x))$ is a family of equivalences if and only if it induces a pullback square has the very useful corollary that a square of the form
\begin{equation*}
  \begin{tikzcd}
    C \arrow[d,swap,"p"] \arrow[r] & D \arrow[d,"q"] \\
    A \arrow[r,swap,"f"] & B
  \end{tikzcd}
\end{equation*}
is a pullback square if and only if the induced family of maps between the fibers
\begin{equation*}
  \prd{x:A}\fib{p}{x}\to\fib{q}{f(x)}
\end{equation*}
is a family of equivalences. This connection between pullbacks and \emph{fiberwise equivalences} has an important role in the descent theorem\index{descent} in \cref{chap:descent}.

A second reason for studying pullback squares is that the dual notion of \emph{pushouts} is an important tool to construct new types, including the $n$-spheres for arbitrary $n$. The duality of pullbacks and pushouts makes it possible to obtain proofs of many statements about pushouts from their dual statements about pullbacks.

\subsection{The universal property of pullbacks}

\begin{defn}\label{defn:cospan}
  A \define{cospan}\index{cospan} consists of three types $A$, $X$, and $B$, and maps $f:A\to X$ and $g:B\to X$.
\end{defn}

\begin{defn}
  Consider a cospan
  \begin{equation*}
    \begin{tikzcd}
      A \arrow[r,"f"] & X & B \arrow[l,swap,"g"] 
    \end{tikzcd}
  \end{equation*}
  and a type $C$. A \define{cone}\index{cone!on a cospan} on the cospan $A \rightarrow X \leftarrow B$ with \define{vertex} $C$\index{vertex!of a cone} consists of maps $p:C\to A$, $q:C\to B$ and a homotopy $H:f\circ p\htpy g\circ q$ witnessing that the square
  \begin{equation*}
    \begin{tikzcd}
      C \arrow[r,"q"] \arrow[d,swap,"p"] & B \arrow[d,"g"] \\
      A \arrow[r,swap,"f"] & X
    \end{tikzcd}
  \end{equation*}
  commutes. We write\index{cone(C)@{$\mathsf{cone}(\blank)$}}
\begin{equation*}
\mathsf{cone}(C)\defeq \sm{p:C\to A}{q:C\to B}f\circ p\htpy g\circ q
\end{equation*}
for the type of cones with vertex $C$.
\end{defn}

It is good practice to characterize the identity type of any type of importance. In the following lemma we give a characterization of the identity type of the type $\mathsf{cone}(C)$ of cones on $A\rightarrow X\leftarrow B$ with vertex $C$. Such characterizations are entirely routine in homotopy type theory.

\begin{lem}\label{lem:id_cone}%
\index{identity type!of cone@{of $\mathsf{cone}(C)$}}%
Let $(p,q,H)$ and $(p',q',H')$ be cones on a cospan $f:A\rightarrow X \leftarrow B:g$, both with vertex $C$. Then the type $(p,q,H)=(p',q',H')$ is equivalent to the type of triples $(K,L,M)$ consisting of
\begin{align*}
K & : p \htpy p' \\
L & : q \htpy q'
\end{align*}
and a homotopy $M : \ct{H}{(g\cdot L)} \htpy \ct{(f\cdot K)}{H'}$ witnessing that the square
\begin{equation*}
\begin{tikzcd}
f\circ p \arrow[r,"f\cdot K"] \arrow[d,swap,"H"] & f\circ p' \arrow[d,"{H'}"] \\
g\circ q \arrow[r,swap,"g\cdot L"] & g\circ q'
\end{tikzcd}
\end{equation*}
of homotopies commutes.
\end{lem}

\begin{comment}
\begin{rmk}
The homotopy $M$ is a homotopy of homotopies, and for each $z:C$ the identification $M(z)$ witnesses that the square of identifications
\begin{equation*}
\begin{tikzcd}[column sep=huge]
f(p(z)) \arrow[r,equals,"\ap{f}{K(z)}"] \arrow[d,equals,swap,"H(z)"] & f(p'(z)) \arrow[d,equals,"{H'(z)}"] \\
g(q(z)) \arrow[r,equals,swap,"\ap{g}{L(z)}"] & g(q'(z))
\end{tikzcd}
\end{equation*}
commutes. 
\end{rmk}
\end{comment}

\begin{proof}
By the fundamental theorem of identity types (\cref{thm:id_fundamental}) it suffices to show that the type
\begin{equation*}
  \sm{(p',q',H'):\sm{p':C\to A}{q':C\to B}f\circ p'\htpy g \circ q'}{K:p\htpy p'}{L:q\htpy q'} \ct{H}{(g\cdot L)} \htpy \ct{(f\cdot K)}{H'}
\end{equation*}
is contractible. Using associativity of $\Sigma$-types and commutativity of cartesian products, it is easy to show that this type is equivalent to the type
\begin{equation*}
  \sm{(p',K):\sm{p':C\to A}p\htpy p'}\sm{(q',L):\sm{q':C\to B}q\htpy q'}\sm{H:f\circ p'\htpy g \circ q'}\ct{H}{(g\cdot L)} \htpy \ct{(f\cdot K)}{H'}
\end{equation*}
Now we observe that the types $\sm{p':C\to A}p\htpy p'$ and $\sm{q':C\to B}q\htpy q'$ are contractible, with centers of contraction
\begin{samepage}
\begin{align*}
(p,\reflhtpy_p) & : \sm{p':C'\to A} p\htpy p' \\
(q,\reflhtpy_q) & : \sm{q':C'\to B} q\htpy q'.
\end{align*}%
\end{samepage}%
Thus we apply \cref{ex:contr_in_sigma} to see that the type of tuples $((p',K),(q',L),(H',M))$ is equivalent to the type
\begin{equation*}
\sm{H':f\circ p'\htpy g\circ q'} \ct{H}{\reflhtpy_{g\circ q}}\htpy \ct{\reflhtpy_{f\circ p}}{H'}.
\end{equation*}
Of course, the type $\ct{H}{\reflhtpy_{g\circ q}}\htpy \ct{\reflhtpy_{f\circ p}}{H'}$ is equivalent to the type $H\htpy H'$, and $\sm{H':f\circ p\htpy g\circ q} H\htpy H'$ is contractible.
\end{proof}

Given a cone with vertex $C$ on a span $A\stackrel{f}{\rightarrow} X \stackrel{g}{\leftarrow} B$ and a map $h:C'\to C$, we construct a new cone with vertex $C'$ in the following definition.

\begin{defn}
For any cone $(p,q,H)$ with vertex $C$ and any type $C'$, we define a map\index{cone map@{$\mathsf{cone\usc{}map}$}}
\begin{equation*}
\mathsf{cone\usc{}map}(p,q,H):(C'\to C)\to\mathsf{cone}(C')
\end{equation*}
by $h\mapsto (p\circ h,q\circ h,H\circ h)$. 
\end{defn}

\begin{defn}
We say that a commuting square
\begin{equation*}
\begin{tikzcd}
C \arrow[r,"q"] \arrow[d,swap,"p"] & B \arrow[d,"g"] \\
A \arrow[r,swap,"f"] & X
\end{tikzcd}
\end{equation*}
with $H:f\circ p\htpy g\circ q$ is a \define{pullback square}\index{pullback square}, or that it is \define{cartesian}\index{cartesian square}, if it satisfies the \define{universal property} of pullbacks\index{universal property!of pullbacks}, which asserts that the map
\begin{equation*}
\mathsf{cone\usc{}map}(p,q,H):(C'\to C)\to\mathsf{cone}(C')
\end{equation*}
is an equivalence for every type $C'$. 
\end{defn}

We often indicate the universal property with a diagram as follows:
\begin{equation*}
\begin{tikzcd}
C' \arrow[drr,bend left=15,"{q'}"] \arrow[dr,densely dotted,"h"] \arrow[ddr,bend right=15,swap,"{p'}"] \\
& C \arrow[r,"q"] \arrow[d,swap,"p"] & B \arrow[d,"g"] \\
& A \arrow[r,swap,"f"] & X
\end{tikzcd}
\end{equation*}
since the universal property states that for every cone $(p',q',H')$ with vertex $C'$, the type of pairs $(h,\alpha)$ consisting of $h:C'\to C$ equipped with $\alpha:\mathsf{cone\usc{}map}((p,q,H),h)=(p',q',H')$ is contractible by \cref{thm:contr_equiv}.

As a corollary we obtain the following characterization of the universal property of pullbacks.

\begin{lem}\label{thm:pullback_up}
Consider a commuting square
\begin{equation*}
\begin{tikzcd}
C \arrow[r,"q"] \arrow[d,swap,"p"] & B \arrow[d,"g"] \\
A \arrow[r,swap,"f"] & X
\end{tikzcd}
\end{equation*}
with $H:f\circ p\htpy g\circ q$
Then the following are equivalent:\index{universal property!of pullbacks (characterization)}
\begin{enumerate}
\item The square is a pullback square.
\item For every type $C'$ and every cone $(p',q',H')$ with vertex $C'$, the type of quadruples $(h,K,L,M)$ consisting of a map $h:C'\to C$, homotopies
\begin{align*}
K & : p\circ h \htpy p' \\
L & : q\circ h \htpy q',
\end{align*}
and a homotopy $M : \ct{(H\cdot h)}{(g\cdot L)} \htpy \ct{(f\cdot K)}{H'}$ witnessing that the square
\begin{equation*}
\begin{tikzcd}
f\circ p\circ h \arrow[r,"f\cdot K"] \arrow[d,swap,"H\cdot h"] & f\circ p' \arrow[d,"{H'}"] \\
g\circ q\circ h \arrow[r,swap,"g\cdot L"] & g\circ q'
\end{tikzcd}
\end{equation*}
commutes, is contractible.
\end{enumerate}
\end{lem}

\begin{proof}
The map $\mathsf{cone\usc{}map}(p,q,H)$ is an equivalence if and only if its fibers are contractible. By \cref{lem:id_cone} it follows that the fibers of $\mathsf{cone\usc{}map}(p,q,H)$ are equivalent the the described type of quadruples ($h,K,L,M)$.
\end{proof}

In the following lemma we establish the uniqueness of pullbacks up to equivalence via a \emph{3-for-2 property} for pullbacks.

\begin{lem}\label{lem:pb_3for2}\index{pullback!3-for-2 property}\index{3-for-2 property!of pullbacks}%
Consider the squares
\begin{equation*}
\begin{tikzcd}
C \arrow[r,"q"] \arrow[d,swap,"p"] & B \arrow[d,"g"] & {C'} \arrow[r,"{q'}"] \arrow[d,swap,"{p'}"] & B \arrow[d,"g"] \\
A \arrow[r,swap,"f"] & X & A \arrow[r,swap,"f"] & X
\end{tikzcd}
\end{equation*}
with homotopies $H:f\circ p \htpy g\circ q$ and $H':f\circ p'\htpy g\circ q'$.
Furthermore, suppose we have a map $h:C'\to C$ equipped with
\begin{align*}
K & : p\circ h \htpy p' \\
L & : q\circ h \htpy q' \\
M & : \ct{(H\cdot h)}{(g\cdot L)} \htpy \ct{(f\cdot K)}{H'}.
\end{align*}
If any two of the following three properties hold, so does the third:
\begin{samepage}%
\begin{enumerate}
\item $C$ is a pullback.
\item $C'$ is a pullback.
\item $h$ is an equivalence.
\end{enumerate}%
\end{samepage}%
\end{lem}

\begin{proof}
By the characterization of the identity type of $\mathsf{cone}(C')$ given in \cref{lem:id_cone} we obtain an identification
\begin{equation*}
\mathsf{cone\usc{}map}((p,q,H),h)=(p',q',H')
\end{equation*}
from the triple $(K,L,M)$. 
Let $D$ be a type, and let $k:D\to C'$ be a map. We observe that
\begin{align*}
\mathsf{cone\usc{}map}((p,q,H),(h\circ k)) & \jdeq (p\circ (h\circ k),q\circ (h\circ k),H\circ (h\circ k)) \\
& \jdeq ((p\circ h)\circ k,(q\circ h)\circ k, (H\circ h)\circ k) \\
& \jdeq \mathsf{cone\usc{}map}(\mathsf{cone\usc{}map}((p,q,H),h),k) \\
& = \mathsf{cone\usc{}map}((p',q',H'),k).
\end{align*}
Thus we see that the triangle 
\begin{equation*}
\begin{tikzcd}[column sep=-1em]
(D\to C') \arrow[rr,"{h\circ \blank}"] \arrow[dr,swap,"{\mathsf{cone\usc{}map}(p',q',H')}"] & & (D\to C) \arrow[dl,"{\mathsf{cone\usc{}map}(p,q,H)}"] \\
& \mathsf{cone}(D)
\end{tikzcd}
\end{equation*}
commutes. Therefore it follows from the 3-for-2 property of equivalences established in \cref{ex:3_for_2}, that if any two of the maps in this triangle is an equivalence, then so is the third. Now the claim follows from the fact that $h$ is an equivalence if and only if $h\circ\blank : (D\to C')\to (D\to C)$ is an equivalence for any type $D$, which was established in \cref{lem:postcomp_equiv}.
\end{proof}

Pullbacks are not only unique in the sense that any two pullbacks of the same cospan are equivalent, they are \emph{uniquely unique}\index{uniquely uniqueness!of pullbacks} in the sense that the type of quadruples $(h,K,L,M)$ as in \cref{lem:pb_3for2} is contractible.

\begin{cor}\label{cor:uniquely-unique-pullback}
Suppose both commuting squares
\begin{equation*}
\begin{tikzcd}
C \arrow[r,"q"] \arrow[d,swap,"p"] & B \arrow[d,"g"] & {C'} \arrow[r,"{q'}"] \arrow[d,swap,"{p'}"] & B \arrow[d,"g"] \\
A \arrow[r,swap,"f"] & X & A \arrow[r,swap,"f"] & X
\end{tikzcd}
\end{equation*}
with homotopies $H:f\circ p \htpy g\circ q$ and $H':f\circ p'\htpy g\circ q'$ are pullback squares.
Then the type of quadruples $(e,K,L,M)$ consisting of an equivalence $e:\eqv{C'}{C}$ equipped with
\begin{align*}
K & : p\circ e \htpy p' \\
L & : q\circ e \htpy q' \\
M & : \ct{(H\cdot h)}{(g\cdot L)} \htpy \ct{(f\cdot K)}{H'}.
\end{align*}
is contractible.
\end{cor}

\begin{proof}
We have seen that the type of quadruples $(h,K,L,M)$ is equivalent to the fiber of $\mathsf{cone\usc{}map}(p,q,H)$ at $(p',q',H')$. By \cref{lem:pb_3for2} it follows that $h$ is an equivalence. Since $\isequiv(h)$ is a proposition by \cref{ex:isprop_isequiv}, and hence contractible as soon as it is inhabited, it follows that the type of quadruples $(e,K,L,M)$ is contractible. 
\end{proof}

\begin{cor}
  For any two maps $f:A\to X$ and $g:B\to X$, and any universe $\UU$, the type
  \begin{equation*}
    \sm{C:\UU}{c:\mathsf{cone}(f,g,C)}\prd{C':\UU}\mathsf{is\usc{}equiv}(\mathsf{cone\usc{}map}_{C'}(c))
  \end{equation*}
  of pullbacks in $\UU$, is a proposition.
\end{cor}

\begin{proof}
  It is straightforward to see that the type of identifications
  \begin{equation*}
    (C,(p,q,H),u)=(C',(p',q',H'),u')
  \end{equation*}
  of any two pullbacks is equivalent to the type of quadruples $(e,K,L,M)$ as in \cref{cor:uniquely-unique-pullback}. Since \cref{cor:uniquely-unique-pullback} claims that this type of quadruples is contractible, the claim follows.
\end{proof}

\subsection{Canonical pullbacks}

For every cospan we can construct a \emph{canonical pullback}.

\begin{defn}
Let $f:A\to X$ and $g:B\to X$ be maps. Then we define
\begin{align*}
A\times_X B & \defeq \sm{x:A}{y:B}f(x)=g(y) \\
\pi_1 & \defeq \proj 1 & & : A\times_X B\to A \\
\pi_2 & \defeq \proj 1\circ\proj 2 & & : A\times_X B\to B\\
\pi_3 & \defeq \proj 2\circ\proj 2 & & : f\circ \pi_1 \htpy g\circ\pi_2.
\end{align*}
The type $A\times_X B$ is called the \define{canonical pullback}\index{canonical pullback} of $f$ and $g$.
\end{defn}

Note that $A\times_X B$ depends on $f$ and $g$, although this dependency is not visible in the notation.

\begin{rmk}
  Given $(x,y,p)$ and $(x',y',p')$ in the canonical pullback $A\times_X B$, the identity type $(x,y,p)=(x',y',p')$ is equivalent to the type of triples $(\alpha,\beta,\gamma)$ consisting of $\alpha:x=x'$, $\beta:y=y'$, and an identification $\gamma:\ct{p}{\ap{g}{\beta}}=\ct{\ap{f}{\alpha}}{p'}$ witnessing that the square
  \begin{equation*}
    \begin{tikzcd}[column sep=large]
      f(x) \arrow[r,equal,"\ap{f}{\alpha}"] \arrow[d,swap,equal,"p"] & f(x') \arrow[d,equal,"{p'}"] \\
      g(y) \arrow[r,swap,equal,"\ap{g}{\beta}"] & g(y')
    \end{tikzcd}
  \end{equation*}
  commutes. The proof of this fact is similar to the proof of \cref{lem:id_cone}.
\end{rmk}

\begin{thm}\label{thm:canonical-pullback}
Given maps $f:A\to X$ and $g:B\to X$, the commuting square\index{canonical pullback}
\begin{equation*}
\begin{tikzcd}
A\times_X B \arrow[r,"\pi_2"] \arrow[d,swap,"\pi_1"] & B \arrow[d,"g"] \\
A \arrow[r,swap,"f"] & X,
\end{tikzcd}
\end{equation*}
is a pullback square.
\end{thm}

\begin{proof}
Let $C$ be a type. Our goal is to show that the map
\begin{equation*}
\mathsf{cone\usc{}map}(\pi_1,\pi_2,\pi_3): (C\to A\times_X B)\to \mathsf{cone}(C)
\end{equation*}
is an equivalence. Note that we have the commuting triangle
\begin{equation*}
  \begin{tikzcd}[column sep=-4em]
    C\to\sm{x:A}{y:B}f(x)=g(y) \arrow[dd,swap,"\mathsf{cone\usc{}map}"] \arrow[dr,"\mathsf{choice}"] \\
    & \sm{p:C\to A}\prd{z:C}\sm{y:B} f(p(z))= g(y) \arrow[dl,"\mathsf{choice}"] \\
    \sm{p:C\to A}{q:C\to B} f\circ p \htpy g\circ q.
  \end{tikzcd}
\end{equation*}
In this triangle the functions $\mathsf{choice}$ are equivalences by \cref{thm:choice}. Therefore, their composite is an equivalence.
\end{proof}

The following corollary is now a special case of \cref{cor:is-prop-UU-pullback}, where we make sure that $f:A\to X$ and $g:B\to X$ are both maps in $\UU$.

\begin{cor}
  For any two maps $f:A\to X$ and $g:B\to X$ in $\UU$, the type
  \begin{equation*}
    \sm{C:\UU}{c:\mathsf{cone}(f,g,C)}\prd{C':\UU}\mathsf{is\usc{}equiv}(\mathsf{cone\usc{}map}_{C'}(c))
  \end{equation*}
  of pullbacks in $\UU$, is contractible.
\end{cor}

\begin{defn}
Given a commuting square
\begin{equation*}
\begin{tikzcd}
C \arrow[r,"q"] \arrow[d,swap,"p"] & B \arrow[d,"g"] \\
A \arrow[r,swap,"f"] & X
\end{tikzcd}
\end{equation*}
with $H:f\circ p \htpy g \circ q$, we define the \define{gap map}\index{gap map}\index{pullback!gap map}
\begin{equation*}
\mathsf{gap}(p,q,H):C \to A\times_X B
\end{equation*}
by $\lam{z}(p(z),q(z),H(z))$.
\end{defn}

The following theorem provides a useful characterization of pullback squares, because in many situations it is easier to show that the gap map is an equivalence.

\begin{thm}\label{thm:is_pullback}
Consider a commuting square
\begin{equation*}
\begin{tikzcd}
C \arrow[r,"q"] \arrow[d,swap,"p"] & B \arrow[d,"g"] \\
A \arrow[r,swap,"f"] & X
\end{tikzcd}
\end{equation*}
with $H:f\circ p \htpy g \circ q$. The following are equivalent:
\begin{enumerate}
\item The square is a pullback square
\item There is a term of type
\begin{equation*}
\mathsf{is\usc{}pullback}(p,q,H)\defeq \isequiv(\mathsf{gap}(p,q,H)).
\end{equation*}
\end{enumerate}
\end{thm}

\begin{proof}
  Observe that we are in the situation of \cref{lem:pb_3for2}. Indeed, we have two commuting squares
  \begin{equation*}
    \begin{tikzcd}
      A\times_X B \arrow[r,"\pi_2"] \arrow[d,swap,"\pi_2"] & B \arrow[d,"g"] &[2em] C \arrow[r,"q"] \arrow[d,swap,"p"] & B \arrow[d,"g"] \\
      A \arrow[r,swap,"f"] & X & A \arrow[r,swap,"f"] & X,
    \end{tikzcd}
  \end{equation*}
  and we have the gap map $\mathsf{gap}:C\to A\times_X B$, which comes equipped with the homotopies
\begin{align*}
K & : \pi_1\circ \mathsf{gap} \htpy p & K & \defeq \lam{z}\refl{p(z)} \\
L & : \pi_2\circ \mathsf{gap} \htpy q & L & \defeq \lam{z}\refl{q(z)} \\
M & : \ct{(\pi_3\cdot \mathsf{gap})}{(g\cdot L)} \htpy \ct{(f\cdot K)}{H} & M & \defeq \lam{z}\mathsf{right\usc{}unit}(H(z)).
\end{align*}
Since $A\times_X B$ is shown to be a pullback in \cref{thm:canonical-pullback}, it follows from \cref{lem:pb_3for2} that $C$ is a pullback if and only if the gap map is an equivalence.
\end{proof}

\subsection{Cartesian products and fiberwise products as pullbacks}

An important special case of pullbacks occurs when the cospan is of the form
\begin{equation*}
\begin{tikzcd}
A \arrow[r] & \unit & B. \arrow[l]
\end{tikzcd}
\end{equation*}
In this case, the pullback is just the \emph{cartesian product}.

\begin{lem}\label{lem:prod_pb}
Let $A$ and $B$ be types. Then the square
\begin{equation*}
\begin{tikzcd}
A\times B \arrow[r,"\proj 2"] \arrow[d,swap,"\proj 1"] & B \arrow[d,"\mathsf{const}_{\ttt}"] \\
A \arrow[r,swap,"\mathsf{const}_{\ttt}"] & \unit
\end{tikzcd}
\end{equation*}
which commutes by the homotopy $\mathsf{const}_{\refl{\ttt}}$ is a pullback square.\index{cartesian product!as pullback}
\end{lem}

\begin{proof}
By \cref{thm:is_pullback} it suffices to show that
\begin{equation*}
\mathsf{gap}(\proj 1,\proj2,\lam{(a,b)}\refl{\ttt})
\end{equation*}
is an equivalence. Its inverse is the map $\lam{(a,b,p)}(a,b)$.
\end{proof}

The following generalization of \cref{lem:prod_pb} is the reason why pullbacks are sometimes called \define{fiber products}\index{fiber product}.

\begin{thm}
Let $P$ and $Q$ be families over a type $X$. Then the square
\begin{equation*}
\begin{tikzcd}[column sep=8em]
\sm{x:X}P(x)\times Q(x) \arrow[r,"{\lam{(x,(p,q))}(x,q)}"] \arrow[d,swap,"{\lam{(x,(p,q))}(x,p)}"] & \sm{x:X}Q(x) \arrow[d,"\proj 1"] \\
\sm{x:X}P(x) \arrow[r,swap,"\proj 1"] & X,
\end{tikzcd}
\end{equation*}
which commutes by the homotopy
\begin{equation*}
H\defeq \lam{(x,(p,q))}\refl{x},
\end{equation*}
is a pullback square.
\end{thm}

\begin{proof}
By \cref{thm:is_pullback} it suffices to show that the gap map is an equivalence. The gap map is homotopic to the function
\begin{equation*}
\lam{(x,(p,q))}((x,p),(x,q),\refl{x}).
\end{equation*}
It is easy to check that this function is an equivalence. I
ts inverse is the map 
\begin{equation*}
\lam{((x,p),(y,q),\alpha)}(y,(\mathsf{tr}_P(\alpha,p),q)).\qedhere
\end{equation*}
\end{proof}

\begin{cor}
For any $f:A\to X$ and $g:B\to X$, the square
\begin{equation*}
\begin{tikzcd}[column sep=8em]
\sm{x:X}\fib{f}{x}\times\fib{g}{x} \arrow[r,"{\lam{(x,((a,p),(b,q)))}b}"] \arrow[d,swap,"{\lam{(x,((a,p),(b,q)))}a}"] & B \arrow[d,"g"]  \\
A \arrow[r,swap,"f"] & X
\end{tikzcd}
\end{equation*}
is a pullback square.
\end{cor}

\subsection{Fibers of maps as pullbacks}

\begin{lem}\label{lem:fib_pb}
For any function $f:A\to B$, and any $b:B$, consider the square
\begin{equation*}
\begin{tikzcd}[column sep=large]
\fib{f}{b} \arrow[r,"\mathsf{const}_\ttt"] \arrow[d,swap,"\proj 1"] & \unit \arrow[d,"\mathsf{const}_b"] \\
A \arrow[r,swap,"f"] & B
\end{tikzcd}
\end{equation*}
which commutes by $\proj 2 : \prd{t:\fib{f}{b}} f(\proj 1(t))=b$. This is a pullback square.\index{fiber!as pullback}
\end{lem}

\begin{proof}
By \cref{thm:is_pullback} it suffices to show that the gap map is an equivalence. The gap map is homotopic to the function
\begin{equation*}
\tot(\lam{x}{p}(\ttt,p))
\end{equation*}
The map $\lam{x}{p}(\ttt,p)$ is a family of equivalences by \cref{ex:contr_in_sigma}, so it induces an equivalence on total spaces by \cref{thm:fib_equiv}.
\end{proof}

\begin{cor}
For any type family $B$ over $A$ and any $a:A$ the square
\begin{equation*}
\begin{tikzcd}[column sep=large]
B(a) \arrow[d,swap,"{\lam{y}(a,y)}"] \arrow[r,"\mathsf{const}_\ttt"] & \unit \arrow[d,"\lam{\ttt}a"] \\
\sm{x:A}B(x) \arrow[r,swap,"\proj 1"] & A
\end{tikzcd}
\end{equation*}
is a pullback square.
\end{cor}

\begin{proof}
  To see this, note that the triangle
  \begin{equation*}
    \begin{tikzcd}[column sep=0]
      B(a) \arrow[rr,"{\lam{b}((a,b),\refl{a})}"] \arrow[dr,swap,"\mathsf{gap}"] & & \fib{\proj 1}{a} \arrow[dl,"\mathsf{gap}"] \\
      & \Big(\sm{x:A}B(x)\Big)\times_A\unit.
    \end{tikzcd}
  \end{equation*}
  Since the top map is an equivalence by \cref{ex:fib_replacement}, and the map on the right is an equivalence by \cref{lem:fib_pb}, it follows that the map on the left is an equivalence. The claim follows.
\end{proof}

\subsection{Families of equivalences}

\begin{lem}\label{lem:pb_subst}
Let $f:A\to B$, and let $Q$ be a type family over $B$. Then the square
\begin{equation*}
\begin{tikzcd}[column sep=6em]
\sm{x:A}Q(f(x)) \arrow[r,"{\lam{(x,q)}(f(x),q)}"] \arrow[d,swap,"\proj 1"] & \sm{y:B}Q(b) \arrow[d,"\proj 1"] \\
A \arrow[r,swap,"f"] & B
\end{tikzcd}
\end{equation*}
commutes by $H\defeq \lam{(x,q)}\refl{f(x)}$. This is a pullback square.\index{substitution!as pullback}
\end{lem}

\begin{proof}
By \cref{thm:is_pullback} it suffices to show that the gap map is an equivalence. The gap map is homotopic to the function
\begin{equation*}
\lam{(x,q)}(x,(f(x),q),\refl{f(x)}).
\end{equation*}
The inverse of this map is given by $\lam{(x,((y,q),p))}(x,\mathsf{tr}_Q(p^{-1},q))$, and it is straightforward to see that these maps are indeed mutual inverses.
\end{proof}

\begin{thm}\label{thm:pb_fibequiv}
Let $f:A\to B$, and let $g:\prd{a:A}P(a)\to Q(f(a))$ be a family of maps\index{family of maps}. The following are equivalent:
\begin{enumerate}
\item The commuting square
\begin{equation*}
\begin{tikzcd}[column sep=large]
\sm{a:A}P(a) \arrow[r,"{\tot[f]{g}}"] \arrow[d,->>] & \sm{b:B}Q(b) \arrow[d,->>] \\
A \arrow[r,swap,"f"] & B
\end{tikzcd}
\end{equation*}
is a pullback square.
\item $g$ is a family of equivalences.\index{family of equivalences}
\end{enumerate}
\end{thm}

\begin{proof}
The gap map is homotopic to the composite
\begin{equation*}
\begin{tikzcd}[column sep=large]
\sm{x:A}P(x) \arrow[r,"\tot{g}"] & \sm{x:A}Q(f(x)) \arrow[r,"{\mathsf{gap}'}"] & A \times_B \Big(\sm{y:B}Q(y)\Big)
\end{tikzcd}
\end{equation*}
where $\mathsf{gap}'$ is the gap map for the square in \cref{lem:pb_subst}. Since $\mathsf{gap}'$ is an equivalence, it follows by \cref{ex:3_for_2,thm:fib_equiv} that the gap map is an equivalence if and only if $g$ is a family of equivalences.
\end{proof}

Our goal is now to extend \cref{thm:pb_fibequiv} to
arbitrary pullback squares. Note that every commuting
square
\begin{equation*}
\begin{tikzcd}
A \arrow[d,swap,"f"] \arrow[r,"h"] & B \arrow[d,"g"] \\
X \arrow[r,swap,"i"] & Y
\end{tikzcd}
\end{equation*}
with $H: i\circ f ~ g \circ h$ induces a map
\begin{equation*}
\fibsquare : \prd{x:X} \fib{f}{x} \to \fib{g}{f(x)}
\end{equation*}
on the fibers, by
\begin{equation*}
\fibsquare(x,(a,p))\defeq (h(a),\ct{H(a)^{-1}}{\ap{i}{p}}).
\end{equation*}

\begin{thm}\label{cor:pb_fibequiv}
Consider a commuting square
\begin{equation*}
\begin{tikzcd}
A \arrow[d,swap,"f"] \arrow[r,"h"] & B \arrow[d,"g"] \\
X \arrow[r,swap,"i"] & Y
\end{tikzcd}
\end{equation*}
with $H: i\circ f ~ g \circ h$. The following are equivalent:
\begin{enumerate}
\item The square is a pullback square.\index{pullback square!characterized by families of equivalences}
\item The induced map on fibers
\begin{equation*}
\fibsquare : \prd{x:X} \fib{f}{x} \to \fib{g}{f(x)}
\end{equation*}
is a family of equivalences.
\end{enumerate}
\end{thm}

\begin{proof}
First we observe that the square
\begin{equation*}
\begin{tikzcd}[column sep=huge]
\sm{x:X}\fib{f}{x} \arrow[d,swap,"\eqvsym"] \arrow[r,"\tot{\fibsquare}"] &
\sm{x:X}\fib{g}{f(x)} \arrow[d,"\tot{\tot{\mathsf{inv}}}"] \\
A \arrow[r,swap,"\mathsf{gap}"] & X \times_Y B
\end{tikzcd}
\end{equation*}
commutes. To construct such a homotopy, we need to construct an identification
\begin{equation*}
(f(a),h(a),H(a))=(x,h(a),(\ct{H(a)^{-1}}{\ap{i}{p}})^{-1})
\end{equation*}
for every $x : X$, $a : A$, and $p : f(a) = x$. This is shown by path induction on $p : f(a)=x$. Thus, it suffices to show that
\begin{equation*}
(f(a),h(a),H(a))=(f(a),h(a),(\ct{H(a)^{-1}}{\refl{i(f(a))}})^{-1}),
\end{equation*}
which is a routine exercise. 

Now we note that the left and right maps in this square are both equivalences. Therefore it follows that the top map is an equivalence if and only if the bottom map is. The claim now follows by \cref{thm:fib_equiv}.
\end{proof}

\begin{cor}\label{cor:pb_trunc}
Consider a pullback square
\begin{equation*}
\begin{tikzcd}
C \arrow[r,"q"] \arrow[d,swap,"p"] & B \arrow[d,"g"] \\
A \arrow[r,swap,"f"] & X.
\end{tikzcd}
\end{equation*}
If $g$ is a $k$-truncated map, then so is $p$. In particular, if $g$ is an embedding then so is $p$.\index{truncated!map!pullbacks of truncated maps}\index{embedding!pullbacks of embeddings}
\end{cor}

\begin{proof}
Since the square is assumed to be a pullback square, it follows from \cref{cor:pb_fibequiv} that for each $x:A$, the fiber $\fib{p}{x}$ is equivalent to the fiber $\fib{g}{f(x)}$, which is $k$-truncated. Since $k$-truncated types are closed under equivalences by \cref{thm:ktype_eqv}, it follows that $p$ is a $k$-truncated map.
\end{proof}

\begin{cor}\label{cor:pb_equiv}
Consider a commuting square
\begin{equation*}
\begin{tikzcd}
C \arrow[r,"q"] \arrow[d,swap,"p"] & B \arrow[d,"g"] \\
A \arrow[r,swap,"f"] & X.
\end{tikzcd}
\end{equation*}
and suppose that $g$ is an equivalence. Then the following are equivalent:
\begin{enumerate}
\item The square is a pullback square.
\item The map $p:C\to A$ is an equivalence.\index{equivalence!pullback of}
\end{enumerate}
\end{cor}

\begin{proof}
If the square is a pullback square, then by \cref{thm:pb_fibequiv} the fibers of $p$ are equivalent to the fibers of $g$, which are contractible by \cref{thm:contr_equiv}. Thus it follows that $p$ is a contractible map, and hence that $p$ is an equivalence.

If $p$ is an equivalence, then by \cref{thm:contr_equiv} both $\fib{p}{x}$ and $\fib{g}{f(x)}$ are contractible for any $x:X$. It follows by \cref{ex:contr_equiv} that the induced map $\fib{p}{x}\to\fib{g}{f(x)}$ is an equivalence. Thus we apply \cref{cor:pb_fibequiv} to conclude that the square is a pullback.
\end{proof}

\begin{thm}\label{thm:pb_fibequiv_complete}
Consider a diagram of the form
\begin{equation*}
\begin{tikzcd}
A \arrow[d,swap,"f"] & B \arrow[d,"g"] \\
X \arrow[r,swap,"h"] & Y.
\end{tikzcd}
\end{equation*}
Then the type of triples $(i,H,p)$ consisting of a map $i:A\to B$, a homotopy $H:h\circ f\htpy g\circ i$, and a term $p$ witnessing that the square
\begin{equation*}
\begin{tikzcd}
A \arrow[d,swap,"f"] \arrow[r,"i"] & B \arrow[d,"g"] \\
X \arrow[r,swap,"h"] & Y.
\end{tikzcd}
\end{equation*}
is a pullback square, is equivalent to the type of families of equivalences
\begin{equation*}
\prd{x:X}\eqv{\fib{f}{x}}{\fib{g}{h(x)}}.
\end{equation*}
\end{thm}

\begin{cor}\label{cor:pb_fibequiv_complete}
Let $h:X\to Y$ be a map, and let $P$ and $Q$ be families over $X$ and $Y$, respectively.
Then the type of triples $(i,H,p)$ consisting of a map 
\begin{equation*}
i:\Big(\sm{x:X}P(x)\Big)\to \Big(\sm{y:Y}Q(y)\Big),
\end{equation*}
a homotopy $H:h\circ \proj 1\htpy \proj 1\circ i$, and a term $p$ witnessing that the square
\begin{equation*}
\begin{tikzcd}
\sm{x:X}P(x) \arrow[d,swap,"\proj 1"] \arrow[r,"i"] & \sm{y:Y}Q(y) \arrow[d,"\proj 1"] \\
X \arrow[r,swap,"h"] & Y.
\end{tikzcd}
\end{equation*}
is a pullback square, is equivalent to the type of families of equivalences
\begin{equation*}
\prd{x:X}\eqv{P(x)}{Q(h(x))}.
\end{equation*}
\end{cor}

One useful application of the connection between pullbacks and families of equivalences is the following theorem, which is also called the \define{pasting property} of pullbacks.\index{pasting property!of pullbacks}

\begin{thm}\label{thm:pb_pasting}
Consider a commuting diagram of the form
\begin{equation*}
\begin{tikzcd}
A \arrow[r,"k"] \arrow[d,swap,"f"] & B \arrow[r,"l"] \arrow[d,"g"] & C \arrow[d,"h"] \\
X \arrow[r,swap,"i"] & Y \arrow[r,swap,"j"] & Z
\end{tikzcd}
\end{equation*}
with homotopies $H:i\circ f\htpy g\circ k$ and $K:j\circ g\htpy h\circ l$, and the homotopy
\begin{equation*}
\ct{(j\cdot H)}{(K\cdot k)}:j\circ i\circ f\htpy h\circ l\circ k
\end{equation*}
witnessing that the outer rectangle commutes. Furthermore, suppose that the square on the right is a pullback square. Then the following are equivalent:
\begin{samepage}%
\begin{enumerate}
\item The square on the left is a pullback square.
\item The outer rectangle is a pullback square.
\end{enumerate}%
\end{samepage}%
\end{thm}

\begin{proof}
The commutativity of the two squares and the outer rectangle induces a commuting triangle
\begin{equation*}
\begin{tikzcd}[column sep=tiny]
\fib{f}{x} \arrow[rr,"\fibsquare_{(f,k,H)}(x)"] \arrow[dr,swap,"\fibsquare_{f,l\circ k,\ct{(j\cdot H)}{(K\cdot k)}}(x)"] & & \fib{g}{i(x)} \arrow[dl,"\fibsquare_{(g,l,K)}(i(x))"] \\
& \fib{h}{j(i(x))}.
\end{tikzcd}
\end{equation*}
A homotopy witnessing that the triangle commutes is constructed by a routine calculation.

Since the triangle commutes, and since the map $\fibsquare_{(g,l,K)}(i(x))$ is an equivalence for each $x:X$ by \cref{cor:pb_fibequiv}, it follows
by the 3-for-2 property of equivalences that for each $x:X$ the top map in the triangle is an equivalence if and only if the left map is an equivalence.
The claim now follows by a second application of \cref{cor:pb_fibequiv}.
\end{proof}

\subsection{Descent theorems for coproducts and \texorpdfstring{$\Sigma$}{Σ}-types}

\begin{thm}\label{thm:descent-coprod}
Consider maps $f:A'\to A$ and $g:B'\to B$, a map $h:X'\to X$, and commuting squares of the form
\begin{equation*}
\begin{tikzcd}
A' \arrow[r] \arrow[d,swap,"f"] & X' \arrow[d,"h"] & B' \arrow[r] \arrow[d,swap,"g"] & X' \arrow[d,"h"] \\
A \arrow[r] & X & B \arrow[r] & X.
\end{tikzcd}
\end{equation*}
Then the following are equivalent:
\begin{enumerate}
\item Both squares are pullback squares.
\item The commuting square 
\begin{equation*}
\begin{tikzcd}
A'+B' \arrow[d,swap,"f+g"] \arrow[r] & X' \arrow[d,"h"] \\
A+B \arrow[r] & X
\end{tikzcd}
\end{equation*}
is a pullback square.
\end{enumerate}
\end{thm}

\begin{proof}
By \cref{cor:pb_fibequiv} it suffices to show that the following are equivalent:
\begin{enumerate}
\item For each $x:A$ the map
\begin{equation*}
\fibsquare:\fib{f}{x}\to\fib{h}{\alpha_A(x)}
\end{equation*}
is an equivalence, and for each $y:B$ the map
\begin{equation*}
\fibsquare:\fib{g}{y}\to\fib{h}{\alpha_B(y)}
\end{equation*}
is an equivalence.
\item For each $t:A+B$ the map
\begin{equation*}
\fibsquare:\fib{f+g}{t} \to \fib{h}{\alpha(t)}
\end{equation*}
is an equivalence.
\end{enumerate}
By the dependent universal property of coproducts, the second claim is equivalent to the claim that both for each $x:A$ the map
\begin{equation*}
\fibsquare:\fib{f+g}{\inl(x)}\to \fib{h}{\alpha_A(x)}
\end{equation*}
is an equivalence, and for each $y:B$, the map
\begin{equation*}
\fibsquare:\fib{f+g}{\inr(y)}\to \fib{h}{\alpha_B(y)} 
\end{equation*}
is an equivalence.

We claim that there is a commuting triangle
\begin{equation*}
\begin{tikzcd}[column sep=-1em]
\fib{f}{x} \arrow[rr] \arrow[dr] & & \fib{f+g}{\inl(x)} \arrow[dl] \\
\phantom{\fib{f+g}{\inl(x)}} & \fib{h}{\alpha_A(x)}
\end{tikzcd}
\end{equation*}
for every $x:A$. To see that the triangle commutes, we need to construct an identification


The top map is given by
\begin{equation*}
(a',p)\mapsto (\inl(a'),\ap{\inl}{p}).
\end{equation*}
The triangle then commutes by the homotopy
\begin{equation*}
(a',p)\mapsto \mathsf{eq\usc{}pair}(\refl,\ap{\mathsf{concat}(H(a')^{-1})}{\mathsf{ap\usc{}comp}_{[\alpha_A,\alpha_B],inl}})
\end{equation*}
We note that the top map is an equivalence, so it follows by the 3-for-2 property of equivalences that the left map is an equivalence if and only if the right map is an equivalence. 

Similarly, there is a commuting triangle
\begin{equation*}
\begin{tikzcd}[column sep=-1em]
\fib{g}{y} \arrow[rr] \arrow[dr] & & \fib{f+g}{\inr(y)} \arrow[dl] \\
\phantom{\fib{f+g}{\inr(y)}} & \fib{h}{\alpha_B(y)}
\end{tikzcd}
\end{equation*}
in which the top map is an equivalence, completing the proof.
\end{proof}

In the following corollary we conclude that coproducts distribute over pullbacks. 

\begin{cor}
Consider a cospan of the form
\begin{equation*}
\begin{tikzcd}
& Y \arrow[d] \\
A+B \arrow[r] & X.
\end{tikzcd}
\end{equation*}
Then there is an equivalence
\begin{equation*}
(A+B)\times_X Y \simeq (A\times_X Y)+(B\times_X Y).
\end{equation*}
\end{cor}

\begin{thm}\label{thm:descent-Sigma}
Consider a family of maps $f_i:A'_i\to A_i$ indexed by a type $I$, a map $h:X'\to X$, and a commuting square
\begin{equation*}
\begin{tikzcd}
A'_i \arrow[r] \arrow[d,swap,"f_i"] & X' \arrow[d,"h"] \\
A_i \arrow[r,swap,"\alpha_i"] & X
\end{tikzcd}
\end{equation*}
for each $i:I$. Then the following are equivalent:
\begin{enumerate}
\item For each $i:I$ the square is a pullback square.
\item The commuting square
\begin{equation*}
\begin{tikzcd}
\sm{i:I}A'_i \arrow[d,swap,"\tot{f}"] \arrow[r] & X' \arrow[d,"h"] \\
\sm{i:I}A_i \arrow[r,"\alpha"] & X
\end{tikzcd}
\end{equation*}
is a pullback square.
\end{enumerate}
\end{thm}

\begin{proof}
By \cref{cor:pb_fibequiv} it suffices to show that the following are equivalent for each $i:I$ and $a:A_i$:
\begin{enumerate}
\item The map
\begin{equation*}
\fibsquare : \fib{f_i}{a} \to \fib{g}{\alpha_i(a)}
\end{equation*}
is an equivalence.
\item The map
\begin{equation*}
\fibsquare : \fib{\tot{f}}{i,a}\to \fib{g}{\alpha_i(a)}
\end{equation*}
is an equivalence.
\end{enumerate}
To see this, note that we have a commuting triangle
\begin{equation*}
\begin{tikzcd}[column sep=-1em]
\fib{f_i}{a} \arrow[rr] \arrow[dr] & & \fib{\tot{f}}{i,a} \arrow[dl] \\
\phantom{\fib{\tot{f}}{i,a}} & \fib{g}{\alpha_i(a)},
\end{tikzcd}
\end{equation*}
where the top map is an equivalence by \cref{lem:fib_total}. Therefore the claim follows by the 3-for-2 property of equivalences.
\end{proof}

In the following corollary we conclude that $\Sigma$ distributes over coproducts.

\begin{cor}
Consider a cospan of the form
\begin{equation*}
\begin{tikzcd}
& Y \arrow[d] \\
\sm{i:I}A_i \arrow[r] & X.
\end{tikzcd}
\end{equation*}
Then there is an equivalence
\begin{equation*}
\Big(\sm{i:I}A_i\Big)\times_X Y \simeq \sm{i:I} (A_i\times_X Y).
\end{equation*}
\end{cor}



\begin{exercises}
\exercise \label{ex:id_pb}\index{identity type!as pullback}
\begin{subexenum}
\item Show that the square\index{identity type!as pullback}
\begin{equation*}
\begin{tikzcd}
(x=y) \arrow[r] \arrow[d] & \unit \arrow[d,"\mathsf{const}_y"] \\
\unit \arrow[r,swap,"\mathsf{const}_x"] & A
\end{tikzcd}
\end{equation*}
is a pullback square.
\item Show that the square\index{diagonal!of a type!fibers of}
\begin{equation*}
\begin{tikzcd}[column sep=large]
(x=y) \arrow[r,"\mathsf{const}_{x}"] \arrow[d,swap,"\mathsf{const}_\ttt"] & A \arrow[d,"\delta_A"] \\
\unit \arrow[r,swap,"{\mathsf{const}_{(x,y)}}"] & A\times A
\end{tikzcd}
\end{equation*}
is a pullback square, where $\delta_A:A\to A\times A$ is the diagonal of $A$, defined in \cref{ex:diagonal}.
\end{subexenum}
\exercise \label{ex:trunc_diagonal_map}In this exercise we give an alternative characterization of the notion of $k$-truncated map, compared to \cref{thm:trunc_ap}. Given a map $f:A\to X$ define the \define{diagonal}\index{diagonal!of a map} of $f$ to be the map $\delta_f:A\to A\times_X A$ given by $x\mapsto (x,x,\refl{f(x)})$.
\begin{subexenum}
\item Construct an equivalence
\begin{equation*}
\eqv{\fib{\delta_f}{(x,y,p)}}{\fib{\apfunc{f}}{p}}
\end{equation*}
to show that the square\index{action on paths!fibers of}\index{diagonal!of a map!fibers of}
\begin{equation*}
\begin{tikzcd}[column sep=large]
\fib{\apfunc{f}}{p} \arrow[r,"\mathsf{const}_x"] \arrow[d,swap,"\mathsf{const}_\ttt"] & A \arrow[d,"\delta_f"] \\
\unit \arrow[r,swap,"{\mathsf{const}_{(x,y,p)}}"] & A\times_X A
\end{tikzcd}
\end{equation*}
is a pullback square, for every $x,y:A$ and $p:f(x)=f(y)$.
\item Show that a map $f:A\to X$ is $(k+1)$-truncated if and only if $\delta_f$ is $k$-truncated.\index{truncated!map!by truncatedness of diagonal}
\end{subexenum}
Conclude that $f$ is an embedding if and only if $\delta_f$ is an equivalence.\index{embedding!diagonal is an equivalence}
\exercise Consider a commuting square 
\begin{equation*}
\begin{tikzcd}
C \arrow[r,"q"] \arrow[d,swap,"p"] & B \arrow[d,"g"] \\
A \arrow[r,swap,"f"] & X 
\end{tikzcd}
\end{equation*}
with $H:f\circ p\htpy g\circ q$. Show that this square is a pullback square if and only if the square
\begin{equation*}
\begin{tikzcd}
C \arrow[r,"p"] \arrow[d,swap,"q"] & A \arrow[d,"f"] \\
B \arrow[r,swap,"g"] & X 
\end{tikzcd}
\end{equation*}
with $H^{-1}:g\circ q\htpy f\circ p$ is a pullback square.
\exercise Show that any square of the form
\begin{equation*}
\begin{tikzcd}
C \arrow[r] \arrow[d] & B \arrow[d] \\
\emptyt \arrow[r] & X
\end{tikzcd}
\end{equation*}
commutes and is a pullback square. This is the \emph{descent property} of the empty type.\index{descent!empty type}
\exercise Consider a commuting square
\begin{equation*}
\begin{tikzcd}
C \arrow[r,"q"] \arrow[d,swap,"p"] & B \arrow[d,"g"] \\
A \arrow[r,swap,"f"] & X
\end{tikzcd}
\end{equation*}
with $H:f\circ p\htpy g\circ q$. Show that the following are equivalent:
\begin{enumerate}
\item The square is a pullback square.
\item For every type $T$, the commuting square
\begin{equation*}
\begin{tikzcd}
C^T \arrow[r,"q\circ\blank"] \arrow[d,swap,"p\circ\blank"] & B^T \arrow[d,"g\circ\blank"] \\
A^T \arrow[r,swap,"f\circ\blank"] & X^T
\end{tikzcd}
\end{equation*}
is a pullback square.
\end{enumerate}
Note: property (ii) is really just a rephrasing of the universal property of pullbacks.\index{pullback square!universal property}
\exercise \label{ex:pb_diagonal}Consider a commuting square
\begin{equation*}
\begin{tikzcd}
C \arrow[r,"q"] \arrow[d,swap,"p"] & B \arrow[d,"g"] \\
A \arrow[r,swap,"f"] & X
\end{tikzcd}
\end{equation*}
with $H:f\circ p\htpy g\circ q$. Show that the following are equivalent:
\begin{enumerate}
\item The square is a pullback square.
\item The square
\begin{equation*}
\begin{tikzcd}
C \arrow[r,"g\circ q"] \arrow[d,swap,"{\lam{x}(p(x),q(x))}"] & X \arrow[d,"\delta_X"] \\
A\times B \arrow[r,swap,"f\times g"] & X\times X
\end{tikzcd}
\end{equation*}
which commutes by $\lam{z}\mathsf{eq\usc{}pair}(H(z),\refl{g(q(z))})$ is a pullback square.
\end{enumerate}
\exercise \label{ex:pb_prod}Consider two commuting squares\index{pullback!cartesian products of pullbacks}
\begin{equation*}
\begin{tikzcd}
C_1 \arrow[r] \arrow[d] & B_1 \arrow[d] & C_2 \arrow[r] \arrow[d] & B_2 \arrow[d] \\
A_1 \arrow[r] & X_1 & A_2 \arrow[r] & X_2.
\end{tikzcd}
\end{equation*}
\begin{subexenum}
\item Show that if both squares are pullback squares, then the square
\begin{equation*}
\begin{tikzcd}
C_1\times C_2 \arrow[r] \arrow[d] & B_1\times B_2 \arrow[d] \\
A_1 \times A_2 \arrow[r] & X_1\times X_2. 
\end{tikzcd}
\end{equation*}
is also a pullback square.
\item Show that if there are terms $t_1:A_1\times_{X_1}B_1$ and $t_2:A_2\times_{X_2}B_2$, then the converse of (a) also holds.
\end{subexenum}
\exercise Consider for each $i:I$ a pullback square
\begin{equation*}
\begin{tikzcd}
C_i \arrow[r,"q_i"] \arrow[d,swap,"p_i"] & B_i \arrow[d,"g_i"] \\
A_i \arrow[r,swap,"f_i"] & X_i
\end{tikzcd}
\end{equation*}
with $H_i: f_i\circ p_i\htpy g_i\circ q_i$.\index{pullback!Pi-type of pullbacks@{$\Pi$-type of pullbacks}}\label{ex:pb_pi}
Show that the commuting square
\begin{equation*}
\begin{tikzcd}
\prd{i:I}C_i \arrow[r] \arrow[d] & \prd{i:I}B_i \arrow[d] \\
\prd{i:I}A_i \arrow[r] & \prd{i:I}X_i
\end{tikzcd}
\end{equation*}
is a pullback square.
  \exercise Let $f:A\to B$ be a map. Show that the following are equivalent:
  \begin{enumerate}
  \item The commuting square
    \begin{equation*}
      \begin{tikzcd}
        A \arrow[d,swap,"f"] \arrow[r] & \brck{A} \arrow[d,"\brck{f}"] \\
        B \arrow[r] & \brck{B}.
      \end{tikzcd}
    \end{equation*}
    is a pullback square.
  \item There is a term of type $A\to\isequiv(f)$.
  \item The commuting square
    \begin{equation*}
      \begin{tikzcd}
        A\times A \arrow[r,"f\times f"] \arrow[d,swap,"\proj 1"] & B \times B \arrow[d,"\proj 1"] \\
        A \arrow[r,swap,"f"] & B
      \end{tikzcd}
    \end{equation*}
    is a pullback square. 
  \end{enumerate}
  \exercise Consider a pullback square
  \begin{equation*}
    \begin{tikzcd}
      A' \arrow[d,swap,"{f'}"] \arrow[r,"p"] & A \arrow[d,"f"] \\
      B' \arrow[r,swap,"q"] & B,
    \end{tikzcd}
  \end{equation*}
  in which $q:B'\to B$ is surjective. Show that if $f':A'\to B'$ is an embedding, then so is $f:A\to B$.
    \exercise Consider a family of diagrams of the form
  \begin{equation*}
    \begin{tikzcd}
      A_i \arrow[r] \arrow[d,swap,"{f_i}"] &
      C \arrow[r] \arrow[d,"g"] & X \arrow[d,"h"] \\
      B_i \arrow[r] & D \arrow[r] & Y 
    \end{tikzcd}
  \end{equation*}
  indexed by $i:I$, in which the left squares are pullback squares,
  and assume that the induced map
  \begin{equation*}
    \Big(\sm{i:I}B_i\Big)\to D
  \end{equation*}
  is surjective. Show that the following are equivalent:
  \begin{enumerate}
  \item For each $i:I$ the outer rectangle is a pullback square.
  \item The right square is a pullback square.
  \end{enumerate}
  Hint: By \cref{thm:descent-Sigma} it suffices to prove this equivalence for a single diagram of the form
  \begin{equation*}
    \begin{tikzcd}
      A \arrow[r] \arrow[d,swap,"{f}"] &
      C \arrow[r] \arrow[d,swap,"g"] & X \arrow[d,"h"] \\
      B \arrow[r] & D \arrow[r] & Y 
    \end{tikzcd}
  \end{equation*}
  where the map $B \to D$ is assumed to be surjective.
    \exercise Consider a pullback square
  \begin{equation*}
    \begin{tikzcd}
      E' \arrow[d,swap,"{p'}"] \arrow[r,"g"] & E \arrow[d,"p"] \\
      B' \arrow[r,swap,"f"] & B
    \end{tikzcd}
  \end{equation*}
  in which $p$ is assumed to be surjective. Show that $p'$ is also surjective, and show that the following are equivalent:
  \begin{enumerate}
  \item The map $f$ is an equivalence.
  \item The map $g$ is an equivalence.
  \end{enumerate}
  \exercise Show that a map $f:A\to B$ is an equivalence if and only if the square
  \begin{equation*}
    \begin{tikzcd}
      A^B \arrow[r,"f\circ\blank"] \arrow[d,swap,"\blank\circ f"] & B^B \arrow[d,"\blank\circ f"] \\
      A^A \arrow[r,swap,"f\circ\blank"] & B^A
    \end{tikzcd}
  \end{equation*}
  is a pullback square.
%\exercise
%\begin{subexenum}
%\item Show that \index{equivalence!type of equivalences!as pullback}
%\begin{equation*}
%\begin{tikzcd}[column sep=8em]
%\eqv{A}{B} \arrow[r] \arrow[d] & \unit \arrow[d,"{(\idfunc[A],\idfunc[B])}"] \\
%A^B\times B^A \times A^B \arrow[r,swap,"{(h,f,g)\mapsto (h\circ f,f\circ g)}"] & A^A \times B^B
%\end{tikzcd}
%\end{equation*}
%is a pullback square.
%\item Show that \index{contractible!type of contractibility!as pullback}
%\begin{equation*}
%\begin{tikzcd}[column sep=6em]
%\iscontr(A) \arrow[r,"\mathsf{const}_{\ttt}"] \arrow[d,swap,"\proj 1"] & \unit \arrow[d,"{\lam{\ttt}\idfunc[A]}"] \\
%A \arrow[r,swap,"{\lam{x}\mathsf{const}_x}"] & A^A
%\end{tikzcd}
%\end{equation*}
%is a pullback square.
%\end{subexenum}
%\exercise Consider a commuting square
%\begin{equation*}
%\begin{tikzcd}
%C \arrow[r] \arrow[d] & A \arrow[d] \\
%B \arrow[r] & X.
%\end{tikzcd}
%\end{equation*}
%Show that this square is cartesian if and only if the induced map $C\to A\times_X B$ has a retraction.
%\end{subexenum}
%\exercise Suppose that the squares
%\begin{equation*}
%\begin{tikzcd}
%C \arrow[r,"q"] \arrow[d,swap,"p"] & B \arrow[d,"g"] & {C'} \arrow[r,"{q'}"] \arrow[d,swap,"{p'}"] & B \arrow[d,"g"] \\
%A \arrow[r,swap,"f"] & X & A \arrow[r,swap,"f"] & X
%\end{tikzcd}
%\end{equation*}
%with homotopies $H:f\circ p \htpy g\circ q$ and $H':f\circ p'\htpy g\circ q'$ are both pullback squares. Show that the type of equivalences $e:\eqv{C'}{C}$ equipped with an identification
%\begin{equation*}
%\mathsf{cone\usc{}map}((p,q,H),e)=(p',q',H')
%\end{equation*}
%is contractible.
\begin{comment}
\exercise Consider a \define{natural transformation of cospans}\index{cospan!natural transformation of}, i.e., a commuting diagram of the form
\begin{equation*}
\begin{tikzcd}
A \arrow[r,"f"] \arrow[d,swap,"i"] & X \arrow[d,swap,"j"] & B \arrow[l,swap,"g"] \arrow[d,"k"] \\
A' \arrow[r,swap,"{f'}"] & X' & B'. \arrow[l,"{g'}"]
\end{tikzcd}
\end{equation*}
Show that the map
\begin{equation*}
(a,b,p)\mapsto (i(a),j(b),\mathsf{ap}_k(p)): A \times_X B \to A'\times_{X'} B'
\end{equation*}
is $k$-truncated if each of the vertical maps is.
\end{comment}
\begin{comment}
\exercise \label{ex:pb_fib}Consider a commuting square
\begin{equation*}
\begin{tikzcd}
C \arrow[d,swap,"p"] \arrow[r,"q"] & B \arrow[d,"g"] \\
A \arrow[r,swap,"f"] & X.
\end{tikzcd}
\end{equation*}
with $H:f\circ p\htpy g\circ q$, and let $h:C\to A\times_X B$ be the map given by $h(z)\defeq (p(c),q(c),H(c))$. 
Show that the square
\begin{equation*}
\begin{tikzcd}[column sep=6.5em]
\fib{\mathsf{gap}(p,q,H)}{(a,b,\alpha)} \arrow[d,swap,"\mathsf{const}_{\ttt}"] \arrow[r,"{\lam{(c,\beta)}(c,\ap{\pi_1}{\beta})}"] & \fib{p}{a} \arrow[d,"{\fibf{(f,g,H)}}"] \\
\unit \arrow[r,swap,"\mathsf{const}_{(b,\alpha^{-1})}"] & \fib{g}{f(a)}
\end{tikzcd}
\end{equation*}
\end{comment}
\end{exercises}

\input{pushout}
\section{Cubical diagrams}

In order to proceed with the development of pullbacks and pushouts, it is useful to study commuting diagrams of the form
\begin{equation*}
  \begin{tikzcd}
    & C' \arrow[dl] \arrow[d] \arrow[dr] \\
    A' \arrow[d] & C \arrow[dl] \arrow[dr] & B' \arrow[dl,crossing over] \arrow[d] \\
    A \arrow[dr] & X' \arrow[from=ul,crossing over] \arrow[d] & B \arrow[dl] \\
    & X.
  \end{tikzcd}
\end{equation*}
In these diagrams there are six homotopies witnessing that the faces of the cube commute, as well as a homotopy of homotopies witnessing that the cube as a whole commutes.

Once the basic definitions of cubes are established, we focus on pullbacks and pushouts that appear in different configurations in these cubical diagrams. For example, if all the vertical maps in a commuting cube are equivalences, then the top square is a pullback square if and only if the bottom square is a pullback square. In \cref{chap:descent} we will use cubical diagrams in our formulation of the universality and descent theorems for pushouts.

In the first main theorem of this section we show that given a commuting cube in which the bottom square is a pullback square, the top square is a pullback square if and only if the induced square of fibers of the vertical maps is a pullback square. This theorem should be compared to \cref{cor:pb_fibequiv}, where we showed that a square is a pullback square if and only if it induces equivalences on the fibers of the vertical maps.

In our second main theorem we use the previous result to derive the 3-by-3 properties for pullbacks and pushouts.

\subsection{Commuting cubes}
\begin{defn}\label{defn:cube}
A \define{commuting cube}\index{commuting cube}
\begin{equation*}
\begin{tikzcd}[column sep=large,row sep=large]
& C' \arrow[dl,swap,"{p'}"] \arrow[dr,"{q'}"] \arrow[d,swap,"{h_C}" near end] \\
A' \arrow[d,swap,"{h_A}"] & C \arrow[dl,swap,"{p}" very near start] \arrow[dr,"{q}" very near start] & B' \arrow[dl,crossing over,"{g'}" near end] \arrow[d,"{h_B}"] \\
A \arrow[dr,swap,"f"] & X' \arrow[d,swap,"{h_X}" near start] \arrow[from=ul,crossing over,swap,"{f'}" near end] & B \arrow[dl,"{g}"] \\
& X
\end{tikzcd}
\end{equation*}
consists of types and maps as indicated in the diagram, equipped with
\begin{enumerate}
\item homotopies
  \begin{align*}
    \mathsf{top} & : f' \circ p' \htpy g' \circ q' \\
    \mathsf{back\usc{}left} & : p \circ h_C \htpy h_A \circ p' \\
    \mathsf{back\usc{}right} & : q \circ h_C \htpy h_B \circ q' \\
    \mathsf{front\usc{}left} & : f \circ h_A \htpy h_X \circ f' \\
    \mathsf{front\usc{}right} & : g \circ h_B \htpy h_X \circ g' \\
    \mathsf{bottom} & : f \circ p \htpy g \circ q
  \end{align*}
  witnessing that the 6 faces of the cube commute,
\item and a homotopy 
  \begin{align*}
    % ((((h ·l back-left) ∙h (front-left ·r f')) ∙h (hD ·l top))) ~
    % ((bottom ·r hA) ∙h ((k ·l back-right) ∙h (front-right ·r g')))
\mathsf{coh\usc{}cube} & : \ct{(\ct{(f \cdot \mathsf{back\usc{}left})}{(\mathsf{front\usc{}left}\cdot p')})}{(h_X \cdot \mathsf{top})} \\
& \qquad \htpy \ct{(\mathsf{bottom}\cdot h_C)}{(\ct{(g \cdot \mathsf{back\usc{}right})}{(\mathsf{front\usc{}right}\cdot q')})}
\end{align*}
filling the cube.
\end{enumerate}
\end{defn}

In the following lemma we show that if a cube commutes, then so do its rotations and mirror symmetries (that preserve the directions of the arrows).\footnote{The group acting on commuting cubes of maps is the \emph{dihedral group} $D_3$ which has order $6$.} This fact is obviously true, but there is some `path algebra' involved that we wish to demonstrate at least once.

\begin{lem}
  Consider a commuting cube
  \begin{equation*}
    \begin{tikzcd}
      & C' \arrow[dl] \arrow[d] \arrow[dr] \\
      A' \arrow[d] & C \arrow[dl] \arrow[dr] & B' \arrow[dl,crossing over] \arrow[d] \\
      A \arrow[dr] & X' \arrow[from=ul,crossing over] \arrow[d] & B \arrow[dl] \\
      & X.
    \end{tikzcd}
  \end{equation*}
  Then the cubes

  \begin{center}
  \begin{minipage}{.3\textwidth}
  \begin{equation*}
    \begin{tikzcd}
      & C' \arrow[dl] \arrow[d] \arrow[dr] \\
      C \arrow[d] & B' \arrow[dl] \arrow[dr] & A' \arrow[dl,crossing over] \arrow[d] \\
      B \arrow[dr] & A \arrow[from=ul,crossing over] \arrow[d] & X' \arrow[dl] \\
      & X
    \end{tikzcd}
  \end{equation*}
  \end{minipage}
  \begin{minipage}{.3\textwidth}
  \begin{equation*}
    \begin{tikzcd}
      & C' \arrow[dl] \arrow[d] \arrow[dr] \\
      B' \arrow[d] & A' \arrow[dl] \arrow[dr] & C \arrow[dl,crossing over] \arrow[d] \\
      X' \arrow[dr] & B \arrow[from=ul,crossing over] \arrow[d] & A \arrow[dl] \\
      & X
    \end{tikzcd}
  \end{equation*}
  \end{minipage}

  \begin{minipage}{.3\textwidth}
  \begin{equation*}
    \begin{tikzcd}
      & C' \arrow[dl] \arrow[d] \arrow[dr] \\
      C \arrow[d] & A' \arrow[dl] \arrow[dr] & B' \arrow[dl,crossing over] \arrow[d] \\
      A \arrow[dr] & B \arrow[from=ul,crossing over] \arrow[d] & X' \arrow[dl] \\
      & X.
    \end{tikzcd}
  \end{equation*}
  \end{minipage}
  \begin{minipage}{.3\textwidth}
  \begin{equation*}
    \begin{tikzcd}
      & C' \arrow[dl] \arrow[d] \arrow[dr] \\
      A' \arrow[d] & B' \arrow[dl] \arrow[dr] & C \arrow[dl,crossing over] \arrow[d] \\
      X' \arrow[dr] & A \arrow[from=ul,crossing over] \arrow[d] & B \arrow[dl] \\
      & X.
    \end{tikzcd}
  \end{equation*}
  \end{minipage}
  \begin{minipage}{.3\textwidth}
  \begin{equation*}
    \begin{tikzcd}
      & C' \arrow[dl] \arrow[d] \arrow[dr] \\
      B' \arrow[d] & C \arrow[dl] \arrow[dr] & A' \arrow[dl,crossing over] \arrow[d] \\
      B \arrow[dr] & X' \arrow[from=ul,crossing over] \arrow[d] & A \arrow[dl] \\
      & X.
    \end{tikzcd}
  \end{equation*}
  \end{minipage}
  \end{center}
  also commute.
\end{lem}

\begin{proof}
  We only show that the first cube commutes, which is obtained by a counter-clockwise rotation of the original cube around the axis through $C'$ and $X$. The other cases are similar, and they are formalized in the accompagnying Agda library.

  First we list the homotopies witnessing that the faces of the cube commute:
  \begin{align*}
    \mathsf{top}' & \defeq \mathsf{back\usc{}left} \\
    \mathsf{back\usc{}left}' & \defeq \mathsf{back\usc{}right}^{-1} \\
    \mathsf{back\usc{}right}' & \defeq \mathsf{top}^{-1} \\
    \mathsf{front\usc{}left}' & \defeq \mathsf{bottom}^{-1} \\
    \mathsf{front\usc{}right}' & \defeq \mathsf{front\usc{}left}^{-1} \\
    \mathsf{bottom}' & \defeq \mathsf{front\usc{}right}. 
  \end{align*}
  Thus, to show that the cube commutes, we have to show that there is a homotopy of type
  \begin{align*}
    & \ct{\Big(\ct{(g \cdot \mathsf{back\usc{}right}^{-1})}{(\mathsf{bottom}^{-1}\cdot h_C)}\Big)}{(f \cdot \mathsf{back\usc{}left})} \\
    & \qquad\qquad \htpy \ct{(\mathsf{front\usc{}right}\cdot q')}{\Big(\ct{(h_X \cdot \mathsf{top}^{-1})}{(\mathsf{front\usc{}left}^{-1}\cdot p')}\Big)}.
  \end{align*}
  Recall that $h\cdot H^{-1}\htpy (h\cdot H)^{-1}$ and $H^{-1}\cdot h\htpy (H\cdot h)^{-1}$, so it suffices to construct a homotopy
  \begin{align*}
    & \ct{\Big(\ct{(g \cdot \mathsf{back\usc{}right})^{-1}}{(\mathsf{bottom}\cdot h_C)^{-1}}\Big)}{(f \cdot \mathsf{back\usc{}left})} \\
    & \qquad\qquad \htpy \ct{(\mathsf{front\usc{}right}\cdot q')}{\Big(\ct{(h_X \cdot \mathsf{top})^{-1}}{(\mathsf{front\usc{}left}\cdot p')^{-1}}\Big)}.
  \end{align*}
  Now we note that pointwise, our goal is of the form
  \begin{equation*}
    \ct{(\ct{\varepsilon^{-1}}{\delta^{-1}})}{\alpha}=\ct{\zeta}{(\ct{\gamma^{-1}}{\beta^{-1}})}, %%% check greek alphabet
  \end{equation*}
  whereas the assumption that the original cube commutes yields an identification of the form
  \begin{equation*}
    \ct{(\ct{\alpha}{\beta})}{\gamma}=\ct{\delta}{(\ct{\varepsilon}{\zeta})}
  \end{equation*}
  Indeed, in the case that $\alpha$, $\beta$, $\gamma$, $\delta$, $\varepsilon$, and $\zeta$ are general identifications, we can conclude our goal using path induction on all of them.
\end{proof}

\begin{lem}
Given a commuting cube as in \cref{defn:cube} we obtain a commuting square
\begin{equation*}
\begin{tikzcd}
\fib{f_{1\check{1}1}}{x} \arrow[r] \arrow[d] & \fib{f_{0\check{1}1}}{f_{\check{1}01}(x)} \arrow[d] \\
\fib{f_{1\check{1}0}}{f_{10\check{1}}(x)} \arrow[r] & \fib{f_{0\check{1}0}}{f_{00\check{1}}(x)}
\end{tikzcd}
\end{equation*}
for any $x:A_{101}$. 
\end{lem}

\begin{lem}
Consider a commuting cube
\begin{equation*}
\begin{tikzcd}[column sep=large,row sep=large]
& C' \arrow[dl] \arrow[dr] \arrow[d] \\
A' \arrow[d] & C \arrow[dl] \arrow[dr] & B' \arrow[dl,crossing over] \arrow[d] \\
A \arrow[dr] & X' \arrow[d] \arrow[from=ul,crossing over] & B \arrow[dl] \\
& X,
\end{tikzcd}
\end{equation*}
If the bottom and front right squares are pullback squares, then the back left square is a pullback if and only if the top square is.
\end{lem}

\begin{rmk}\label{rmk:strongly-cartesian}
By rotating the cube we also obtain:
\begin{enumerate}
\item If the bottom and front left squares are pullback squares, then the back right square is a pullback if and only if the top square is.
\item If the front left and front right squares are pullback, then the back left square is a pullback if and only if the back right square is.
\end{enumerate}
By combining these statements it also follows that if the front left, front right, and bottom squares are pullback squares, then if any of the remaining three squares are pullback squares, all of them are. Cubes that consist entirely of pullback squares are sometimes called \define{strongly cartesian}\index{strongly cartesian cube}.
\end{rmk}

\subsection{Families of pullbacks}

\begin{lem}\label{lem:fiberwise-pullback}
Consider a pullback square\index{pullback!Sigma-type of pullbacks@{$\Sigma$-type of pullbacks}}
  \begin{equation*}
    \begin{tikzcd}
      C \arrow[r,"q"] \arrow[d,swap,"p"] & B \arrow[d,"g"] \\
      A \arrow[r,swap,"f"] & X
    \end{tikzcd}
  \end{equation*}
  with $H : f \circ p \htpy g \circ h$. Furthermore, consider type families $P_X$, $P_A$, $P_B$, and $P_C$ over $X$, $A$, $B$, and $C$ respectively, equipped with families of maps
  \begin{align*}
    f' & : \prd{a:A} P_A(a) \to P_X(f(a)) \\
    g' & : \prd{b:B} P_B(b) \to P_X(g(b)) \\
    p' & : \prd{c:C} P_C(c) \to P_A(p(c)) \\
    q' & : \prd{c:C} P_C(c) \to P_B(q(c)),
  \end{align*}
  and for each $c:C$ a homotopy $H'_c$ witnessing that the square
  \begin{equation}\label{eq:family-squares-pullback}
    \begin{tikzcd}
      P_C(c) \arrow[rr,"{q'_c}"] \arrow[d,swap,"{p'_c}"] & &[3em] P_B(q(c)) \arrow[d,"{g'_{q(c)}}"] \\
      P_A(p(c)) \arrow[r,swap,"{f'_{p(c)}}"] & P_X(f(p(c))) \arrow[r,swap,"{\tr_{P_X}(H(c))}"] & P_X(g(q(c)))
    \end{tikzcd}
  \end{equation}
  commutes. Then the following are equivalent:
  \begin{enumerate}
  \item For each $c:C$ the square in \cref{eq:family-squares-pullback} is a pullback square.
  \item The square
    \begin{equation}\label{eq:total-square-pullback}
      \begin{tikzcd}[column sep=huge]
        \sm{c:C}P_C(c)
        \arrow[r,"{\tot[q]{q'}}"] \arrow[d,swap,"{\tot[p]{p'}}"] &
        \sm{b:B}P_B(b) \arrow[d,"{\tot[g]{g'}}"] \\
        \sm{a:A}P_A(a) \arrow[r,swap,"{\tot[f]{f'}}"] & \sm{x:X}P_X(x)
      \end{tikzcd}
    \end{equation}
    is a pullback square.
  \end{enumerate}
\end{lem}


\begin{cor}
Consider a pullback square
\begin{equation*}
\begin{tikzcd}
C \arrow[r,"q"] \arrow[d,swap,"p"] & B \arrow[d,"g"] \\
A \arrow[r,swap,"f"] & X,
\end{tikzcd}
\end{equation*}
with $H:f\circ p\htpy g\circ q$, and let $c_1,c_2:C$. Then the square
\begin{equation*}
\begin{tikzcd}[column sep=8em]
(c_1=c_2) \arrow[r,"\apfunc{q}"] \arrow[d,swap,"\apfunc{p}"] & (q(c_1)=q(c_2)) \arrow[d,"\lam{\beta}\ct{H(c_1)}{\ap{g}{\beta}}"] \\
(p(c_1)=p(c_2)) \arrow[r,swap,"\lam{\alpha}\ct{\ap{f}{\alpha}}{H(c_2)}"] & f(p(c_1))=g(q(c_2)),
\end{tikzcd}
\end{equation*}
commutes and is a pullback square.
\end{cor}


\begin{thm}
  Consider a commuting cube
  \begin{equation*}
    \begin{tikzcd}
      & C' \arrow[dl] \arrow[dr] \arrow[d] \\
      A' \arrow[d] & C \arrow[dl] \arrow[dr] & B' \arrow[crossing over,dl] \arrow[d] \\
      A \arrow[dr] & X' \arrow[d] \arrow[from=ul,crossing over] & B \arrow[dl] \\
      & X
    \end{tikzcd}
  \end{equation*}
  in which the bottom square is a pullback square. Then the following are equivalent:
  \begin{enumerate}
  \item The top square is a pullback square.
  \item The square
    \begin{equation*}
      \begin{tikzcd}
        \fib{\gamma}{c} \arrow[d] \arrow[r] & \fib{\beta}{q(c)} \arrow[d] \\
        \fib{\alpha}{p(c)} \arrow[r] & \fib{\varphi}{f(p(c))}
      \end{tikzcd}
    \end{equation*}
    is a pullback square for each $c:C$.
  \end{enumerate}
\end{thm}


\subsection{The 3-by-3-properties for pullbacks and pushouts}

\begin{thm}
  Consider a commuting diagram of the form
  \begin{equation*}
    \begin{tikzcd}[column sep=large,row sep=large]
      AA \arrow[r,"Af"] \arrow[d,swap,"fA"] \arrow[dr,phantom,"\Rightarrow" description] & AX \arrow[d,swap,"fX"] & AB \arrow[l,swap,"Ag"] \arrow[d,"gB"] \arrow[dl,phantom,"\Leftarrow" description] \\
      XA \arrow[r,"Xf"] & XX & XB \arrow[l,swap,"Xg"] \\
      BA \arrow[u,"gA"] \arrow[r,swap,"Bf"] & BX \arrow[u,"gX"] & BB \arrow[u,swap,"gB"] \arrow[l,"Bg"]
    \end{tikzcd}
  \end{equation*}
  with homotopies
  \begin{align*}
    ff & : Xf \circ fA \htpy Af \circ fX \\
    fg & : Xg \circ gB \htpy Ag \circ fX \\
    gf & : 
  \end{align*}
  filling the (small) squares. Furthermore, consider
  pullback squares
  \begin{equation*}
    \begin{tikzcd}
      AC \arrow[r] \arrow[d] & AB \arrow[d] & XC \arrow[r] \arrow[d] & XB \arrow[d] & BC \arrow[r] \arrow[d] & BB \arrow[d] \\
      AA \arrow[r] & AX & XA \arrow[r] & XX & BA \arrow[r] & BX
    \end{tikzcd}
  \end{equation*}
  \begin{equation*}
    \begin{tikzcd}
      CA \arrow[r] \arrow[d] & BA \arrow[d] & CX \arrow[r] \arrow[d] & BX \arrow[d] & CB \arrow[r] \arrow[d] & BB \arrow[d] \\
      AA \arrow[r] & XA & AX \arrow[r] & XX & AB \arrow[r] & XB.
    \end{tikzcd}
  \end{equation*}
  Finally, consider a commuting square
  \begin{equation*}
    \begin{tikzcd}
      D_3 \arrow[r] \arrow[d] & D_2 \arrow[d] \\
      D_0 \arrow[r] & D_1.
    \end{tikzcd}
  \end{equation*}
  Then the following are equivalent:
  \begin{enumerate}
  \item This square is a pullback square.
  \item The induced square
    \begin{equation*}
      \begin{tikzcd}
        D_3 \arrow[r] \arrow[d] & C_3 \arrow[d] \\
        A_3 \arrow[r] & B_3
      \end{tikzcd}
    \end{equation*}
    is a pullback square.
  \end{enumerate}
\end{thm}

\begin{proof}
  First we construct an equivalence
  \begin{equation*}
    (A_0\times_{B_0}C_0)\times_{(A_1\times_{B_1}C_1)}(A_2\times_{B_2} C_2) \eqvsym (A_0\times_{A_1}A_2)\times_{(B_0\times_{B_1} B_2)} (C_0\times_{C_1}C_2).
  \end{equation*}
  Now it follows that we have an equivalence
  \begin{equation*}
    \mathsf{cone}(f_0,g_0)
  \end{equation*}
\end{proof}


\begin{exercises}
\item Some exercises.
\end{exercises}

\section{Universality and descent for pushouts}\label{chap:descent}

We begin this section with the idea that pushouts can be presented as higher inductive types. The general idea behind higher inductive types is that we can introduce new inductive types not only with constructors at the level of points, but also with constructors at the level of identifications. Pushouts form a basic class of examples that can be obtained as higher inductive types, because they come equipped with the structure of a cocone. The cocone $(i,j,H)$ in the commuting square
\begin{equation*}
  \begin{tikzcd}
    S \arrow[r,"g"] \arrow[d,swap,"f"] & B \arrow[d,"j"] \\
    A \arrow[r,swap,"i"] & C
  \end{tikzcd}
\end{equation*}
equips the type $C$ with two \emph{point constructors}
\begin{align*}
  i & : A \to C \\
  j & : B \to C \\
  \intertext{and a \emph{path constructor}}
  H & : \prd{s:S}i(f(s)) = j(g(s))
\end{align*}
that provides an identification $H(s):i(f(s))=j(g(s))$ for every $s:S$. The induction principle then specifies how to construct sections of families over $C$. Naturally, it takes not only the point constructors $i$ and $j$, but also the path constructor $H$ into account. 

The induction principle is one of several equivalent characterizations of pushouts. We will prove a theorem providing five equivalent characterizations of homotopy pushouts. Two of those we have already seen in \cref{thm:pushout_up}: the universal property and the pullback property. The other three are
\begin{enumerate}
\item the \emph{dependent pullback property},
\item the \emph{dependent universal property},
\item the \emph{induction principle}.
\end{enumerate}

An implication that is particularly useful among our five characterizations of pushouts, is the fact that the pullback property implies the dependent pullback property. We use the dependent pullback property to derive the \emph{universality of pushouts} (not to be confused with the universal property of pushouts), showing that for any commuting cube
\begin{equation*}
  \begin{tikzcd}
    & S' \arrow[dl] \arrow[d] \arrow[dr] \\
    A' \arrow[d] & S \arrow[dl] \arrow[dr] & B' \arrow[d] \arrow[dl,crossing over] \\
    A \arrow[dr] & D' \arrow[from=ul,crossing over] \arrow[d] & B \arrow[dl] \\
    & D
  \end{tikzcd}
\end{equation*}
in which the back left and right squares are pullback squares, if the front left and right squares are also pullback squares, then so is the induced square
\begin{equation*}
  \begin{tikzcd}
    A'\sqcup^{\mathcal{S}'}B' \arrow[r,densely dotted] \arrow[d,densely dotted] & D' \arrow[d] \\
    A\sqcup^S B \arrow[r,densely dotted] & D
  \end{tikzcd}
\end{equation*}

We then observe that the univalence axiom can be used together with the universal property of pushouts to obtain such families over pushouts in the first place. We prove the descent theorem, which asserts that for any diagram of the form
\begin{equation*}
  \begin{tikzcd}
    & S' \arrow[dl] \arrow[d] \arrow[dr] \\
    A' \arrow[d] & S \arrow[dl] \arrow[dr] & B' \arrow[d] \\
    A \arrow[dr] & & B \arrow[dl] \\
    & C
  \end{tikzcd}
\end{equation*}
in which the bottom square is a pushout square and the back left and right squares are pullback squares, there is a unique way of extending this to a commuting cube
\begin{equation*}
  \begin{tikzcd}
    & S' \arrow[dl] \arrow[d] \arrow[dr] \\
    A' \arrow[d] & S \arrow[dl] \arrow[dr] & B' \arrow[d] \\
    A \arrow[dr] & C' \arrow[from=ul,crossing over,densely dotted] \arrow[from=ur,crossing over,densely dotted] \arrow[d,densely dotted] & B \arrow[dl] \\
    & C
  \end{tikzcd}
\end{equation*}
in which also the front left and right squares are pullback squares. Thus the converse of the universality theorem for pushouts also follows. The descent property used to show that pullbacks distribute over pushouts, and to compute the fibers of maps out of pushouts (the source of many exercises).

We note that the computation rules in our treatment for the induction principle of homotopy pushouts are weak. In other words, they are identifications. In this course we have no need for judgmental computation rules. Our focus is instead on universal properties. We refer the reader who is interested in the more `traditional' higher inductive types with judgmental computation rules to \cite{hottbook}.

\subsection{Five equivalent characterizations of homotopy pushouts}

Consider a commuting square
\begin{equation}\label{eq:descent-pushout-square}
  \begin{tikzcd}
    S \arrow[r,"g"] \arrow[d,swap,"f"] & B \arrow[d,"j"] \\
    A \arrow[r,swap,"i"] & H
  \end{tikzcd}
\end{equation}
with $H:i\circ f \htpy j \circ g$, where we will sometimes write $\mathcal{S}$ for the span $A\leftarrow S\rightarrow B$. Our first goal is to formulate the induction
principle for pushouts, which specifies how to construct a section of an arbitrary type family $P$ over $X$. Like the induction principle for the circle, the induction principle of pushouts has to take both the point constructors and the path constructors of $X$ into account. In our case, the point constructors are the maps
\begin{align*}
  i & : A \to X \\
  j & : B \to X,
  \intertext{and the path constructor is the homotopy}
  H & : \prd{s:S}i(f(s))=j(g(s)).
\end{align*}
Therefore, we obtain for any section $h:\prd{x:X}P(x)$ a triple $(h_A,h_B,h_S)$ consisting of
\begin{align*}
  h_A & : \prd{a:A}P(i(a)) \\
  h_B & : \prd{b:B}P(j(b)) \\
  h_S & : \prd{s:S} \mathsf{tr}_P(H(s),h(i(f(s))))=h(j(g(s))).
\end{align*}
The dependent functions $h_A$ and $h_B$ are simply given by
\begin{align*}
  h_A & \defeq h\circ i \\
  h_B & \defeq h\circ j.
\end{align*}
The homotopy $h_S$ is defined by $h_S(s)\defeq\apd{h}{H(s)}$, using the dependent action on paths of $h$. We call such triples $(h_A,h_B,h_S)$ \define{dependent cocones} on $P$ over the cocone $(i,j,H)$, and will write $\mathsf{dep\usc{}cocone}_{(i,j,H)}(P)$ for this type of dependent cocones. Thus, we have a function
\begin{equation*}
  \mathsf{ev\usc{}pushout}(P):\Big(\prd{x:X}P(x)\Big)\to \mathsf{dep\usc{}cocone}_{(i,j,H)}(P).
\end{equation*}
We are now in position to define the induction principle and the dependent universal property of pushouts.

\begin{defn}
  We say that $X$ satisfies the \define{induction principle of the pushout of $\mathcal{S}$} if the function
  \begin{equation*}
  \mathsf{ev\usc{}pushout}(P):\Big(\prd{x:X}P(x)\Big)\to \mathsf{dep\usc{}cocone}_{(i,j,H)}(P).
  \end{equation*}
  has a section for every type family $P$ over $X$.
\end{defn}

\begin{defn}
  We say that $X$ satisfies the \define{dependent universal property of the pushout of $\mathcal{S}$} if the function
  \begin{equation*}
  \mathsf{ev\usc{}pushout}(P):\Big(\prd{x:X}P(x)\Big)\to \mathsf{dep\usc{}cocone}_{(i,j,H)}(P).
  \end{equation*}
  is an equivalence for every type family $P$ over $X$.
\end{defn}

\begin{rmk}\label{rmk:comp-pushout}
  For $(h_A,h_B,h_S)$ and $(h'_A,h'_B,h'_S)$ in $\mathsf{dep\usc{}cocone}_{(i,j,H)}(P)$, the type of identifications $(h_A,h_B,h_S)=(h'_A,h'_B,h'_S)$ is equivalent to the type of triples $(K_A,K_B,K_S)$ consisting of
  \begin{align*}
    K_A & : \prd{a:A}h_A(a)=h'_A(a) \\
    K_B & : \prd{b:B}h_B(b)=h'_B(b),
  \end{align*}
  and a homotopy $K_S$ witnessing that the square
  \begin{equation*}
    \begin{tikzcd}[column sep=8em]
      \mathsf{tr}_P(H(s),h_A(f(s))) \arrow[r,equals,"\ap{\mathsf{tr}_P(H(s))}{K_A(f(s))}"] \arrow[d,equals,swap,"h_S(s)"] & \mathsf{tr}_P(H(s),{h'_A(f(s))}) \arrow[d,equals,"h'_S(s)"] \\
      h_B(g(s)) \arrow[r,equals,swap,"K_B(g(s))"] & h_B(g(s))
    \end{tikzcd}
  \end{equation*}
  commutes for every $s:S$.

  Therefore we see that the induction principle of the pushout of $\mathcal{S}$ provides us, for every dependent cocone $(h_A,h_B,h_S)$ of $P$ over $(i,j,H)$, with a dependent function $h:\prd{x:A}P(x)$ equipped with homotopies
  \begin{align*}
    K_A & : \prd{a:A}h(i(a))=h_A(a) \\
    K_B & : \prd{b:B}h(j(b))=h_B(b),
  \end{align*}
  and a homotopy $K_S$ witnessing that the square
  \begin{equation*}
    \begin{tikzcd}[column sep=8em]
      \mathsf{tr}_P(H(s),h(i(f(s)))) \arrow[r,equals,"\ap{\mathsf{tr}_P(H(s))}{K_A(f(s))}"] \arrow[d,equals,swap,"\apd{h}{H(s)}"] & \mathsf{tr}_P(H(s),{h_A(f(s))}) \arrow[d,equals,"h_S(s)"] \\
      h(j(g(s))) \arrow[r,equals,swap,"K_B(g(s))"] & h_B(g(s))
    \end{tikzcd}
  \end{equation*}
  commutes for every $s:S$. These homotopies are the \define{computation rules} for pushouts\index{computation rules!for pushouts}. The dependent universal property is equivalent to the assertion that for every dependent cocone $(h_A,h_B,h_S)$, the type of quadruples $(h,K_A,K_B,K_S)$ is contractible.
\end{rmk}

\begin{thm}\label{thm:dependent-pullback-property-pushout}
  Consider a commuting square
  \begin{equation}\label{eq:dppp1}
    \begin{tikzcd}
      S \arrow[d,swap,"f"] \arrow[r,"g"] & B \arrow[d,"j"] \\
      A \arrow[r,swap,"i"] & C
    \end{tikzcd}
  \end{equation}
  with $H:(i\circ f) \htpy (j \circ g)$. Then the following are equivalent:
  \begin{enumerate}
  \item The square in \cref{eq:dppp1} is a pushout square.
  \item The square in \cref{eq:dppp1} satisfies the pullback property of pushouts.
  \item The square satisfies the \define{dependent pullback property} of pushouts: For every family $P$ over $C$, the square
    \begin{equation}\label{eq:dppp2}
      \begin{tikzcd}[column sep=large]
        \prd{z:C}P(z) \arrow[rr,"h\mapsto h\circ j"] \arrow[d,swap,"h\mapsto h\circ i"] & &[1em] \prd{y:B}P(j(y)) \arrow[d,"h\mapsto h\circ g"] \\
        \prd{x:A}P(i(x)) \arrow[r,swap,"{h\mapsto h \circ f}"] & \prd{s:S}P(i(f(s))) \arrow[r,swap,"{\lam{h}{s}\mathsf{tr}_P(H(s),h(s))}" yshift=-2ex] & \prd{s:S}P(j(g(s))),
      \end{tikzcd}
    \end{equation}
    which commutes by the homotopy
    \begin{equation*}
      \lam{h}\mathsf{eq\usc{}htpy}(\lam{s}\apd{h}{H(s)}),
    \end{equation*}
    is a pullback square.
  \item The type $C$ satisfies the \define{dependent universal property} of pushouts.
  \item The type $C$ satisfies the \define{induction principle} of pushouts.
  \end{enumerate}
\end{thm}

\begin{proof}
  We have already seen in \cref{thm:pushout_up} that (i) and (ii) are equivalent.

  To see that (ii) implies (iii), note that we have a commuting cube
  \begin{equation*}
    \begin{tikzcd}[column sep=tiny]
      &  \sm{h:C\to C}\prd{c:C}P(h(c)) \arrow[dl] \arrow[d] \arrow[dr] \\
      \sm{h:A\to C}\prd{a:A}P(h(a)) \arrow[d] & \big(\sm{c:C}P(c)\big)^C \arrow[dl] \arrow[dr] & \sm{h:B\to C}\prd{b:B}P(h(b)) \arrow[d] \arrow[dl,crossing over] \\
      \big(\sm{c:C}P(c)\big)^A \arrow[dr] & \sm{h:S\to C}\prd{s:S}P(h(s)) \arrow[d] \arrow[from=ul,crossing over] & \big(\sm{c:C}P(c)\big)^B \arrow[dl] \\
      & \big(\sm{c:C}P(c)\big)^S
    \end{tikzcd}
  \end{equation*}
  in which the vertical maps are equivalences. Moreover, the bottom square is a pullback square by the pullback property of pushouts, so we conclude that the top square is a pullback square. Since this is a square of total spaces over a pullback square, we invoke \cref{lem:fiberwise-pullback} to conclude that for each $h:C \to C$, the square
  \begin{equation*}
    \begin{tikzcd}
      \prd{c:C}P(h(c)) \arrow[rr] \arrow[d] & &[2em] \prd{b:B} P(h(j(b))) \arrow[d] \\
      \prd{a:A}P(h(i(a))) \arrow[r] & \prd{s:S}P(h(i(f(s)))) \arrow[r,swap,"\mathsf{tr}_{((k:S\to C)\mapsto \prd{s:S}P(k(s)))}(\mathsf{eq\usc{}htpy}(h\cdot H))" yshift=-2ex] & \prd{s:S}P(h(j(g(s))))
    \end{tikzcd}
  \end{equation*}
  is a pullback square. Note that the transport with respect to the family $k\mapsto \prd{s:S}(Pk(s))$ along the identification $\mathsf{eq\usc{}htpy}(h\cdot H)$ is homotopic to the map
  \begin{equation*}
    \lam{h}{s}\mathsf{tr}_{P\circ h}(H(s),h(s)):\prd{s:S}P(h(i(f(s)))) \to \prd{s:S}P(h(j(g(s)))).
  \end{equation*}
  Therefore we conclude that the square
  \begin{equation*}
    \begin{tikzcd}
      \prd{c:C}P(h(c)) \arrow[rr] \arrow[d] & &[2em] \prd{b:B} P(h(j(b))) \arrow[d] \\
      \prd{a:A}P(h(i(a))) \arrow[r] & \prd{s:S}P(h(i(f(s)))) \arrow[r,swap,"{\lam{h}{s}\mathsf{tr}_{P\circ h}(H(s),h(s))}" yshift=-2ex] & \prd{s:S}P(h(j(g(s))))
    \end{tikzcd}
  \end{equation*}
  is a pullback square for each $h:C\to C$. Using the case $h\jdeq\idfunc : C \to C$ we conclude that the cocone $(i,j,H)$ satisfies the dependent pullback property.

  To see that (iii) implies (ii) we recall that transport with respect to a trivial family is homotopic to the identity function. Thus we obtain the pullback property from the dependent pullback property using the trivial family $\lam{c}T$ over $C$.

  To see that (iii) implies (iv) we note that $\mathsf{ev\usc{}pushout}(P)$ is an equivalence if and only if the gap map of the square in \cref{eq:dppp2} is an equivalence.

  It is clear that (iv) implies (v), so it remains to show that (v) implies (iv). If $X$ satisfies the induction principle of pushouts, then the map
  \begin{equation*}
    \mathsf{ev\usc{}pushout}:\Big(\prd{x:X}P(x)\Big)\to\mathsf{dep\usc{}cocone}_{(i,j,H)}(P)
  \end{equation*}
  has a section, i.e., it comes equipped with
  \begin{align*}
    \mathsf{ind\usc{}pushout} & :\mathsf{dep\usc{}cocone}_{(i,j,H)}(P)\to\Big(\prd{x:X}P(x)\Big) \\
    \mathsf{comp\usc{}pushout} & : \mathsf{ev\usc{}pushout}\circ\mathsf{ind\usc{}pushout} \htpy\idfunc.
  \end{align*}
  To see that $\mathsf{ev\usc{}pushout}$ is an equivalence it therefore suffices to construct a homotopy
  \begin{equation*}
    \mathsf{ind\usc{}pushout}(\mathsf{ev\usc{}pushout}(h))\htpy h
  \end{equation*}
  for any $h:\prd{x:X}P(x)$. From the fact that $\mathsf{ind\usc{}pushout}$ is a section of $\mathsf{ev\usc{}pushout}$ we obtain an identification
  \begin{equation*}
    \mathsf{ev\usc{}pushout}(\mathsf{ind\usc{}pushout}(\mathsf{ev\usc{}pushout}(h)))= \mathsf{ev\usc{}pushout}(h).
  \end{equation*}
  Therefore we observe that it suffices to construct a homotopy $h\htpy h'$ for any two functions $h,h':\prd{x:X}P(x)$ that come equipped with an identification
  \begin{equation*}
    \mathsf{ev\usc{}pushout}(h)=\mathsf{ev\usc{}pushout}(h').
  \end{equation*}
  Now we recall from \cref{rmk:comp-pushout} that this type of identifications is equivalent to the type of triples $(K_A,K_B,K_S)$ consisting of
  \begin{align*}
    K_A & : \prd{a:A}h(i(a))=h'(i(a)) \\
    K_B & : \prd{b:B}h(j(b))=h'(j(b))
  \end{align*}
  and a homotopy $K_S$ witnessing that the square
  \begin{equation*}
    \begin{tikzcd}[column sep=8em]
      \mathsf{tr}_P(H(s),h(i(f(s)))) \arrow[r,equals,"\ap{\mathsf{tr}_P(H(s))}{K_A(f(s))}"] \arrow[d,equals,swap,"\apd{h}{H(s)}"] & \mathsf{tr}_P(H(s),{h'(i(f(s)))}) \arrow[d,equals,"\apd{h'}{H(s)}"] \\
      h(j(g(s))) \arrow[r,equals,swap,"K_B(g(s))"] & h'(j(g(s)))
    \end{tikzcd}
  \end{equation*}
  commutes for every $s:S$. Note that from such an identification $K_S(s)$ we also obtain an identification
  \begin{equation*}
    \mathsf{K'_S(s)} : \mathsf{tr}_{x\mapsto h(x)=h'(x)}(H(s),K_A(f(s)))=K_B(g(s)).
  \end{equation*}
  Indeed, by path inducgtion on $p:x=x'$ we obtain an identification $\mathsf{tr}_{x\mapsto h(x)=h'(x)}(p,q)=q'$, for any $p:x=x'$, any $q:h(x)=h'(x)$ and any $q':h(x')=h'(x')$ for which the square
  \begin{equation*}
    \begin{tikzcd}[column sep=huge]
      \mathsf{tr}_P(p,h(x)) \arrow[r,equals,"\ap{\mathsf{tr}_P(p)}{q}"] \arrow[d,equals,swap,"\apd{h}{p}"] & \mathsf{tr}_P(p,h'(x)) \arrow[d,equals,"\apd{h'}{p}"] \\
      h(x') \arrow[r,equals,swap,"{q'}"] & h'(x')
    \end{tikzcd}
  \end{equation*}
  Now we see that the triple $(K_A,K_B,K'_S)$ forms a dependent cocone on the family $x\mapsto h(x)=h'(x)$. Therefore we obtain a homotopy $h\htpy h'$ as an application of the induction principle for pushouts at the family $x\mapsto h(x)=h'(x)$.
\end{proof}

\subsection{Type families over pushouts}

Given a pushout square
\begin{equation*}
\begin{tikzcd}
S \arrow[r,"g"] \arrow[d,swap,"f"] & B \arrow[d,"j"] \\
A \arrow[r,swap,"i"] & X.
\end{tikzcd}
\end{equation*}
with $H:i\circ f\htpy j\circ g$, and a family $P:X\to\UU$, we obtain
\begin{align*}
P\circ i & : A \to \UU \\
P\circ j & : B \to \UU \\
\lam{x}\mathsf{tr}_P(H(x)) & : \prd{x:S} \eqv{P(i(f(x)))}{P(j(g(x)))}.
\end{align*}
Our goal in the current section is to show that the triple $(P_A,P_B,P_S)$ consisting of $P_A\defeq P\circ i$, $P_B\defeq P\circ j$, and $P_S\defeq \lam{x}\mathsf{tr}_P(H(x))$ characterizes the family $P$ over $X$.

\begin{defn}
Consider a commuting square
\begin{equation*}
\begin{tikzcd}
S \arrow[r,"g"] \arrow[d,swap,"f"] & B \arrow[d,"j"] \\
A \arrow[r,swap,"i"] & X.
\end{tikzcd}
\end{equation*}
with $H:i\circ f\htpy j\circ g$, where all types involved are in $\UU$. The type $\mathsf{Desc}(\mathcal{S})$\index{Desc@{$\mathsf{Desc}(\mathcal{S})$}} of \define{descent data}\index{descent data} for $X$, is defined defined to be the type of triples $(P_A,P_B,P_S)$ consisting of
\begin{align*}
P_A & : A \to \UU \\
P_B & : B \to \UU \\
P_S & : \prd{x:S} \eqv{P_A(f(x))}{P_B(g(x))}.
\end{align*}
\end{defn}

\begin{defn}
Given a commuting square
\begin{equation*}
\begin{tikzcd}
S \arrow[r,"g"] \arrow[d,swap,"f"] & B \arrow[d,"j"] \\
A \arrow[r,swap,"i"] & X.
\end{tikzcd}
\end{equation*}
with $H:i\circ f\htpy j\circ g$, we define the map\index{desc_fam@{$\mathsf{desc\usc{}fam}_{\mathcal{S}}$}}
\begin{equation*}
\mathsf{desc\usc{}fam}_{\mathcal{S}}(i,j,H) : (X\to \UU)\to \mathsf{Desc}(\mathcal{S})
\end{equation*}
by $P\mapsto (P\circ i,P\circ j,\lam{x}\mathsf{tr}_P(H(x)))$.
\end{defn}

\begin{thm}\label{thm:desc_fam}
Consider a pushout square
\begin{equation*}
\begin{tikzcd}
S \arrow[r,"g"] \arrow[d,swap,"f"] & B \arrow[d,"j"] \\
A \arrow[r,swap,"i"] & X.
\end{tikzcd}
\end{equation*}
with $H:i\circ f\htpy j\circ g$, where all types involved are in $\UU$, and suppose we have
\begin{align*}
P_A & : A \to \UU \\
P_B & : B \to \UU \\
P_S & : \prd{x:S} \eqv{P_A(f(x))}{P_B(g(x))}.
\end{align*}
Then the function\index{desc_fam@{$\mathsf{desc\usc{}fam}_{\mathcal{S}}$}!is an equivalence}
\begin{equation*}
\mathsf{desc\usc{}fam}_{\mathcal{S}}(i,j,H) : (X\to \UU)\to \mathsf{Desc}(\mathcal{S})
\end{equation*}
is an equivalence.
\end{thm}

\begin{proof}
By the 3-for-2 property of equivalences it suffices to construct an equivalence $\varphi:\mathsf{cocone}_{\mathcal{S}}(\UU)\to\mathsf{Desc}(\mathcal{S})$ such that the triangle
\begin{equation*}
\begin{tikzcd}[column sep=tiny]
& \UU^X \arrow[dl,swap,"{\mathsf{cocone\usc{}map}_{\mathcal{S}}(i,j,H)}"] \arrow[dr,"{\mathsf{desc\usc{}fam}_{\mathcal{S}}(i,j,H)}"] & \phantom{\mathsf{cocone}_{\mathcal{S}}(\UU)} \\
\mathsf{cocone}_{\mathcal{S}}(\UU) \arrow[rr,densely dotted,"\eqvsym","\varphi"'] & & \mathsf{Desc}(\mathcal{S})
\end{tikzcd}
\end{equation*}
commutes.

Since we have equivalences
\begin{equation*}
\mathsf{equiv\usc{}eq}:\eqv{\Big(P_A(f(x))=P_B(g(x))\Big)}{\Big(\eqv{P_A(f(x))}{P_B(g(x))}\Big)}
\end{equation*}
for all $x:S$, we obtain by \cref{ex:equiv_pi} an equivalence on the dependent products
\begin{equation*}
{\Big(\prd{x:S}P_A(f(x))=P_B(g(x))\Big)}\to{\Big(\prd{x:S}\eqv{P_A(f(x))}{P_B(g(x))}\Big)}.
\end{equation*}
We define $\varphi$ to be the induced map on total spaces. Explicitly, we have
\begin{equation*}
\varphi\defeq \lam{(P_A,P_B,K)}(P_A,P_B,\lam{x}\mathsf{equiv\usc{}eq}(K(x))).
\end{equation*}
Then $\varphi$ is an equivalence by \cref{thm:fib_equiv}, and the triangle commutes by \cref{ex:tr_ap}.
\end{proof}

\begin{cor}\label{cor:desc_fam}
Consider descent data $(P_A,P_B,P_S)$ for a pushout square as in \cref{thm:desc_fam}.
Then the type of quadruples $(P,e_A,e_B,e_S)$ consisting of a family $P:X\to\UU$ equipped with two families of equivalences
\begin{samepage}
\begin{align*}
e_A & : \prd{a:A}\eqv{P_A(a)}{P(i(a))} \\
e_B & : \prd{b:B}\eqv{P_B(a)}{P(j(b))}
\end{align*}
\end{samepage}%
and a homotopy $e_S$ witnessing that the square
\begin{equation*}
\begin{tikzcd}[column sep=huge]
P_A(f(x)) \arrow[r,"e_A(f(x))"] \arrow[d,swap,"P_S(x)"] & P(i(f(x))) \arrow[d,"\mathsf{tr}_P(H(x))"] \\
P_B(g(x)) \arrow[r,swap,"e_B(g(x))"] & P(j(g(x)))
\end{tikzcd}
\end{equation*}
commutes, is contractible.
\end{cor}

\begin{proof}
The fiber of this map at $(P_A,P_B,P_S)$ is equivalent to the type of quadruples $(P,e_A,e_B,e_S)$ as described in the theorem, which are contractible by \cref{thm:contr_equiv}.
\end{proof}

\subsection{The flattening lemma for pushouts}

In this section we consider a pushout square
\begin{equation*}
\begin{tikzcd}
S \arrow[r,"g"] \arrow[d,swap,"f"] & B \arrow[d,"j"] \\
A \arrow[r,swap,"i"] & X.
\end{tikzcd}
\end{equation*}
with $H:i\circ f\htpy j\circ g$, descent data
\begin{align*}
P_A & : A \to \UU \\
P_B & : B \to \UU \\
P_S & : \prd{x:S} \eqv{P_A(f(x))}{P_B(g(x))},
\end{align*}
and a family $P:X\to\UU$ equipped with 
\begin{align*}
e_A & : \prd{a:A}\eqv{P_A(a)}{P(i(a))} \\
e_B & : \prd{b:B}\eqv{P_B(a)}{P(j(b))}
\end{align*}
and a homotopy $e_S$ witnessing that the square
\begin{equation*}
\begin{tikzcd}[column sep=huge]
P_A(f(x)) \arrow[r,"e_A(f(x))"] \arrow[d,swap,"P_S(x)"] & P(i(f(x))) \arrow[d,"\mathsf{tr}_P(H(x))"] \\
P_B(g(x)) \arrow[r,swap,"e_B(g(x))"] & P(j(g(x)))
\end{tikzcd}
\end{equation*}
commutes.

\begin{defn}
We define a commuting square
\begin{equation*}
\begin{tikzcd}
\sm{x:S}P_A(f(x)) \arrow[d,swap,"{f'}"] \arrow[r,"{g'}"] & \sm{b:B}P_B(b) \arrow[d,"{j'}"] \\
\sm{a:A}P_A(a) \arrow[r,swap,"{i'}"] & \sm{x:X}P(x)
\end{tikzcd}
\end{equation*}
with a homotopy $H':i'\circ f'\htpy j'\circ g'$. We will write $\mathcal{S'}$ for the span
\begin{equation*}
\begin{tikzcd}
\sm{a:A}P_A(a) & \sm{x:S}P_A(f(x)) \arrow[l,swap,"{f'}"] \arrow[r,"{g'}"] & \sm{b:B}P_B(b).
\end{tikzcd}
\end{equation*}
\end{defn}

\begin{constr}
We define
\begin{align*}
f' & \defeq \tot[f]{\lam{x}\idfunc[P_A(f(x))]} \\
g' & \defeq \tot[g]{e_S} \\
i' & \defeq \tot[i]{e_A} \\
j' & \defeq \tot[j]{e_B}.
\end{align*}
Then it remains to construct a homotopy $H':i'\circ f'\htpy j'\circ g'$. In order to construct this homotopy, we have to construct an identification
\begin{equation*}
(i(f(x)),e_A(y))=(j(g(x)),e_B(e_S(y)))
\end{equation*}
for any $x:S$ and $y:P_A(f(x))$. Note that have the identification
\begin{equation*}
\mathsf{eq\usc{}pair}(H(x),e_S(x,y)^{-1})
\end{equation*}
of this type.
\end{constr}

\begin{lem}[The flattening lemma]\label{lem:flattening}
The commuting square\index{flattening lemma!for pushouts}
\begin{equation*}
\begin{tikzcd}
\sm{x:S}P_A(f(x)) \arrow[d,swap,"{f'}"] \arrow[r,"{g'}"] & \sm{b:B}P_B(b) \arrow[d,"{j'}"] \\
\sm{a:A}P_A(a) \arrow[r,swap,"{i'}"] & \sm{x:X}P(x)
\end{tikzcd}
\end{equation*}
is a pushout square.
\end{lem}

\begin{proof}
  To show that the square of total spaces satisfies the pullback property of pullbacks, note that we have a commuting cube
  \begin{equation*}
    \begin{tikzcd}
      & T^{\sm{x:X}P(x)} \arrow[dl] \arrow[d] \arrow[dr] \\
      T^{\sm{a:A}P_A(a)} \arrow[d] & \prd{x:X}T^{P(x)} \arrow[dl] \arrow[dr] & T^{\sm{b:B}P_B(b)} \arrow[dl,crossing over] \arrow[d] \\
      \prd{a:A}T^{P_A(a)} \arrow[dr] & T^{\sm{x:S}P_A(f(x))} \arrow[from=ul,crossing over] \arrow[d] & \prd{b:B}T^{P_B(b)} \arrow[dl] \\
      & \prd{x:S}T^{P_A(f(x))}
    \end{tikzcd}
  \end{equation*}
  for any type $T$. In this cube, the vertical maps are all equivalences, and the bottom square is a pullback square by the dependent pullback property of pushouts. Therefore it follows that the top square is a pullback square.
\end{proof}

\subsection{The universality theorem}
\begin{thm}\label{thm:descent}
  Consider two pushout squares
  \begin{equation*}
    \begin{tikzcd}
      S' \arrow[r] \arrow[d] & B' \arrow[d] & S \arrow[r] \arrow[d] & B \arrow[d] \\
      A' \arrow[r] & C' & A \arrow[r] & C
    \end{tikzcd}
  \end{equation*}
  and a commuting cube
  \begin{equation*}
    \begin{tikzcd}
      & S' \arrow[dl] \arrow[d] \arrow[dr] \\
      A' \arrow[d] & S \arrow[dl] \arrow[dr] & B' \arrow[d] \arrow[dl,crossing over] \\
      A \arrow[dr] & D' \arrow[from=ul,crossing over] \arrow[d] & B \arrow[dl] \\
      & D
    \end{tikzcd}
  \end{equation*}
  in which the back left and right squares are pullback squares. The following are equivalent:
  \begin{enumerate}
  \item The front left and right squares are pullback squares.
  \item The induced commuting square
    \begin{equation*}
      \begin{tikzcd}
        C' \arrow[r,densely dotted] \arrow[d,densely dotted] & D' \arrow[d] \\
        C \arrow[r,densely dotted] & D
      \end{tikzcd}
    \end{equation*}
    is a pullback square.
  \end{enumerate}
\end{thm}


\subsection{The descent property for pushouts}

In the previous section there was a significant role for families of equivalences, and we know by \cref{thm:pb_fibequiv,cor:pb_fibequiv}: families of equivalences indicate the presence of pullbacks. In this section we reformulate the results of the previous section using pullbacks where we used families of equivalences before, to obtain new and useful results. We begin by considering the type of descent data from the perspective of pullback squares.

\begin{defn}
Consider a span $\mathcal{S}$ from $A$ to $B$, and a span $\mathcal{S}'$ from $A'$ to $B'$. A \define{cartesian transformation} of spans\index{cartesian transformation!of spans} from $\mathcal{S}'$ to $\mathcal{S}$ is a diagram of the form
\begin{equation*}
\begin{tikzcd}
A' \arrow[d,swap,"h_A"]  & S' \arrow[l,swap,"{f'}"] \arrow[r,"{g'}"] \arrow[d,swap,"h_S"] & B' \arrow[d,"h_B"] \\
A & S \arrow[l,"f"] \arrow[r,swap,"g"] & B
\end{tikzcd}
\end{equation*}
with $F:f\circ h_S\htpy h_A\circ f'$ and $G:g\circ h_S\htpy h_B\circ g'$, where both squares are pullback squares. 

The type $\mathsf{cart}(\mathcal{S}',\mathcal{S})$\index{cart(S,S')@{$\mathsf{cart}(\mathcal{S},\mathcal{S}')$}} of cartesian transformation is the type of tuples
\begin{equation*}
(h_A,h_S,h_B,F,G,p_f,p_g)
\end{equation*}
where $p_f:\mathsf{is\usc{}pullback}(h_S,h_A,F)$ and $p_g:\mathsf{is\usc{}pullback}(h_S,h_B,G)$, and we write
\begin{equation*}
\mathsf{Cart}(\mathcal{S}) \defeq \sm{A',B':\UU}{\mathcal{S}':\mathsf{span}(A',B')}\mathsf{cart}(\mathcal{S}',\mathcal{S}).
\end{equation*}
\end{defn}

\begin{lem}\label{lem:cart_desc}
There is an equivalence\index{cart_desc@{$\mathsf{cart\usc{}desc}_{\mathcal{S}}$}}
\begin{equation*}
\mathsf{cart\usc{}desc}_{\mathcal{S}}:\mathsf{Desc}(\mathcal{S})\to \mathsf{Cart}(\mathcal{S}).
\end{equation*}
\end{lem}

\begin{proof}
Note that by \cref{thm:pb_fibequiv_complete} it follows that the types of triples $(f',F,p_f)$ and $(g',G,p_g)$ are equivalent to the types of families of equivalences
\begin{align*}
& \prd{x:S}\eqv{\fib{h_S}{x}}{\fib{h_A}{f(x)}} \\
& \prd{x:S}\eqv{\fib{h_S}{x}}{\fib{h_B}{g(x)}}
\end{align*} 
respectively. Furthermore, by \cref{thm:fam_proj} the types of pairs $(S',h_S)$, $(A',h_A)$, and $(B',h_B)$ are equivalent to the types $S\to \UU$, $A\to \UU$, and $B\to \UU$, respectively. Therefore it follows that the type $\mathsf{Cart}(\mathcal{S})$ is equivalent to the type of tuples $(Q,P_A,\varphi,P_B,P_S)$ consisting of
\begin{align*}
Q & : S\to \UU \\
P_A & : A \to \UU \\
P_B & : B \to \UU \\
\varphi & : \prd{x:S}\eqv{Q(x)}{P_A(f(x))} \\
P_S & : \prd{x:S}\eqv{Q(x)}{P_B(g(x))}.
\end{align*}
However, the type of $\varphi$ is equivalent to the type $P_A\circ f=Q$. Thus we see that the type of pairs $(Q,\varphi)$ is contractible, so our claim follows.
\end{proof}

\begin{defn}
We define an operation\index{cart map!{$\mathsf{cart\usc{}map}_{\mathcal{S}}$}}
\begin{equation*}
\mathsf{cart\usc{}map}_{\mathcal{S}}:{\Big(\sm{X':\UU}X'\to X\Big)}\to \mathsf{Cart}(\mathcal{S}).
\end{equation*}
\end{defn}

\begin{constr}
Let $X':\UU$ and $h_X:X'\to X$. Then we define the types
\begin{align*}
A' & \defeq A\times_X X' \\
B' & \defeq B\times_X X'.
\end{align*}
Next, we define a span $\mathcal{S'}\defeq(S',f',g')$ from $A'$ to $B'$. We take
\begin{align*}
S' & \defeq S\times_A A' \\
f' & \defeq \pi_2.
\end{align*}
To define $g'$, let $s:S$, let $(a,x',p):A\times_X X'$, and let $q:f(s)=a$. Our goal is to construct a term of type $B\times_X X'$. We have $g(s):B$ and $x':X'$, so it remains to show that $j(g(s))=h_X(x')$. We construct such an identification as a concatenation
\begin{equation*}
\begin{tikzcd}
j(g(s)) \arrow[r,equals,"H(s)^{-1}"] &[1ex] i(f(s)) \arrow[r,equals,"\ap{i}{q}"] &[1ex] i(a) \arrow[r,equals,"p"] & h_X(x').
\end{tikzcd}
\end{equation*}
To summarize, the map $g'$ is defined as
\begin{equation*}
g' \defeq \lam{(s,(a,x',p),q)}(g(s),x',\ct{H(s)^{-1}}{(\ct{\ap{i}{q}}{p})}).
\end{equation*}
Then we have commuting squares
\begin{equation*}
\begin{tikzcd}
A\times_X X' \arrow[d] & S\times_A A' \arrow[d] \arrow[l] \arrow[r] & B\times_X X' \arrow[d] \\
A & S \arrow[l] \arrow[r] & B.
\end{tikzcd}
\end{equation*}
Moreover, these squares are pullback squares by \cref{thm:pb_pasting}.
\end{constr}

The following theorem is analogous to \cref{thm:desc_fam}.

\begin{thm}[The descent theorem for pushouts]\label{thm:cart_map}\index{descent theorem!for pushouts}
The operation $\mathsf{cart\usc{}map}_{\mathcal{S}}$\index{cart map!{$\mathsf{cart\usc{}map}_{\mathcal{S}}$}!is an equivalence} is an equivalence
\begin{equation*}
\eqv{\Big(\sm{X':\UU}X'\to X\Big)}{\mathsf{Cart}(\mathcal{S})}
\end{equation*}
\end{thm}

\begin{proof}
It suffices to show that the square
\begin{equation*}
\begin{tikzcd}[column sep=huge]
X\to \UU \arrow[r,"{\mathsf{desc\usc{}fam}_{\mathcal{S}}(i,j,H)}"] \arrow[d,swap,"\mathsf{map\usc{}fam}_X"] & \mathsf{Desc}(\mathcal{S}) \arrow[d,"\mathsf{cart\usc{}desc}_{\mathcal{S}}"] \\
\sm{X':\UU}X'\to X \arrow[r,swap,"\mathsf{cart\usc{}map}_{\mathcal{S}}"] & \mathsf{Cart}(\mathcal{S})
\end{tikzcd}
\end{equation*}
commutes. To see that this suffices, note that the operation $\mathsf{map\usc{}fam}_X$ is an equivalence by \cref{thm:fam_proj}, the operation $\mathsf{desc\usc{}fam}_{\mathcal{S}}(i,j,H)$ is an equivalence by \cref{thm:desc_fam}, and the operation $\mathsf{cart\usc{}desc}_{\mathcal{S}}$ is an equivalence by \cref{lem:cart_desc}.

To see that the square commutes, note that the composite
\begin{equation*}
\mathsf{cart\usc{}map}_{\mathcal{S}}\circ \mathsf{map\usc{}fam}_X
\end{equation*}
takes a family $P:X\to \UU$ to the cartesian transformation of spans
\begin{equation*}
\begin{tikzcd}
A\times_X\tilde{P} \arrow[d,swap,"\pi_1"] & S\times_A\Big(A\times_X\tilde{P}\Big) \arrow[l] \arrow[r] \arrow[d,swap,"\pi_1"] & B\times_X\tilde{P} \arrow[d,"\pi_1"] \\
A & S \arrow[l] \arrow[r] & B,
\end{tikzcd}
\end{equation*}
where $\tilde{P}\defeq\sm{x:X}P(x)$.

The composite 
\begin{equation*}
\mathsf{cart\usc{}desc}_{\mathcal{S}}\circ \mathsf{desc\usc{}fam}_X
\end{equation*}
takes a family $P:X\to \UU$ to the cartesian transformation of spans
\begin{equation*}
\begin{tikzcd}
\sm{a:A}P(i(a)) \arrow[d] & \sm{s:S}P(i(f(s))) \arrow[l] \arrow[r] \arrow[d] & \sm{b:B}P(j(b)) \arrow[d] \\
A & S \arrow[l] \arrow[r] & B
\end{tikzcd}
\end{equation*}
These cartesian natural transformations are equal by \cref{lem:pb_subst}
\end{proof}

Since $\mathsf{cart\usc{}map}_{\mathcal{S}}$ is an equivalence it follows that its fibers are contractible. This is essentially the content of the following corollary.

\begin{cor}
Consider a diagram of the form 
\begin{equation*}
\begin{tikzcd}
& S' \arrow[d,swap,"h_S"] \arrow[dl,swap,"{f'}"] \arrow[dr,"{g'}"] \\
A' \arrow[d,swap,"h_A"] & S \arrow[dl,swap,"f"] \arrow[dr,"g"] & B' \arrow[d,"{h_B}"] \\
A \arrow[dr,swap,"i"] & & B \arrow[dl,"j"] \\
& X
\end{tikzcd}
\end{equation*}
with homotopies
\begin{align*}
F & : f\circ h_S \htpy h_A\circ f' \\
G & : g\circ h_S \htpy h_B\circ g' \\
H & : i\circ f \htpy j\circ g,
\end{align*}
and suppose that the bottom square is a pushout square, and the top squares are pullback squares.
Then the type of tuples $((X',h_X),(i',I,p),(j',J,q),(H',C))$ consisting of
\begin{enumerate}
\item A type $X':\UU$ together with a morphism
\begin{equation*}
h_X : X'\to X,
\end{equation*}
\item A map $i':A'\to X'$, a homotopy $I:i\circ h_A\htpy h_X\circ i'$, and a term $p$ witnessing that the square
\begin{equation*}
\begin{tikzcd}
A' \arrow[d,swap,"h_A"] \arrow[r,"{i'}"] & X' \arrow[d,"h_X"] \\
A \arrow[r,swap,"i"] & X
\end{tikzcd}
\end{equation*}
is a pullback square.
\item A map $j':B'\to X'$, a homotopy $J:j\circ h_B\htpy h_X\circ j'$, and a term $q$ witnessing that the square
\begin{equation*}
\begin{tikzcd}
B' \arrow[d,swap,"h_B"] \arrow[r,"{j'}"] & X' \arrow[d,"h_X"] \\
B \arrow[r,swap,"j"] & X
\end{tikzcd}
\end{equation*}
is a pullback square,
\item A homotopy $H':i'\circ f'\htpy j'\circ g'$, and a homotopy
\begin{equation*}
C : \ct{(i\cdot F)}{(\ct{(I\cdot f')}{(h_X\cdot H')})} \htpy \ct{(H\cdot h_S)}{(\ct{(j\cdot G)}{(J\cdot g')})}
\end{equation*}
witnessing that the cube
\begin{equation*}
\begin{tikzcd}
& S' \arrow[dl] \arrow[dr] \arrow[d] \\
A' \arrow[d] & S \arrow[dl] \arrow[dr] & B' \arrow[dl,crossing over] \arrow[d] \\
A \arrow[dr] & X' \arrow[d] \arrow[from=ul,crossing over] & B \arrow[dl] \\
& X,
\end{tikzcd}
\end{equation*}
commutes,
\end{enumerate}
is contractible.
\end{cor}

The following theorem should be compared to the flattening lemma, \cref{lem:flattening}.\index{flattening lemma!for pushouts}

\begin{thm}
Consider a commuting cube
\begin{equation*}
\begin{tikzcd}
& S' \arrow[dl,swap,"{f'}"] \arrow[dr,"{g'}"] \arrow[d,"h_S"] \\
A' \arrow[d,swap,"h_A"] & S \arrow[dl,swap,"f" near start] \arrow[dr,"g" near start] & B' \arrow[dl,crossing over,"{j'}" near end] \arrow[d,"h_B"] \\
A \arrow[dr,swap,"i"] & X' \arrow[d,"h_X" near start] \arrow[from=ul,crossing over,"{i'}"' near end] & B \arrow[dl,"j"] \\
& X.
\end{tikzcd}
\end{equation*}
If each of the vertical squares is a pullback, and the bottom square  is a pushout, then the top square is a pushout.
\end{thm}

\begin{proof}
By \cref{cor:pb_fibequiv} we have families of equivalences
\begin{align*}
F & : \prd{x:S}\eqv{\fib{h_S}{x}}{\fib{h_A}{f(x)}} \\
G & : \prd{x:S}\eqv{\fib{h_S}{x}}{\fib{h_B}{g(x)}} \\
I & : \prd{a:A}\eqv{\fib{h_A}{a}}{\fib{h_X}{i(a)}} \\
J & : \prd{b:B}\eqv{\fib{h_B}{b}}{\fib{h_X}{j(b)}}. 
\end{align*}
Moreover, since the cube commutes we obtain a family of homotopies
\begin{equation*}
K : \prd{x:S} I(f(x))\circ F(x) \htpy J(g(x))\circ G(x).
\end{equation*}
We define the descent data $(P_A,P_B,P_S)$ consisting of $P_A:A\to\UU$, $P_B:B\to\UU$, and $P_S:\prd{x:S}\eqv{P_A(f(x))}{P_B(g(x))}$ by
\begin{align*}
P_A(a) & \defeq \fib{h_A}{a} \\
P_B(b) & \defeq \fib{h_B}{b} \\
P_S(x) & \defeq G(x)\circ F(x)^{-1}.
\end{align*}
We have
\begin{align*}
P & \defeq \fibf{h_X} \\
e_A & \defeq I \\
e_B & \defeq J \\
e_S & \defeq K.
\end{align*}
Now consider the diagram
\begin{equation*}
\begin{tikzcd}
\sm{s:S}\fib{h_S}{s} \arrow[r] \arrow[d] & \sm{s:S}\fib{h_A}{f(s)} \arrow[r] \arrow[d] & \sm{b:B}\fib{h_B}{b} \arrow[d] \\
\sm{a:A}\fib{h_A}{a} \arrow[r] & \sm{a:A}\fib{h_A}{a} \arrow[r] & \sm{x:X}\fib{h_X}{x}
\end{tikzcd}
\end{equation*}
Since the top and bottom map in the left square are equivalences, we obtain from \cref{ex:pushout_equiv} that the left square is a pushout square. Moreover, the right square is a pushout by \cref{lem:flattening}. Therefore it follows by \cref{thm:pushout_pasting} that the outer rectangle is a pushout square.

Now consider the commuting cube
\begin{equation*}
\begin{tikzcd}
& \sm{s:S}\fib{h_S}{s} \arrow[dl] \arrow[dr] \arrow[d] \\
\sm{a:A}\fib{h_A}{a} \arrow[d] & S' \arrow[dl] \arrow[dr] & \sm{b:B}\fib{h_B}{b} \arrow[dl,crossing over] \arrow[d] \\
A' \arrow[dr,swap] & \sm{x:X}\fib{h_X}{x} \arrow[d] \arrow[from=ul,crossing over] & B' \arrow[dl] \\
& X'.
\end{tikzcd}
\end{equation*}
We have seen that the top square is a pushout. The vertical maps are all equivalences, so the vertical squares are all pushout squares. Thus it follows from one more application of \cref{thm:pushout_pasting} that the bottom square is a pushout.
\end{proof}

%\begin{cor}
%For any map $f:A\sqcup^S B\to X$, and any $x:X$, the square
%\begin{equation*}
%\begin{tikzcd}
%\fib{f_S}{x} \arrow[r] \arrow[d] & \fib{f_B}{x} \arrow[d] \\
%\fib{f_A}{x} \arrow[r] & \fib{f}{x}
%\end{tikzcd}
%\end{equation*}
%is a pushout square.
%\end{cor}

\begin{thm}\label{thm:effectiveness-pullback}
Consider a commuting cube of types 
\begin{equation*}\label{eq:cube}
\begin{tikzcd}
& S' \arrow[dl] \arrow[dr] \arrow[d] \\
A' \arrow[d] & S \arrow[dl] \arrow[dr] & B' \arrow[dl,crossing over] \arrow[d] \\
A \arrow[dr] & X' \arrow[d] \arrow[from=ul,crossing over] & B \arrow[dl] \\
& X,
\end{tikzcd}
\end{equation*}
and suppose the vertical squares are pullback squares. Then the commuting square
\begin{equation*}
\begin{tikzcd}
A' \sqcup^{S'} B' \arrow[r] \arrow[d] & X' \arrow[d] \\
A\sqcup^{S} B \arrow[r] & X
\end{tikzcd}
\end{equation*}
is a pullback square.
\end{thm}

\begin{proof}
It suffices to show that the pullback 
\begin{equation*}
(A\sqcup^{S} B)\times_{X}X'
\end{equation*}
has the universal property of the pushout. This follows by the descent theorem, since the vertical squares in the cube
\begin{equation*}
\begin{tikzcd}
& S' \arrow[dl] \arrow[dr] \arrow[d] \\
A' \arrow[d] & S \arrow[dl] \arrow[dr] & B' \arrow[dl,crossing over] \arrow[d] \\
A \arrow[dr] & (A\sqcup^{S} B)\times_{X}X' \arrow[d] \arrow[from=ul,crossing over] & B \arrow[dl] \\
& A\sqcup^{S} B
\end{tikzcd}
\end{equation*}
are pullback squares by \cref{thm:pb_pasting}.
\end{proof}

\subsection{Applications of the descent theorem}

\begin{thm}
  Consider a commuting cube
  \begin{equation*}
    \begin{tikzcd}
      & S' \arrow[dl] \arrow[dr] \arrow[d] \\
      A' \arrow[d] & S \arrow[dl] \arrow[dr] & B' \arrow[dl,crossing over] \arrow[d] \\
      A \arrow[dr] & C' \arrow[from=ul,crossing over] \arrow[d] & B \arrow[dl] \\
      & C
    \end{tikzcd}
  \end{equation*}
  in which the bottom square is a pushout square. If the vertical sides are pullback squares, then for each $c:C$ the square of fibers
  \begin{equation*}
    \begin{tikzcd}
      \fib{i\circ f\circ h_S}{c} \arrow[d] \arrow[r] & \fib{j\circ g\circ h_S}{c} \arrow[r] & \fib{j\circ h_B}{c} \arrow[d] \\
      \fib{i\circ h_A}{c} \arrow[rr] & & \fib{h_C}{c}
    \end{tikzcd}
  \end{equation*}
  is a pushout square.
\end{thm}

\begin{exercises}
\exercise Use the characterization of the circle\index{circle} as a pushout given in \cref{eg:circle_pushout} to show that the square
\begin{equation*}
\begin{tikzcd}[column sep=large]
\sphere{1}+\sphere{1} \arrow[r,"{[\idfunc,\idfunc]}"] \arrow[d,swap,"{[\idfunc,\idfunc]}"] & \sphere{1} \arrow[d,"{\lam{t}(t,\base)}"] \\
\sphere{1} \arrow[r,swap,"{\lam{t}(t,\base)}"] & \sphere{1}\times\sphere{1}
\end{tikzcd}
\end{equation*}
is a pushout square.
\exercise Let $f:A\to B$ be a map. The \define{codiagonal}\index{codiagonal}\index{nabla@{$\nabla_f$}} $\nabla_f$ of $f$ is the map obtained from the universal property of the pushout, as indicated in the diagram
\begin{equation*}
\begin{tikzcd}
A \arrow[d,swap,"f"] \arrow[r,"f"] \arrow[dr, phantom, "\ulcorner", very near end] & B \arrow[d,"\inr"] \arrow[ddr,bend left=15,"{\idfunc[B]}"] \\
A \arrow[r,"\inl"] \arrow[drr,bend right=15,swap,"{\idfunc[B]}"] & B\sqcup^{A} B \arrow[dr,densely dotted,near start,swap,"\nabla_f"] \\
& & B
\end{tikzcd}
\end{equation*}
Show that $\fib{\nabla_f}{b}\eqvsym \susp(\fib{f}{b})$ for any $b:B$.
\exercise \label{ex:fib_join}Consider two maps $f:A\to X$ and $g:B\to X$. The \define{fiberwise join}\index{fiberwise join} $\join{f}{g}$ is defined by the universal property of the pushout as the unique map rendering the diagram
\begin{equation*}
\begin{tikzcd}
A\times_X B \arrow[d,"\pi_1"] \arrow[r,"\pi_2"] \arrow[dr, phantom, "\ulcorner", very near end] & B \arrow[d,"\inr"] \arrow[ddr,bend left=15,"g"] \\
A \arrow[r,"\inl"] \arrow[drr,bend right=15,swap,"f"] & \join[X]{A}{B} \arrow[dr,densely dotted,near start,swap,"\join{f}{g}"] \\
& & X
\end{tikzcd}
\end{equation*}
commutative, where $\join[X]{A}{B}$ is defined as a pushout, as indicated.
Construct an equivalence
\begin{equation*}
\eqv{\fib{\join{f}{g}}{x}}{\join{\fib{f}{x}}{\fib{g}{x}}}
\end{equation*}
for any $x:X$. 
\exercise Consider two maps $f:A\to B$ and $g:C\to D$.
The \define{pushout-product}\index{pushout-product}
\begin{equation*}
f\square g : (A\times D)\sqcup^{A\times C} (B\times C)\to B\times D
\end{equation*}
of $f$ and $g$ is defined by the universal property of the pushout as the unique map rendering the diagram
\begin{equation*}
\begin{tikzcd}
A\times C \arrow[r,"{f\times \idfunc[C]}"] \arrow[d,swap,"{\idfunc[A]\times g}"] & B\times C \arrow[d,"\inr"] \arrow[ddr,bend left=15,"{\idfunc[B]\times g}"] \\
A\times D \arrow[r,"\inl"] \arrow[drr,bend right=15,swap,"{f\times\idfunc[D]}"] & (A\times D)\sqcup^{A\times C} (B\times C) \arrow[dr,densely dotted,swap,near start,"f\square g"] \\
& & B\times D
\end{tikzcd}
\end{equation*}
commutative. Construct an equivalence
\begin{equation*}
\eqv{\fib{f\square g}{b,d}}{\join{\fib{f}{b}}{\fib{g}{d}}}
\end{equation*}
for all $b:B$ and $d:D$.
\exercise Let $A$ and $B$ be pointed types with base points $a_0:A$ and $b_0:B$. The \define{wedge inclusion}\index{wedge inclusion} is defined as follows by the universal property of the wedge:
\begin{equation*}
\begin{tikzcd}[column sep=huge]
\unit \arrow[r] \arrow[d] & B \arrow[d,"\inr"] \arrow[ddr,bend left=15,"{\lam{b}(a_0,b)}"] \\
A \arrow[r,"\inl"] \arrow[drr,bend right=15,swap,"{\lam{a}(a,b_0)}"] & A\vee B \arrow[dr,densely dotted,swap,"{\mathsf{wedge\usc{}in}_{A,B}}"{near start,xshift=1ex}] \\
& & A\times B
\end{tikzcd}
\end{equation*}
Show that the fiber of the wedge inclusion $A\vee B\to A\times B$ is equivalent to $\join{\loopspace{B}}{\loopspace{A}}$.
\exercise Let $f:X\vee X\to X$ be the map defined by the universal property of the wedge as indicated in the diagram
\begin{equation*}
\begin{tikzcd}
\unit \arrow[d,swap,"x_0"] \arrow[r,"x_0"] \arrow[dr, phantom, "\ulcorner", very near end] & X \arrow[d,"\inr"] \arrow[ddr,bend left=15,"{\idfunc[X]}"] \\
X \arrow[r,"\inl"] \arrow[drr,bend right=15,swap,"{\idfunc[X]}"] & X\vee X \arrow[dr,densely dotted,near start,swap,"f"] \\
& & X.
\end{tikzcd}
\end{equation*}
\begin{subexenum}
\item Show that $\eqv{\fib{f}{x_0}}{\susp\loopspace{X}}$. 
\item Show that $\eqv{\mathsf{cof}_f}{\susp X}$.
\end{subexenum}
\exercise Consider a pushout square
\begin{equation*}
\begin{tikzcd}
S \arrow[r,"g"] \arrow[d,swap,"f"] & B \arrow[d,"j"] \\
A \arrow[r,swap,"i"] & X,
\end{tikzcd}
\end{equation*}
and suppose that $f$ is an embedding. Show that $j$ is an embedding, and that the square is also a pullback square.
  \exercise Consider a pushout square
  \begin{equation*}
    \begin{tikzcd}
      S \arrow[d,swap,"f"] \arrow[r,"g"] & B \arrow[d,"j"] \\
      A \arrow[r,swap,"i"] & X.
    \end{tikzcd}
  \end{equation*}
  \begin{subexenum}
  \item Show that if $f$ is surjective, then so is $j$.
  \item Show that the two small squares in the diagram 
    \begin{equation*}
      \begin{tikzcd}
        S \arrow[r,"g"] \arrow[d,swap,"q_f"] & B \arrow[d,"q_j"] \\
        \im(f) \arrow[r,densely dotted] \arrow[d,swap,"i_f"] & \im(j) \arrow[d,"i_j"] \\
        A \arrow[r,swap,"i"] & X
      \end{tikzcd}
    \end{equation*}
    are both pushout squares, and that the bottom square is also a pullback square.
  \end{subexenum}
\end{exercises}

\input{id-pushout}
\input{projective}
\input{sequences}


\chapter{Synthetic homotopy theory}
\input{homotopy-groups}
\input{hopf}
\input{truncation}
\input{connected}
\input{blakers-massey}
\input{higher-group-theory}

\backmatter

\appendix
\chapter{Overview of the axioms in this book}

\begin{enumerate}
\item We assumed the function extensionality axiom in \cref{axiom:funext}. Function extensionality was later derived from the univalence axiom in \cref{thm:funext-univalence}.
\item We assumed the univalence axiom in \cref{axiom:univalence}.
\item We assumed that universes are closed under propositional truncations in \cref{axiom:propositiona-truncations}. It was shown in \cref{thm:join-construction-propositional-truncations} that any universe that is closed under homotopy pushouts is also closed under propositional truncations.
\item We assumed the type theoretic replacement axiom in \cref{axiom:replacement}. It was shown in \cref{thm:replacement} that replacement holds for any universe that is closed under homotopy pushouts.
\item We assumed in \cref{axiom:circle} that the circle is a type in the base universe $\UU_0$. In \cref{thm:circle-replacement} the circle was shown to be a type in the base universe using the replacement axiom. It was shown that the In \cref{thm:circle-pushout}, the circle was constructed as a pushout.
\item In \cref{axiom:pushouts} that universes are closed under pushouts.
\end{enumerate}

\printbibliography

\printindex

\end{document}
